% Options for packages loaded elsewhere
\PassOptionsToPackage{unicode}{hyperref}
\PassOptionsToPackage{hyphens}{url}
%
\documentclass[
]{article}
\usepackage{amsmath,amssymb}
\usepackage{lmodern}
\usepackage{iftex}
\ifPDFTeX
  \usepackage[T1]{fontenc}
  \usepackage[utf8]{inputenc}
  \usepackage{textcomp} % provide euro and other symbols
\else % if luatex or xetex
  \usepackage{unicode-math}
  \defaultfontfeatures{Scale=MatchLowercase}
  \defaultfontfeatures[\rmfamily]{Ligatures=TeX,Scale=1}
\fi
% Use upquote if available, for straight quotes in verbatim environments
\IfFileExists{upquote.sty}{\usepackage{upquote}}{}
\IfFileExists{microtype.sty}{% use microtype if available
  \usepackage[]{microtype}
  \UseMicrotypeSet[protrusion]{basicmath} % disable protrusion for tt fonts
}{}
\makeatletter
\@ifundefined{KOMAClassName}{% if non-KOMA class
  \IfFileExists{parskip.sty}{%
    \usepackage{parskip}
  }{% else
    \setlength{\parindent}{0pt}
    \setlength{\parskip}{6pt plus 2pt minus 1pt}}
}{% if KOMA class
  \KOMAoptions{parskip=half}}
\makeatother
\usepackage{xcolor}
\usepackage{color}
\usepackage{fancyvrb}
\newcommand{\VerbBar}{|}
\newcommand{\VERB}{\Verb[commandchars=\\\{\}]}
\DefineVerbatimEnvironment{Highlighting}{Verbatim}{commandchars=\\\{\}}
% Add ',fontsize=\small' for more characters per line
\newenvironment{Shaded}{}{}
\newcommand{\AlertTok}[1]{\textcolor[rgb]{1.00,0.00,0.00}{\textbf{#1}}}
\newcommand{\AnnotationTok}[1]{\textcolor[rgb]{0.38,0.63,0.69}{\textbf{\textit{#1}}}}
\newcommand{\AttributeTok}[1]{\textcolor[rgb]{0.49,0.56,0.16}{#1}}
\newcommand{\BaseNTok}[1]{\textcolor[rgb]{0.25,0.63,0.44}{#1}}
\newcommand{\BuiltInTok}[1]{\textcolor[rgb]{0.00,0.50,0.00}{#1}}
\newcommand{\CharTok}[1]{\textcolor[rgb]{0.25,0.44,0.63}{#1}}
\newcommand{\CommentTok}[1]{\textcolor[rgb]{0.38,0.63,0.69}{\textit{#1}}}
\newcommand{\CommentVarTok}[1]{\textcolor[rgb]{0.38,0.63,0.69}{\textbf{\textit{#1}}}}
\newcommand{\ConstantTok}[1]{\textcolor[rgb]{0.53,0.00,0.00}{#1}}
\newcommand{\ControlFlowTok}[1]{\textcolor[rgb]{0.00,0.44,0.13}{\textbf{#1}}}
\newcommand{\DataTypeTok}[1]{\textcolor[rgb]{0.56,0.13,0.00}{#1}}
\newcommand{\DecValTok}[1]{\textcolor[rgb]{0.25,0.63,0.44}{#1}}
\newcommand{\DocumentationTok}[1]{\textcolor[rgb]{0.73,0.13,0.13}{\textit{#1}}}
\newcommand{\ErrorTok}[1]{\textcolor[rgb]{1.00,0.00,0.00}{\textbf{#1}}}
\newcommand{\ExtensionTok}[1]{#1}
\newcommand{\FloatTok}[1]{\textcolor[rgb]{0.25,0.63,0.44}{#1}}
\newcommand{\FunctionTok}[1]{\textcolor[rgb]{0.02,0.16,0.49}{#1}}
\newcommand{\ImportTok}[1]{\textcolor[rgb]{0.00,0.50,0.00}{\textbf{#1}}}
\newcommand{\InformationTok}[1]{\textcolor[rgb]{0.38,0.63,0.69}{\textbf{\textit{#1}}}}
\newcommand{\KeywordTok}[1]{\textcolor[rgb]{0.00,0.44,0.13}{\textbf{#1}}}
\newcommand{\NormalTok}[1]{#1}
\newcommand{\OperatorTok}[1]{\textcolor[rgb]{0.40,0.40,0.40}{#1}}
\newcommand{\OtherTok}[1]{\textcolor[rgb]{0.00,0.44,0.13}{#1}}
\newcommand{\PreprocessorTok}[1]{\textcolor[rgb]{0.74,0.48,0.00}{#1}}
\newcommand{\RegionMarkerTok}[1]{#1}
\newcommand{\SpecialCharTok}[1]{\textcolor[rgb]{0.25,0.44,0.63}{#1}}
\newcommand{\SpecialStringTok}[1]{\textcolor[rgb]{0.73,0.40,0.53}{#1}}
\newcommand{\StringTok}[1]{\textcolor[rgb]{0.25,0.44,0.63}{#1}}
\newcommand{\VariableTok}[1]{\textcolor[rgb]{0.10,0.09,0.49}{#1}}
\newcommand{\VerbatimStringTok}[1]{\textcolor[rgb]{0.25,0.44,0.63}{#1}}
\newcommand{\WarningTok}[1]{\textcolor[rgb]{0.38,0.63,0.69}{\textbf{\textit{#1}}}}
\usepackage{longtable,booktabs,array}
\usepackage{calc} % for calculating minipage widths
% Correct order of tables after \paragraph or \subparagraph
\usepackage{etoolbox}
\makeatletter
\patchcmd\longtable{\par}{\if@noskipsec\mbox{}\fi\par}{}{}
\makeatother
% Allow footnotes in longtable head/foot
\IfFileExists{footnotehyper.sty}{\usepackage{footnotehyper}}{\usepackage{footnote}}
\makesavenoteenv{longtable}
\setlength{\emergencystretch}{3em} % prevent overfull lines
\providecommand{\tightlist}{%
  \setlength{\itemsep}{0pt}\setlength{\parskip}{0pt}}
\setcounter{secnumdepth}{-\maxdimen} % remove section numbering
\usepackage{enumitem}
\setlistdepth{9}

\setlist[itemize,1]{label=$\bullet$}
\setlist[itemize,2]{label=$\bullet$}
\setlist[itemize,3]{label=$\bullet$}
\setlist[itemize,4]{label=$\bullet$}
\setlist[itemize,5]{label=$\bullet$}
\setlist[itemize,6]{label=$\bullet$}
\setlist[itemize,7]{label=$\bullet$}
\setlist[itemize,8]{label=$\bullet$}
\setlist[itemize,9]{label=$\bullet$}
\renewlist{itemize}{itemize}{9}

\setlist[enumerate,1]{label=$\arabic*.$}
\setlist[enumerate,2]{label=$\alph*.$}
\setlist[enumerate,3]{label=$\roman*.$}
\setlist[enumerate,4]{label=$\arabic*.$}
\setlist[enumerate,5]{label=$\alpha*$}
\setlist[enumerate,6]{label=$\roman*.$}
\setlist[enumerate,7]{label=$\arabic*.$}
\setlist[enumerate,8]{label=$\alph*.$}
\setlist[enumerate,9]{label=$\roman*.$}
\renewlist{enumerate}{enumerate}{9}

\PassOptionsToPackage{dvipsnames,svgnames,x11names}{xcolor} 
\ifLuaTeX
  \usepackage{selnolig}  % disable illegal ligatures
\fi
\IfFileExists{bookmark.sty}{\usepackage{bookmark}}{\usepackage{hyperref}}
\IfFileExists{xurl.sty}{\usepackage{xurl}}{} % add URL line breaks if available
\urlstyle{same} % disable monospaced font for URLs
\hypersetup{
  hidelinks,
  pdfcreator={LaTeX via pandoc}}

\author{}
\date{}

\begin{document}

\hypertarget{discrete-mathematics-exercise-solution-suggestions}{%
\section{Discrete Mathematics Exercise Solution
Suggestions}\label{discrete-mathematics-exercise-solution-suggestions}}

This is a collection of solution suggestions for the exercises of the
course Discrete Mathematics from 2021. We do not guarantee their
correctness.

\hypertarget{exercise-sheet-1}{%
\subsection{Exercise Sheet 1}\label{exercise-sheet-1}}

\hypertarget{use-a-suitable-graph-theoretical-model-to-solve-the-following-problems}{%
\paragraph{1) Use a suitable graph theoretical model to solve the
following
problems:}\label{use-a-suitable-graph-theoretical-model-to-solve-the-following-problems}}

\textbf{a) Show that in every city at least two of its inhabitants have
the same number of neighbours!} Let \(G = (V,E)\) such that *
\(v \in V\) if \(v\) is an inhabitant of the city and * \((v,w) \in E\)
iff \(v\) and \(w\) are neighbours. Now, we show the (a) by proving that
in any simple, undirected graph there exist two vertices which have the
same degree: -- Proof by contradiction--

Assume a simple, undirected graph \(G = (V,E)\), such that
\(\forall v,w \in V: v \neq w \rightarrow deg(v) \neq deg(w)\). Then, by
assumption, the nodes \(\{v_1,\dots , v_n\} = V\), have degrees
\(\{0, \dots, n-1\}\) (seeing as \(G\) is simple, no node has an edge to
itself).

Thus, there is a node \(v_i \in V\) with \(deg(v_i) = n-1\) meaning it
is connected to all other nodes, but there also exists a node
\(v_j \in V\) with \(deg(v_j) = 0\) meaning it is connected to no other
nodes, including \(v_i\). \emph{Contradiction!}

\textbf{b) 11 friends want to send postcards according to the following
rules:}

\begin{itemize}
\item
  \begin{enumerate}
  \def\labelenumi{(\roman{enumi})}
  \tightlist
  \item
    Each person sends and receives exactly 3 cards.
  \end{enumerate}
\item
  \begin{enumerate}
  \def\labelenumi{(\roman{enumi})}
  \setcounter{enumi}{1}
  \tightlist
  \item
    Each one only receives cards from those to whom he or she sent a
    card and vice versa. Tell us how this can be done or prove that this
    is impossible.
  \end{enumerate}
\end{itemize}

Let \(G=(V,E)\) a graph such that the nodes
\(V = \{v_1, \dots, v_{11}\}\) represent the 11 friends and there is an
edge \(v,w \in E\) if \(v\) sends a card to \(w\) and vice versa.

--Proof that this is impossible-- By the Handshaking Lemma,
\(\sum_{v \in V} deg(v) = 2*|E|\) holds for any finite, undirected
graph. By (i), for each vertice \(v \in V\), \(deg(v) = 3\) and by
assumption \(|V| = 11\). If we apply this to the handshaking lemma:
\(\sum_{v \in V} deg(v) = 33 = 2* k\), where \(k = |E|\). But this holds
for no integer amount of edges. \emph{Contradiction!}

\textbf{c) Determine all graphs in which all vertices have degree 1.}

In order for all vertices in a graph to have degree 1, each edge in the
graph has to be connected to exactly two vertices, which themselves may
not be connected to any other vertices. Therefore we can construct the
set of all graphs with only vertices of degree 1, based on their number
of edges. For \(G = (V,E)\), with \(|E| = k, k \in \mathbb{N}\): \[
E = \{e_1, \dots , e_k\},\\
V = \{v_{i1}, v_{i2} | e_i \in E\}
\]

\hypertarget{compute-the-number-of-walks-of-length-l-from-i-to-j-in-the-following-graph}{%
\paragraph{\texorpdfstring{2) Compute the number of walks of length
\(l\) from \(i\) to \(j\) in the following
graph:}{2) Compute the number of walks of length l from i to j in the following graph:}}\label{compute-the-number-of-walks-of-length-l-from-i-to-j-in-the-following-graph}}

\begin{Shaded}
\begin{Highlighting}[]
\NormalTok{flowchart LR}
\NormalTok{    1((1)):::myClass {-}{-}{-} 2((2)):::myClass {-}{-}{-} 3((3)):::myClass}
\NormalTok{    classDef myClass fill:\#999; }
\end{Highlighting}
\end{Shaded}

The number of walks of length \(l\) from \(i\) to \(j\) is defined by
the sum of all values in \(A(G)^l\), where \[A(G) = \begin{pmatrix}
0 & 1 & 0\\
1 & 0 & 1\\
0 & 1 & 0
\end{pmatrix}\] The element \((i,j)\) in \(A(G)^l\) represents the walks
of length \(l\) from vertex \(i\) to vertex \(j\).

\textbf{How could you use the adjacency matrix to compute the number of
triangles (cycles of length 3) in a (loopless) graph?} The number of
triangles is defined by the trace (the sum of elements on the main
diagonal) of \(A(G)^3\) divided by 6 for undirected, and 3 for directed
graphs.

\textbf{Perform the computation for two graphs of your choice on four
vertices.}

\hypertarget{show-that-each-of-the-following-four-statements-is-equivalent-to-the-statement-t-is-a-tree}{%
\paragraph{3) Show that each of the following four statements is
equivalent to the statement ``T is a
tree'':}\label{show-that-each-of-the-following-four-statements-is-equivalent-to-the-statement-t-is-a-tree}}

\begin{itemize}
\item
  \begin{enumerate}
  \def\labelenumi{(\arabic{enumi})}
  \tightlist
  \item
    Every two nodes of \(T\) are connected by exactly one path.
  \end{enumerate}
\item
  \begin{enumerate}
  \def\labelenumi{(\arabic{enumi})}
  \setcounter{enumi}{1}
  \tightlist
  \item
    T is connected and \(\alpha_0(T) = \alpha_1(T) + 1\).
  \end{enumerate}
\item
  \begin{enumerate}
  \def\labelenumi{(\arabic{enumi})}
  \setcounter{enumi}{2}
  \tightlist
  \item
    T is a minimal, connected graph, i.e., deleting an edge destroys
    connectivity.
  \end{enumerate}
\item
  \begin{enumerate}
  \def\labelenumi{(\arabic{enumi})}
  \setcounter{enumi}{3}
  \tightlist
  \item
    T is a maximal, acyclic graph, i.e., adding an edge generates a
    cycle.
  \end{enumerate}
\end{itemize}

Let (0) := ``T is a tree'', i.e.~\emph{A graph is a tree iff it is
connected, acyclic, and undirected.}

\begin{itemize}
\item
  \begin{enumerate}
  \def\labelenumi{(\arabic{enumi})}
  \setcounter{enumi}{1}
  \tightlist
  \item
    -\textgreater{} (0)
  \end{enumerate}

  \begin{itemize}
  \tightlist
  \item
    Connected: holds by assumption.
  \item
    By the pidgeonhole principle, we can find a node \(l \in T\) , which
    is connected only to one edge, i.e.~a leaf node, since there are
    \(n-1\) edges and \(n\) nodes, not all nodes can be connected to two
    edges.
  \item
    We proceed by induction on the number of nodes:

    \begin{itemize}
    \tightlist
    \item
      n = 2: then, there exist two nodes connected by one edge, thus
      \(T\) is acyclic.
    \item
      IH: For \(T\) with \(n\) nodes and \(n-1\) edges, \(T\) is
      acyclic.
    \item
      Induction step: Let \(T\) be a graph that satisfies (2), with
      \(n+1\) nodes and \(n\) edges. Then there exists a node
      \(l \in T\) which is connected only to one edge, we know that the
      subtree \(T'\) of \(T\) with \(l\) and the edge connected to \(l\)
      removed is acyclic by IH, as we removed exactly one node and one
      edge, thus \(T'\) has \(n\) nodes and \(n-1\) edges. Additionally
      \(T'\) with \(l\) attached to one of its nodes, cannot introduce a
      cycle, as \(l\) is a leaf.
    \end{itemize}
  \item
    Thus, \(T\) is acyclic.
  \end{itemize}
\item
  \begin{enumerate}
  \def\labelenumi{(\arabic{enumi})}
  \setcounter{enumi}{-1}
  \tightlist
  \item
    -\textgreater{} (1)
  \end{enumerate}

  \begin{itemize}
  \tightlist
  \item
    each \(x,y \in V\) are connected by at least one path: by connected

    \begin{itemize}
    \tightlist
    \item
      each \(x,y \in V\) are connected by at most one path: by acyclic
    \end{itemize}
  \end{itemize}
\item
  \begin{enumerate}
  \def\labelenumi{(\arabic{enumi})}
  \tightlist
  \item
    -\textgreater{} (3)
  \end{enumerate}

  \begin{itemize}
  \tightlist
  \item
    Connected: holds by assumption.
  \item
    Minimal: Let \((x,y)\) be an arbitrary edge in \(T\), if we remove
    this edge, then, by (1), there no longer exists a path connecting
    \(x\) and \(y\) in \(T\), thus connectivity of \(T\) is destroyed.
    Since, this holds for an arbitrary edge, \(T\) is minimal.
  \end{itemize}
\item
  \begin{enumerate}
  \def\labelenumi{(\arabic{enumi})}
  \setcounter{enumi}{2}
  \tightlist
  \item
    -\textgreater{} (4)
  \end{enumerate}

  \begin{itemize}
  \tightlist
  \item
    Acyclic: Assume \(T\) contains a cycle. Then for some vertices
    \(v_1,v_2 \in V\), w.l.o.g. assume \((v_1,v_2) \in E\) and there
    exists a path \(v_2 \leftrightarrow y \leftrightsquigarrow v_1\). If
    we now delete \((v_1,v_2)\) from \(E\), connectivity is maintained.
  \item
    Maximal acyclic: For an arbitrary two nodes \(v_1,v_2 \in V\) that
    are not connected by a direct edge, they must be connected by some
    path \(P := v_1 \leftrightarrow y \leftrightsquigarrow v_2\) by
    \(T\) connected. If we add a new edge \((v_1,v_2)\) to \(E\), we
    create a cycle.

    \begin{itemize}
    \tightlist
    \item
      Thus adding a new edge to \(T\) always creats a cycle.
    \end{itemize}
  \end{itemize}
\item
  \begin{enumerate}
  \def\labelenumi{(\arabic{enumi})}
  \setcounter{enumi}{3}
  \tightlist
  \item
    -\textgreater{} (2)
  \end{enumerate}

  \begin{itemize}
  \tightlist
  \item
    Let \(T\) be a maximal acyclic graph.
  \item
    \emph{connected:} assume \(T\) is not connected and maximal acyclic,
    then by \(T\) not being connected there are two nodes
    \(v_1,v_2 \in T\) between which there is no path. Adding an edge
    \((v_1,v_2)\) has to create a cycle by \(T\) maximal acyclic, but
    \((v_1,v_2)\) is the only path between \(v_1,v_2\) by construction.
    Contradiction.

    \begin{itemize}
    \tightlist
    \item
      Thus, \(T\) is connected.
    \end{itemize}
  \item
    \(\alpha_0(T) = \alpha_1(T) + 1\):

    \begin{itemize}
    \tightlist
    \item
      Assume \(T\) is the smallest maximal acyclic graph such that
      \(|V| > |E| + 1\)

      \begin{itemize}
      \tightlist
      \item
        trivially \(|V| \neq 1\) and by the pidgeonhole principle
        \(\exists l \in T\) such that \(l\) is a leaf.
      \item
        Let \(T' = T \backslash \{l\}\) which removes exactly one edge
        and one node from \(T\) by \(l\) leaf. Thus, \(|V| > |E| + 1\)
        holds for T'.
      \item
        Now, assume \(T'\) not maximal acyclic, then
        \(\exists (x_1,x_2)\) with \(x_1,x_2 \in T'\). and adding
        \((x_1,x_2)\) to \(T'\) creates a cycle in \(T'\). But by
        construction of \(T'\), adding \((x_1,x_2)\) to \(T\) would
        create a cycle in \(T\) as well. Contradiction since \(T\) is
        maximal acyclic.
      \item
        Thus \(T\) cannot be the smallest acyclic graph since
        \(T' \subset T\) and \(T'\) maximal acyclic.
      \end{itemize}
    \item
      Assume \(T\) is the smallest maximal acyclic graph such that
      \(|V| < |E| + 1\)

      \begin{itemize}
      \tightlist
      \item
        trivially \(|V| \neq 1\), we distinguish as follows:

        \begin{itemize}
        \tightlist
        \item
          case 1: \(\exists l \in T\) such that \(l\) is a leaf.

          \begin{itemize}
          \tightlist
          \item
            Let \(T' = T \backslash \{l\}\), then \(T'\) is maximal
            acyclic as shown previously and \(V' < E' + 1\) as
            \(V'=V-1\) and \(E' = E-1\).
          \item
            Thus, \(T' \subset T\) and \(T'\) maximal acyclic.
            Contradiction.
          \end{itemize}
        \item
          case 2: No leaf node exists in \(T\).

          \begin{itemize}
          \tightlist
          \item
            We construct a path \(p\) by:

            \begin{itemize}
            \tightlist
            \item
              selecting any node \(v_1\) and choosing some neighbour of
              \(v_1\) to add to \(p\)
            \item
              choosing a neighbour of the current node which we did not
              visit in the last step and adding it to \(p\)
            \item
              repeating this process iteratively. Since there are no
              leaves in \(T\), this process can be repeated infinitely.
            \item
              By the pidgeonhole principle, some node \(x\) repeats in
              \(p\),
              i.e.~\(v_1 \rightsquigarrow x \rightsquigarrow x \rightsquigarrow \dots\),
              from this we construct a cycle
              \(c := x \rightsquigarrow x\). Contradiction.
            \end{itemize}
          \end{itemize}
        \end{itemize}
      \end{itemize}
    \end{itemize}
  \item
    Thus T is the smallest maximal acyclic graph, such that
    \(|V| = |E|+1\).
  \end{itemize}
\end{itemize}

Since we showed, (2) -\textgreater{} (0) -\textgreater{} (1)
-\textgreater{} (3) -\textgreater{} (4) -\textgreater{} (2), we have
shown equivalence of all statements by transitivity of implication.
\#\#\#\# 4) Prove that the edge set of an undirected, simple graph can
be partitioned into cycles iff every vertex set has even degree.
\emph{Hint:} To prove the existence of a cycle, consider a maximal path
and use the even degree condition.

\begin{itemize}
\item
  =\textgreater{} Let \(G = (V,E)\) be a graph such that \(E\) can be
  partitioned into cycles \(C = \{C_1, \dots, C_k\}\). Then for an
  arbitrary vertex \(v\), the degree is defined as follows: Let
  \(d(v)_G = 0\), then for every cycle \(C_i \in C\), - if
  \(v \in C_i\), \(d(v)_{G \cup C_i} = d(v)_{G \cup C_{i-1}} + 2\) -
  else \(d(v)_{G \cup C_i} = d(v)_{G \cup C_{i-1}}\). Since we start
  with even degree and in each step the degree either remains unchanged
  or is increased by 2, the degree of all nodes is even.
\item
  \textless= Let \(G = (V,E)\) be a graph, such that every vertex
  \(v \in V\) has even degree.

  --Proof by induction--

  \begin{itemize}
  \tightlist
  \item
    Base Case: n=1

    \begin{itemize}
    \tightlist
    \item
      Since \(|V| = 1\) and the graph is simple by assumption,
      \(E = \{\}\), thus the partition is simply the empty set.
    \end{itemize}
  \item
    Induction Hypothesis: A graph with \(|V| < n\) can have its edge set
    partitioned into cycles.
  \item
    Induction step:

    \begin{itemize}
    \tightlist
    \item
      Assume an arbitrary graph \(G = (V,E)\) with \(|V| = n\).

      \begin{itemize}
      \tightlist
      \item
        case 1: \(G\) is not connected

        \begin{itemize}
        \tightlist
        \item
          Then \(G = G_1 \dot \cup G_2\) and by induction hypothesis
          both \(G_1\) and \(G_2\) can be partitioned into cycles. as
          their number of vertices are both strictly smaller than \(n\).
        \end{itemize}
      \item
        case 2: \(G\) is connected

        \begin{enumerate}
        \def\labelenumi{\arabic{enumi}.}
        \tightlist
        \item
          Then for an arbitrary vertex \(v_1 \in V\), we know
          \(d(v_1) \geq 2\) since \(G\) is connected and all vertices in
          \(G\) have even degree.
        \item
          Therefore, there exists a vertex \(v_2 \in V\) which is
          connected to \(v_1\) and \(v_1 \neq v2\).
        \item
          Additionally, \(deg(v_2) \geq 2\) by assumption and \(G\)
          being connected.
        \item
          We can apply expansion step 2. an arbitrary number of times
          until we reach a vertex \(v_i\) for which \(v_i = v_j\) holds
          for some \(j \in \{1, \dots, i-2\}\) (i.e.~a node we already
          expanded).
        \item
          We furthermore argue that the amound of expansion steps needed
          to find such a node/cycle is at most \(n-1\) as there are only
          \(n\) vertices in \(G\) and thus after \(n-1\) steps a node
          already on the path has to be encountered.
        \end{enumerate}

        After, such a cycle
        \(C = v_1 \rightarrow v_2 \rightarrow \dots \rightarrow v_1\) is
        found, we can define a subgraph
        \(G' = G \backslash C = (V',E')\). We then repeat steps 1. to 5.
        until there only remain nodes with no edges connected to them,
        which are then removed to construct \(G'\). Now, \(V' < n\)
        holds for \(G'\) and thus \(G'\) is partitionable into a set of
        cycles \(C\) by IH. We can thus partition \(G\) by
        \(C_1 \dot \cup C\).
      \end{itemize}
    \end{itemize}
  \end{itemize}
\end{itemize}

\hypertarget{let-g-ve-be-an-undirected-graph-with-n-vertices-which-does-not-have-any-cycle-of-length-3.-prove}{%
\paragraph{\texorpdfstring{5) Let \(G = (V,E)\) be an undirected graph
with \(n\) vertices which does not have any cycle of length 3.
Prove:}{5) Let G = (V,E) be an undirected graph with n vertices which does not have any cycle of length 3. Prove:}}\label{let-g-ve-be-an-undirected-graph-with-n-vertices-which-does-not-have-any-cycle-of-length-3.-prove}}

\textbf{1. If \(xy \in E\) then \(d(x) + d(y) \leq n\).} Let
\(G = (V,E)\) be an undirected graph such that \(|V| = n\) which
contains no cycles of length 3. --Proof by contradiction-- Assume two
nodes \(x,y \in V\) such that \(d(x) + d(y) > n\), then the nodes \(x\)
and \(y\) are connected to at least \(n-1\) other nodes (since they are
also connected by an edge themselves). Now,
\(|V \backslash \{x,y\}| = n-2\) by assumption. Thus, there exists at
least one node \(z \in V\) such that \(xz \in E\) and \(yz \in E\) which
means \(G\) contains a cycle of length 3. Contradiction! \textbf{2. The
previous inequality \(d(x) + d(y) \leq n\) implies that
\(\sum_{v \in V} d(v)^2 \leq n|E|\).} Since forall edges \(xy \in E\),
\(d(x) + d(y) \leq n\), we know
\(n*|E| \geq \sum_{vw \in E} d(v) + d(w)\). Now:
\(\sum_{vw \in E} d(v) + d(w) = \sum_{v \in V} \sum_{1}^{d(v)} d(v)\)
since for each vertice, its degree is contained in the left sum as many
times as it occurs in an edge (i.e.~as many times as its degree).
Additionally,
\(\sum_{v \in V} \sum_{1}^{d(v)} d(v) = \sum_{v \in V} d(v)^2\) seeing
as the left side simply sums up \(d(v)\) times \(d(v)\). Now we can
apply transitivity of equality and get
\(n* |E| \geq \sum_{v \in V} d(v)^2\) \textbf{3. The graph has at most
\(n^2/4\) edges. \emph{Hint:} Use the Handshaking-Lemma, the
Cauchy-Schwarz inequality
\((\sum_{i=1}^r a_i b_i)^2 \leq (\sum_{i=1}^r a_i^2) (\sum_{i=1}^r b_i^2)\),
and what you have proved so far.} Assume
\(\sum_{v \in V} d(v)^2 \leq |E| * n\), we apply the Cauchy-Schwarz
inequality with \(a = d(v)\) and \(b=1\) to get
\((\sum_{v \in V} d(v) * 1)^2 \leq (\sum_{v \in V} d(v)^2) * (\sum_1^n 1),\)
i.e.~ \((\sum_{v \in V} d(v))^2 \leq \sum_{v \in V} d(v)^2 * n\) Now,
this can be inserted into the Handshaking Lemma,
\(4*|E|^2 = (\sum_{v \in V} d(v))^2 \leq n * \sum_{v \in V} d(v)^2\) And
in turn, from exercise 5.2):
\(4*|E|^2 = (\sum_{v \in V} d(v))^2 \leq n * \sum_{v \in V} d(v)^2 \leq n^2 * |E|\)
we divide by \(n\) and remove the sandwiched inequalities:
\(\frac{4*|E|^2}{n} \leq n * |E|\) now only simple arithmetic
transformations:
\(\frac{4*|E|^2}{n} \leq n * |E| \Leftrightarrow \frac{4m}{n} \leq n \Leftrightarrow 4m \leq n^2 \Leftrightarrow m \leq \frac{n^2}{4}\)

\hypertarget{let-g-ve-and-g-ve-be-two-undirected-graphs.-a-graph-isomorphism-is-a-bijective-mapping-phi-v-rightarrow-v-such-that-two-vertices-xy-in-v-are-adjacent-iff-phix-and-phiy-are-adjacent.-the-two-graphs-g-and-g-are-called-isomorphic-if-there-exists-an-isomorphism-phi-v-rightarrow-v.-prove-the-following-statements}{%
\paragraph{\texorpdfstring{6) Let \(G = (V,E)\) and \(G' = (V',E')\) be
two undirected graphs. A graph isomorphism is a bijective mapping
\(\phi : V \rightarrow V'\) such that two vertices \(x,y \in V\) are
adjacent iff \(\phi(x)\) and \(\phi(y)\) are adjacent. The two graphs
\(G\) and \(G'\) are called isomorphic, if there exists an isomorphism
\(\phi: V \rightarrow V'\). Prove the following
statements:}{6) Let G = (V,E) and G\textquotesingle{} = (V\textquotesingle,E\textquotesingle) be two undirected graphs. A graph isomorphism is a bijective mapping \textbackslash phi : V \textbackslash rightarrow V\textquotesingle{} such that two vertices x,y \textbackslash in V are adjacent iff \textbackslash phi(x) and \textbackslash phi(y) are adjacent. The two graphs G and G\textquotesingle{} are called isomorphic, if there exists an isomorphism \textbackslash phi: V \textbackslash rightarrow V\textquotesingle. Prove the following statements:}}\label{let-g-ve-and-g-ve-be-two-undirected-graphs.-a-graph-isomorphism-is-a-bijective-mapping-phi-v-rightarrow-v-such-that-two-vertices-xy-in-v-are-adjacent-iff-phix-and-phiy-are-adjacent.-the-two-graphs-g-and-g-are-called-isomorphic-if-there-exists-an-isomorphism-phi-v-rightarrow-v.-prove-the-following-statements}}

\begin{itemize}
\item
  If \(G\) and \(G'\) are isomorphic graphs and
  \(\phi : V \rightarrow V'\) is an isomorphism, then
  \(d_G(x) = d_G'(\phi(x))\) forall \(x \in V\). --Proof by
  contradiction-- Let \(x \in V\) be a vertice such that
  \(d_G(x) \neq d_G(\phi(x))\).

  \begin{itemize}
  \tightlist
  \item
    Case 1: There exists some node \(y \in V\) such that \(\phi(y)\) is
    adjacent to \(\phi(x)\) but \(y\) is not adjacent to (x), this
    contradicts the definition of \(\phi\).
  \item
    Case 2: There exists some node \(y \in V\) such that \(x\) is
    adjacent to y, but \(\phi(x)\) is not adjacent to \(\phi(y)\). This
    also contradicts the definition of \(\phi\). Since we arrive at a
    contradiction in both cases, \(d_G(x) = d_G(\phi(x))\) has to hold
    forall \(x \in V\).
  \end{itemize}
\item
  If \(\phi : V \rightarrow V'\) is a bijective mapping satisfying
  \(d_G(x) = d_G(\phi(x))\) forall \(x \in V\), then \(G\) and \(G'\)
  are not necessarily isomorphic. We provide \(G\), \(G'\) and \(\phi\)
  such that \(d_G(x) = d_G(\phi(x))\), but \(G\) and \(G'\) are not
  isomorphic.

  G: ```mermaid flowchart LR

\begin{verbatim}
  1((v1)):::myClass --- 2((v2)):::myClass --- 3((v3)):::myClass;
  3 --- 1
  4((v4)):::myClass --- 5((v5)):::myClass --- 6((v6)):::myClass;
  6 --- 4
  classDef myClass fill:#999; 
\end{verbatim}

\begin{verbatim}
G':
```mermaid
flowchart LR

    1((v1')):::myClass --- 2((v2')):::myClass --- 3((v3')):::myClass;
    3 --- 4
    4((v4')):::myClass --- 5((v5')):::myClass --- 6((v6')):::myClass;
    6 --- 1
    classDef myClass fill:#999; 
\end{verbatim}

  and \(\phi(v_i) = v'_i\) for each \(v_i \in V\).

  This means, forall \(v_i \in V\), \(d_G(x) = d_G(\phi(x))\) as all
  vertices in both graphs have degree 2.

  But \(G\) and \(G'\) are not isomorphic, since \(v_5\) and \(v_4\) are
  adjacent but \(v_4'\) and \(v_5'\) are not.
\end{itemize}

\hypertarget{are-the-following-two-graphs-isomorphic}{%
\paragraph{7) Are the following two graphs
isomorphic?}\label{are-the-following-two-graphs-isomorphic}}

\begin{Shaded}
\begin{Highlighting}[]
\NormalTok{graph G1 \{ }
\NormalTok{    \{rank=same;4,3\}}
\NormalTok{    \{rank=same;b,c\}}
\NormalTok{    \{rank=same;a,d\}}
\NormalTok{    \{rank=same;1,2\}}
\NormalTok{    rankdir=TD;}
\NormalTok{    4,3;}
\NormalTok{    b,c;}
\NormalTok{    a,d;}
\NormalTok{    1,2;}
\NormalTok{        4 {-}{-} \{3\}[minlen=8];}
\NormalTok{        4 {-}{-} \{b, 1\};}
\NormalTok{        3 {-}{-} \{c, 2\};}
\NormalTok{            b {-}{-} \{c, a\}; }
\NormalTok{            c {-}{-} \{d\};}
\NormalTok{            a {-}{-} \{1, d\};}
\NormalTok{            d {-}{-} \{2\};}
\NormalTok{        1 {-}{-} 2[minlen=8];}
\NormalTok{    \}}
\end{Highlighting}
\end{Shaded}

\begin{Shaded}
\begin{Highlighting}[]
\NormalTok{graph G2 \{ }
\NormalTok{    rankdir=TD;}
\NormalTok{        A, B, C, D;}
\NormalTok{        A {-}{-} \{\textless{}\&\#946;\textgreater{}, \textless{}\&\#947;\textgreater{}, \textless{}\&\#948;\textgreater{}\};}
\NormalTok{        B {-}{-} \{\textless{}\&\#945;\textgreater{},\textless{}\&\#947;\textgreater{}, \textless{}\&\#948;\textgreater{}\};}
\NormalTok{        C {-}{-} \{\textless{}\&\#945;\textgreater{}, \textless{}\&\#946;\textgreater{}, \textless{}\&\#948;\textgreater{}\};}
\NormalTok{        D {-}{-} \{\textless{}\&\#945;\textgreater{}, \textless{}\&\#946;\textgreater{}, \textless{}\&\#947;\textgreater{}\};}
\NormalTok{        \textless{}\&\#945;\textgreater{}, \textless{}\&\#946;\textgreater{}, \textless{}\&\#947;\textgreater{}, \textless{}\&\#948;\textgreater{};}
\NormalTok{    \}}
\end{Highlighting}
\end{Shaded}

Yes, we provide the following isomorphism: \(f(1) = A\), \(f(3) = B\),
\(f(b) = C\), \(f(d) = D\), \(f(c) = \alpha\), \(f(a) = \beta\),
\(f(2) = \gamma\), \(f(4) = \delta\), which is edge preserving and
bijective. \#\#\#\# 8) Let \(G = (V,E)\) be a simple graph. Moreover let
\(G_R\) be its reduction. Prove that \(G_R\) is acyclic. --Proof by
contradiction-- Let \(G_R\) be a graph reduction which contains a cycle
\(C\), then \(C\) is a walk \(V_1^R \rightarrow V_2^R \leadsto V_1^R\).
Thus, there exist strongly connected components \(V_1, V_2\) in \(G\)
such that there is a walk from some \(v_1 \in V_1\) to some
\(v_2 \in V_2\) and from \(v_2\) to \(v_1\). By construction of \(G_R\)
this means that \(v_1\) and \(v_2\) have to be in the same component in
\(G_R\). Contradiction!

\hypertarget{find-the-strongly-connected-components-and-the-reduction-g_r-of-the-graph-g-below.-furthermore-determine-all-vertex-bases-of-g.}{%
\paragraph{\texorpdfstring{9) Find the strongly connected components and
the reduction \(G_R\) of the graph \(G\) below. Furthermore, determine
all vertex bases of
\(G\).}{9) Find the strongly connected components and the reduction G\_R of the graph G below. Furthermore, determine all vertex bases of G.}}\label{find-the-strongly-connected-components-and-the-reduction-g_r-of-the-graph-g-below.-furthermore-determine-all-vertex-bases-of-g.}}

\begin{Shaded}
\begin{Highlighting}[]
\NormalTok{digraph G \{ }
\NormalTok{    \{rank=same;1,2,7,8,3,4\}}
\NormalTok{    \{rank=same;5,6,11,12,9,10\}}
\NormalTok{    \{rank=same;13,14;\}}
\NormalTok{    rankdir=DT;}
\NormalTok{        1 {-}\textgreater{} 5 [style = invis];}
\NormalTok{        2 {-}\textgreater{} 6 [style = invis];}
\NormalTok{        7 {-}\textgreater{} 11 [style = invis];}
\NormalTok{        8 {-}\textgreater{} 12 [style = invis];}
\NormalTok{        3 {-}\textgreater{} 9 [style = invis];}
\NormalTok{        4 {-}\textgreater{} 10 [style = invis];}
\NormalTok{        11 {-}\textgreater{} 13 [style = invis];}
\NormalTok{        12 {-}\textgreater{} 14 [style = invis];}
\NormalTok{        10 {-}\textgreater{} 4 [style = invis];}
\NormalTok{        1 {-}\textgreater{} 2;}
\NormalTok{        2 {-}\textgreater{} 5;}
\NormalTok{        7 {-}\textgreater{} \{6,12\};}
\NormalTok{        8 {-}\textgreater{} \{7\};}
\NormalTok{        3 {-}\textgreater{} 8;}
\NormalTok{        4 {-}\textgreater{} \{3,10\};}
\NormalTok{        5 {-}\textgreater{} 1;}
\NormalTok{        6 {-}\textgreater{} \{5,2\};}
\NormalTok{        11 {-}\textgreater{} \{6,7\};}
\NormalTok{        12 {-}\textgreater{} \{11,8\};}
\NormalTok{        13 {-}\textgreater{} 11;}
\NormalTok{        14 {-}\textgreater{} \{12,13\};}
\NormalTok{        9 {-}\textgreater{} \{12,10,3,4\};}
\NormalTok{    \}}
\end{Highlighting}
\end{Shaded}

\begin{itemize}
\tightlist
\item
  Strongly connected components: \(K1 = \{1,2,5\}\),
  \(K2 = \{7,8,11,12\}\).
\item
  \(G_R\):
\end{itemize}

\begin{Shaded}
\begin{Highlighting}[]
\NormalTok{digraph G \{ }
\NormalTok{    \{rank=same;3,4\}}
\NormalTok{    \{rank=same;K1,6,K2,9,10\}}
\NormalTok{    \{rank=same;13,14;\}}
\NormalTok{    rankdir=TD;}
\NormalTok{    K1{-}\textgreater{}6 [style = invis];}
\NormalTok{    6{-}\textgreater{}K2 [style = invis];}
\NormalTok{        K2 {-}\textgreater{} \{6,13,14,3,9\} [style = invis];}
\NormalTok{        10 {-}\textgreater{} \{4,3,9\} [style = invis];}
\NormalTok{        9 {-}\textgreater{} \{K2,3,4,10\} [style = invis];}
\NormalTok{        3 {-}\textgreater{} \{9, K2, 4, 10\} [style = invis];}
\NormalTok{        4 {-}\textgreater{} \{3,10,9\} [style = invis];}
        
        
\NormalTok{        10 {-}\textgreater{} 14 [style = invis];}
\NormalTok{        6 {-}\textgreater{} K1;}
\NormalTok{        K2 {-}\textgreater{} 6;}
\NormalTok{        3 {-}\textgreater{} K2;}
\NormalTok{        4 {-}\textgreater{} \{3,10\};}
\NormalTok{        9 {-}\textgreater{} \{K2,3,4,10\};}
\NormalTok{        14 {-}\textgreater{} \{K2,13\};}
\NormalTok{        13 {-}\textgreater{} \{K2\};}
\NormalTok{    \}}
\end{Highlighting}
\end{Shaded}

\begin{itemize}
\tightlist
\item
  Vertex base: \(K_B\) = \{9,14\}.
\end{itemize}

\hypertarget{use-the-matrix-tree-theorem-to-compute-the-number-of-spanning-forests-of-the-graph-below}{%
\paragraph{10) Use the matrix tree theorem to compute the number of
spanning forests of the graph
below!}\label{use-the-matrix-tree-theorem-to-compute-the-number-of-spanning-forests-of-the-graph-below}}

\begin{Shaded}
\begin{Highlighting}[]
\NormalTok{graph G }
\NormalTok{\{ }
\NormalTok{    \{}
\NormalTok{    1 {-}{-} 3;}
\NormalTok{    3 {-}{-} 2;}
\NormalTok{    2 {-}{-} 1;}
\NormalTok{    \}}
\NormalTok{    \{}
\NormalTok{    rankdir = TD;}
\NormalTok{    \{rank=same;4,8,6\}}
\NormalTok{    7 {-}{-} \{4,8,6\}}
\NormalTok{    4 {-}{-} 8;}
\NormalTok{    8 {-}{-} 6;}
\NormalTok{    8 {-}{-} 5;}
\NormalTok{    \}}
\NormalTok{\}}
\end{Highlighting}
\end{Shaded}

\emph{Kirchoff's Matrix-Tree Theorem:} If \(G=(V,E)\) is an undirected
graph and \(L\) is its graph Laplacian, then the number \(N_T\) of
spanning trees contained in \(G\) is given by: 1.) Choose a vertex
\(v_j\) and eliminate the \(j\)-th row and column from \(L\) to get a
new matrix \(\hat L_j\),

\begin{itemize}
\item
  Compute \(N_T = det(\hat L_j)\) (The number \(N_T\) counts spanning
  trees that are distinct as subgraphs of \(G\). The resulting sum of
  trees that contribute to \(N_T\) may be isomorphic.)

  To get all spanning forests of \(G = G_1 \dot \cup G_2\), we simply
  calculate \(N_T(G1) * N_T (G_2)\): The adjacency matrix of \(G_1\):
  \[A(G_1) = \begin{pmatrix}
    0 & 1 & 1\\
    1 & 0 & 1\\
    1 & 1 & 0
    \end{pmatrix}\]

  The corresponding degree matrix:

  \[D(G_1) = \begin{pmatrix}
    2 & 0 & 0\\
    0 & 2 & 0\\
    0 & 0 & 2
    \end{pmatrix}\]

  The laplacian:

  \[L(G_1) = \begin{pmatrix}
    2 & -1 & -1\\
    -1 & 2 & -1\\
    -1 & -1 & 2
    \end{pmatrix}\]

  after deleting the first row and column:

  \[\hat L(G_1) = \begin{pmatrix}
    2 & -1\\
    -1 & 2
    \end{pmatrix}\]

  calculating the determinant:

  \[ 
    det(\hat L_1(G_1)) = 4 - 1 = 3
    \]

  The adjacency matrix of \(G_2\): \[A(G_2) = \begin{pmatrix}
    0 & 0 & 0 & 1 & 1\\
    0 & 0 & 0 & 0 & 1\\
    0 & 0 & 0 & 1 & 1\\
    1 & 0 & 1 & 0 & 1\\
    1 & 1 & 1 & 1 & 0
    \end{pmatrix}\]

  The laplacian matrix:

  \[L(G_2) = \begin{pmatrix}
    2 & 0 & 0 & -1 & -1 \\
    0 & 1 & 0 & 0 & -1  \\
    0 & 0 & 2 & -1 & -1 \\
    -1 & 0 & -1 & 3 & -1 \\
    -1 & -1 & -1 & -1 & 4
    \end{pmatrix}\]

  after deleting the fourth row and column:

  \[\hat L_4(G_2) = \begin{pmatrix}
    2 & 0 & 0 & -1 \\
    0 & 1 & 0 & 0  \\
    0 & 0 & 2 & -1 \\
    -1 & 0 & -1 & 3 
    \end{pmatrix}\]

  calculating the determinant:

  \[ 
    det(\hat L_4(G_2)) = -1 \times \begin{pmatrix}
    2 & 0 & -1 \\
    0 & 2 & -1 \\
    -1 & -1 & 3
    \end{pmatrix} 
    = -1 \times (12 + 0 + 0 - 2 - 0 - 2) = -8
    \] we take \(8\) as the sign only depends on which row and column
  you eliminate. Thus the number of \(G = 3 \times 8\).
\end{itemize}

\hypertarget{exercise-sheet-2}{%
\subsection{Exercise Sheet 2}\label{exercise-sheet-2}}

\hypertarget{k_n-denotes-the-complete-graph-with-n-vertices.-show-that-the-number-of-spanning-trees-of-k_n-is-nn-2.}{%
\paragraph{\texorpdfstring{11) \(K_n\) denotes the complete graph with
\(n\) vertices. Show that the number of spanning trees of \(K_n\) is
\(n^{n-2}\).}{11) K\_n denotes the complete graph with n vertices. Show that the number of spanning trees of K\_n is n\^{}\{n-2\}.}}\label{k_n-denotes-the-complete-graph-with-n-vertices.-show-that-the-number-of-spanning-trees-of-k_n-is-nn-2.}}

We observe that by completeness of \(K_n\), its adjacency matrix is the
\(n \times n\) matrix of ones, similarly its degree matrix is the
\(n \times n\) matrix with \(n\) on its diagonal.

\(D(K_n) = \begin{pmatrix}  n & 0 & 0 & \dots \\  0 & n & 0 &\dots \\  0 & 0 & n &\dots \\  \vdots & \vdots & \vdots & \ddots & \\  \end{pmatrix}\)

\(A(K_n) = \begin{pmatrix}  1 & 1 & 1 & \dots \\  1 & 1 & 1 &\dots \\  1 & 1 & 1 &\dots \\  \vdots & \vdots & \vdots & \ddots & \\  \end{pmatrix}\)

We can thus see that \(L(K_n) = D(K_n) - A(K_n)\) is defined as follows:

\(L(K_n) = \begin{pmatrix}  n-1 & -1 & -1 & \dots \\  -1 & n-1 & -1 &\dots \\  -1 & -1 & n-1 &\dots \\  \vdots & \vdots & \vdots & \ddots & \\  \end{pmatrix}\)

We then delete the first row and column from \(L(K_n)\) to get a new
\(n-1 \times n-1\) matrix \(L_1\).

\(L_1(K_n) = \begin{pmatrix}  1 & 1 & 1 & \dots \\  -1 & n-1 & -1 &\dots \\  -1 & -1 & n-1 &\dots \\  \vdots & \vdots & \vdots & \ddots & \\  \end{pmatrix}\)

We can then add all other rows to the first row. We observe that each
column in the resulting matrix \(L_1(K_n)'\) contains exactly \(n-2\)
negative ones and one \(n-2\) entry, thus the value in each column of
row 1 is defined by \(n-1-n-2 = 1\).

\(L_1(K_n)' = \begin{pmatrix}  n-1 & -1 & -1 & \dots \\  -1 & n-1 & -1 &\dots \\  -1 & -1 & n-1 &\dots \\  \vdots & \vdots & \vdots & \ddots & \\  \end{pmatrix}\)

We can then derive the matrix \(L_1(K_n)''\) by adding the first row to
each other row.

\(L_1(K_n)'' = \begin{pmatrix}  1 & 1 & 1 & \dots \\  0 & n & 0 &\dots \\  0 & 0 & n &\dots \\  \vdots & \vdots & \vdots & \ddots & \\  \end{pmatrix}\)

We observe that, \(L_1(K_n)''\) is an upper triangular matrix and the
determinant is thus calculated by
\(det(L_1(K_n)'')= l_{11} \times l_{22} \times \dots l_{n-1 \ n-1} = 1 \times n^{n-2} = n-2\).
\#\#\#\# 12) If \(T\) is a tree having no vertex of degree 2, then \(T\)
has more leaves than internal nodes. Prove this claim: \textbf{1) By
induction.} --Proof by induction--

\begin{itemize}
\tightlist
\item
  Base case: 2 leaves, no internal nodes.
\end{itemize}

\begin{Shaded}
\begin{Highlighting}[]
\NormalTok{    graph G \{}
\NormalTok{    1 {-}{-} 2}
\NormalTok{    \}}
\end{Highlighting}
\end{Shaded}

\begin{itemize}
\tightlist
\item
  Induction Hypothesis: A tree \(T=(V,E)\) with size \(|V| < n\) and no
  vertices \(v \in V\) with degree two has more leaves than internal
  nodes.
\item
  Induction step: We observe that \(T= (V,E)\) and \(|V| = n\) has some
  node \(v \in V\) such that \(v\) is connected to at most 1 non-leaf
  node but also at least 1 by n \textgreater{} 1. Since
  \(\forall v \in V: d(v) \neq 2\), v is connected to at least 2 leaves
  and some subtree of \(T\). We can then apply the induction hypothesis
  to \(T'\), where \(T'\) is a new tree, constructed by removing all
  leaves adjacent to \(v\) and the edges connecting the leaves to \(v\).
  \(T'\) still has no vertex of degree 2 as \(v\) is now a leaf,
  i.e.~\(d(v) = 1\) and all other vertices in \(T'\) have degree
  \(\neq 2\) by assumption. Then, by induction hypothesis
  \(L(T') > I (T')\) since we removed at least two leaves from \(T\) and
  one internal node in \(T\) became a leaf in \(T'\),
  \(L(T') + 1 > I(T') + 1 \Rightarrow L(T) > I(T)\). \textbf{2) Consider
  the average degree and use the Handshaking Lemma. Let \(T= (V,E)\) be
  a tree. Therefore, \(|E| = |V| - 1\) as shown in exercise 1). We can
  insert this into the Handshaking Lemma:} \[
  \sum_{v \in V} d(v) = 2 (|V| -1)
  \] We consider the average degree of a node, which is \[
  \frac{\sum_{v \in V} d(v)}{|V|} = \frac{2(|V|-1)}{|V|}
  \] Now, since each internal node, contributes at least 3 to the sum of
  degrees and each leaf contibutes exactly one to that sum, let \(i\) be
  the number of internal nodes and \(l\) the number of leaf nodes in
  \(T\). \[
  \frac{2(|V|-1)}{|V|} \geq \frac{3i+l}{|V|} = 2(|V|-1) \geq 3i+l
  \] and since \(|V| = i+l\), \(2(|V| -1) \geq 2i+|V|\), from which we
  subtract \(|V|\), to get
  \[|V| - 2\geq 2i \Leftrightarrow \frac{|V|}{2} - 1 \geq i \Leftrightarrow \frac{|V|}{2} > i.\]
\end{itemize}

Since \(i\) represents less than half of the nodes in \(V\), and all
other nodes have to be leaf nodes, more nodes in \(V\) are leaf nodes
than internal nodes.

\hypertarget{let-g-ve-be-a-connected-graph-with-an-even-number-of-vertices.-show-that-there-is-a-not-necessarily-connected-spanning-subgraph-in-which-all-vertices-have-odd-degree.}{%
\paragraph{\texorpdfstring{13) Let \(G = (V,E)\) be a connected graph
with an even number of vertices. Show that there is a (not necessarily
connected) spanning subgraph in which all vertices have odd
degree.}{13) Let G = (V,E) be a connected graph with an even number of vertices. Show that there is a (not necessarily connected) spanning subgraph in which all vertices have odd degree.}}\label{let-g-ve-be-a-connected-graph-with-an-even-number-of-vertices.-show-that-there-is-a-not-necessarily-connected-spanning-subgraph-in-which-all-vertices-have-odd-degree.}}

We show that this property holds for trees, as all connected grahs have
a tree as a subgraph. --Proof by induction--

\begin{itemize}
\tightlist
\item
  Base case: \(|V| = 2\), then \(d(v) = 1\) for all vertices in the
  tree.
\item
  Induction hypothesis: Let \(G=(V,E)\) be a connected graph with an
  even number of vertices and \(|V| = k\), \(k < n\), \(G\) has a
  connected spanning subgraph where all \(v \in V\) have an odd degree.
\item
  Induction step: We observe that \(T\) has some node \(v\) which is
  connected to at most one non-leaf. We apply a case distinction on the
  degree of \(v\):

  \begin{itemize}
  \tightlist
  \item
    \(v\) has even degree: Then \(v\) is connected to an odd number of
    leaves and one node \(v_1\) which is connected to at least one other
    node, since otherwise \(G\) would have an odd number of vertices. We
    can then delete the edge between \(v\) and \(v_1\). \(v\) and its
    leaves represent a spanning subgraph \(\dot G\), and since \(v\) has
    an odd number of leaves it has an odd degree, foreach leaf this also
    holds trivially. The spanning tree \(G\) can then be constructed by
    taking the subgraph connected to \(v_1\) without the removed edge
    \(v,v1\) and applying the induction hypothesis to it since we
    removed an even number of vertices it is applicable. We get some
    spanning subgraph \(G'\) by this and the spanning subgraph for \(G\)
    is \(G' \cup \dot G\).
  \item
    \(v\) has odd degree: Then \(v\) is connected to an even number of
    leaves and one node \(v_1\) which could be connected to some other
    node. This means there is a spanning subgraph \(\dot G\), comprised
    of \(v\) and its leaves, where all leaves trivially have odd degree
    and since we keep \((v,v_1)\), \(v\) has an odd degree as well. We
    construct a spanning subgraph \(G'\) by removing \(v'\)'s leaves and
    the edges connecting the leaves to \(v\) and applying the induction
    hypothesis, since the number of vertices we removed were even. We
    can then get a spanning subgraph of \(G\) by \(\dot G \cup G'\).
  \end{itemize}
\end{itemize}

\textbf{Is this also true for non-connected graphs?} No, we provide a
counterexample:

\begin{Shaded}
\begin{Highlighting}[]
\NormalTok{graph \{}
\NormalTok{2{-}{-}\{1,3\}}

\NormalTok{5{-}{-}\{4,6\}}
\NormalTok{\}}
\end{Highlighting}
\end{Shaded}

The number of vertices is even, but there is no spanning subgraph such
that all vertices have odd degree, seeing as we cannot remove edges or
we lose the spanning property.

\hypertarget{list-all-matroids-es-with}{%
\paragraph{\texorpdfstring{14) List all matroids \((E,S)\)
with:}{14) List all matroids (E,S) with:}}\label{list-all-matroids-es-with}}

\begin{itemize}
\tightlist
\item
  \(E = \{1\}\): \[\{\emptyset\}, \{\emptyset, \{1\}\}\]
\item
  \(E=\{1,2\}\): \[\emptyset,\\
    \{\emptyset, \{1\}\}, \{\emptyset, \{2\}\}, \{\emptyset, \{1\}, \{2\}\},\\
    \{\emptyset, \{1\}, \{2\}, \{1,2\} \]
\item
  \(E = \{1,2,3\}\): \[\emptyset,\\
    \{\emptyset, \{1\}\}, \{\emptyset, \{2\}\}, \{\emptyset, \{3\}\},\\
    \{\emptyset, \{1\}, \{2\}\} , \{\emptyset, \{2\}, \{3\}\}, \{\emptyset, \{1\}, \{3\}\}, 
    \{\emptyset, \{1\}, \{2\}, \{3\} \}, \\
    \{\emptyset, \{1\}, \{2\}, \{1,2\} \}, \{\emptyset, \{2\}, \{3\}, \{2,3\}\} , \\
    \{\emptyset, \{1\}, \{3\}, \{1,3\}\}, \\
    \{\emptyset, \{1\}, \{2\}, \{3\}, \{1,2\}, \{2,3\},\{1,3\}\},\\
    \{\emptyset, \{1\}, \{2\}, \{3\}, \{1,2\}, \{2,3\}\},\\
    \{\emptyset, \{1\}, \{2\}, \{3\}, \{1,2\}, \{1,3\}\},\\
    \{\emptyset, \{1\}, \{2\}, \{3\}, \{2,3\}, \{1,3\}\},\\
    \{\emptyset, \{1\}, \{2\}, \{3\}, \{1,2\}, \{2,3\}, \{1,3\}, \{1,2,3\}\}\]
\end{itemize}

\hypertarget{let-e-abcdefg-and-s-a-subseteq-e-a-leq-3-backslash-abc-cde-aef-adg-cfg-beg-bdf.-examine-whether-es-is-a-matroid.}{%
\paragraph{\texorpdfstring{15) Let \(E = \{a,b,c,d,e,f,g\}\) and
\[S = \{A \subseteq E \ | \ |A| \leq 3 \} \backslash \{\{a,b,c\}, \{c,d,e\}, \{a,e,f\}, \{a,d,g\}, \{c,f,g\}, \{b,e,g\}, \{b,d,f\}\}.\]
Examine whether \((E,S)\) is a
matroid.}{15) Let E = \textbackslash\{a,b,c,d,e,f,g\textbackslash\} and S = \textbackslash\{A \textbackslash subseteq E \textbackslash{} \textbar{} \textbackslash{} \textbar A\textbar{} \textbackslash leq 3 \textbackslash\} \textbackslash backslash \textbackslash\{\textbackslash\{a,b,c\textbackslash\}, \textbackslash\{c,d,e\textbackslash\}, \textbackslash\{a,e,f\textbackslash\}, \textbackslash\{a,d,g\textbackslash\}, \textbackslash\{c,f,g\textbackslash\}, \textbackslash\{b,e,g\textbackslash\}, \textbackslash\{b,d,f\textbackslash\}\textbackslash\}. Examine whether (E,S) is a matroid.}}\label{let-e-abcdefg-and-s-a-subseteq-e-a-leq-3-backslash-abc-cde-aef-adg-cfg-beg-bdf.-examine-whether-es-is-a-matroid.}}

\((E,S)\) is a matroid:

\begin{itemize}
\tightlist
\item
  The empty set is in \(S\).
\item
  \(S\) is closed under inclusion, since all two, and one-element
  subsets of \(E\) are in \(S\).
\item
  Assume some \(\{i,h\} \in S\) and \(\{x,y,z\} \in S\), then if
  \(\{i,h\} \subset \{x,y,z\}\) we take the elem \(j\) from
  \(\{x,y,z\}\) which is not in \(\{i,h\}\) and get
  \(\{x,y,z\} = \{i,h\} \cup j\) thus trivially in \(S\). If there is
  some element in \(\{i,h\}\) which is not in \(\{x,y,z\}\) then we have
  a choice of two elements in \(\{x,y,z\}\) which we can add to
  \(\{i,h\}\). Since the excluded sets only have a max one element
  overlap, for any two element subset of \(E\), we can choose the
  element in \(E\) which does not create an excluded set.
\end{itemize}

\hypertarget{prove-that-an-independence-system-es-is-a-matroid-iff-for-every-a-subseteq-e-all-maximal-independent-subsets-of-a-have-the-same-cardinality.}{%
\paragraph{\texorpdfstring{16) Prove that an independence system
\((E,S)\) is a matroid iff for every \(A \subseteq E\), all maximal
independent subsets of \(A\) have the same
cardinality.}{16) Prove that an independence system (E,S) is a matroid iff for every A \textbackslash subseteq E, all maximal independent subsets of A have the same cardinality.}}\label{prove-that-an-independence-system-es-is-a-matroid-iff-for-every-a-subseteq-e-all-maximal-independent-subsets-of-a-have-the-same-cardinality.}}

=\textgreater) Assume \(A,B \in S\) such that \(A\) and \(B\) are both
maximal and w.l.o.g. \(|A| > |B|\) and by the matroid property
\(B \cup \{x\} \in S\). Therefore \(B\) is not a maximal independent
subset of \(E\). Contradiction!

\textless=) This does not hold: --Counterexample-- Let \((E,S) = M\) be
an independence system such that: \[
    E = \{1,2,3,4,5,6,7\}
    \] and \(S\) contains \(\emptyset\), all subsets of \(E\) of size 1
and
\(\{1,2\}, \{1,3\}, \{2,3\}, \{1,5\}, \{3,4\}, \{3,5\}, \{4,5\}, \{1,2,3\}, \{1,3,5\}, \{3,4,5\}\).
Then, all maximal independent subsets \(A \subseteq E\) have the same
cardinality but \(M\) is not a matroid as for \(\{2,3\}\), and
\(\{3,4,5\}\), there exists no \(x \in \{3,4,5\}\) such that
\(\{2,3\} \cup \{x\}\) is in \(S\), since \(\{2,3,4\} \notin S\) and
\(\{2,3,5\} \notin S\).

\hypertarget{let-e_1-and-e_2-be-two-disjoint-sets.-moreover-assume-that-e_1s_1-and-e_2s_2-are-matroids.-define-s-x-cup-y-x-in-s_1-text-and-y-in-s_2.-prove-that-e_1-cup-e_2-s-is-a-matroid.}{%
\paragraph{\texorpdfstring{17) Let \(E_1\) and \(E_2\) be two disjoint
sets. Moreover, assume that \((E_1,S_1)\) and \((E_2,S_2)\) are
matroids. Define
\(S := \{X \cup Y \ | \ X \in S_1 \text{ and } Y \in S_2\}\). Prove that
\((E_1 \cup E_2, S)\) is a
matroid.}{17) Let E\_1 and E\_2 be two disjoint sets. Moreover, assume that (E\_1,S\_1) and (E\_2,S\_2) are matroids. Define S := \textbackslash\{X \textbackslash cup Y \textbackslash{} \textbar{} \textbackslash{} X \textbackslash in S\_1 \textbackslash text\{ and \} Y \textbackslash in S\_2\textbackslash\}. Prove that (E\_1 \textbackslash cup E\_2, S) is a matroid.}}\label{let-e_1-and-e_2-be-two-disjoint-sets.-moreover-assume-that-e_1s_1-and-e_2s_2-are-matroids.-define-s-x-cup-y-x-in-s_1-text-and-y-in-s_2.-prove-that-e_1-cup-e_2-s-is-a-matroid.}}

Let \(A,B\) be arbitrary elements in \(S\) such that, w.l.o.g
\(|A| > |B|\), then by construction of \((E_1 \cup E_2, S)\) we can
deconstruct the sets as follows: \[
A := A_1 \cup A_2\\
B := B_1 \cup B_2
\] where \(A_1,B_1\) are all elements in \(A,B\) which occur in \(S_1\)
and \(A_2,B_2\) the ones from \(S_2\). Now, we know that either
\(|A_1| > |B_1|\) or \(|A_2| > |B_2|\), let \(A_i, B_i\) be the sets for
which this property holds. Then, by \((E_i, S_i)\) being a matroid,
there exists an \(x \in A_i\) such that \(B_i \cup \{x\} \in S_i\).
Therefore, by definition of \((E_1 \cup E_2, S)\),
\(B_i \cup \{x\} \in S\).

\hypertarget{exercise-sheet-3}{%
\subsection{Exercise Sheet 3}\label{exercise-sheet-3}}

\hypertarget{for-a-simple-and-undirected-graph-g-we-define-the-line-graph-bar-g-as-follows-vbar-g-eg-and-ef-in-ebar-g-iff-the-edges-e-and-f-share-a-vertex.-prove-that-the-line-graph-of-a-eularian-graph-is-eularian-and-hamiltonian.}{%
\paragraph{\texorpdfstring{23) For a simple and undirected graph \(G\)
we define the \emph{line graph} \(\bar G\) as follows:
\(V(\bar G) = E(G)\) and \((e,f) \in E(\bar G)\) iff the edges \(e\) and
\(f\) share a vertex. Prove that the line graph of a Eularian graph is
Eularian and
Hamiltonian.}{23) For a simple and undirected graph G we define the line graph \textbackslash bar G as follows: V(\textbackslash bar G) = E(G) and (e,f) \textbackslash in E(\textbackslash bar G) iff the edges e and f share a vertex. Prove that the line graph of a Eularian graph is Eularian and Hamiltonian.}}\label{for-a-simple-and-undirected-graph-g-we-define-the-line-graph-bar-g-as-follows-vbar-g-eg-and-ef-in-ebar-g-iff-the-edges-e-and-f-share-a-vertex.-prove-that-the-line-graph-of-a-eularian-graph-is-eularian-and-hamiltonian.}}

\begin{itemize}
\tightlist
\item
  \textbf{Eularian:} For an arbitrary vertex \(e \in \bar V\), its
  neighbours are defined as follows, by construction of \(\bar G\): for
  \(v_1, v_2\) such that \(e\) connects \(v_1\) to \(v_2\) in \(G\):
  where
  \(v_1 = \{(v_1,v_j) \ | \ v_1 \neq v_j \land (v_1,v_j) \in E\}\), and
  \(v_2 = \{(v_i,v_2) \ | \ v_i \neq v_2 \land (v_i,v_2) \in E\}\).
  Then, \(N(e) = V_1 \cup V_2\) where \(V_1\) and \(V_2\) are disjoint
  by construction. Thus, \(|N(e)| = |V_1| + |V_2|\). Furthermore, we
  know \(|V_1| = deg(v_1)-1\) and \(|V_2| = deg(v_2)-1\) by construction
  of the respective sets. By \(G\) Eularian, we know that \(deg(v_1)\)
  and \(deg(v2)\) are even, thus the sizes of \(V_1,V_2\) are each odd,
  which means \(|N(e)| = deg(e)\) is even by \(|N(e)| = |V_1| + |V_2|\)
  and \(V_1,V_2\) being odd. We have shown that for all
  \(e \in \bar V\), \(deg(e)\) is even, thus \(\bar G\) is eularian.
\item
  \textbf{Hamiltonian:} By construction, there exists a eularian cycle
  in \(G\), which visits all edges once exactly. Denote this cycle by
  \(e_1, \dots, e_m\), where some vertex \(v_1\) is incident to \(e_1\)
  and \(e_m\). By definition, each edge pair
  \((e_1,e_2), \dots, (e_m,e_1)\) is incident to some shared vertex. By
  construction of \(\bar G\) there then exists a hamiltonian cycle in
  \(\bar G\), i.e.~\(e_1 \rightarrow e_2 \leadsto e_m \rightarrow e_1\)
  where no node is revisited.
\end{itemize}

\hypertarget{prove-that-a-graph-g-is-bipartite-iff-each-cycle-in-g-has-even-length.}{%
\paragraph{\texorpdfstring{24) Prove that a graph \(G\) is bipartite iff
each cycle in \(G\) has even
length.}{24) Prove that a graph G is bipartite iff each cycle in G has even length.}}\label{prove-that-a-graph-g-is-bipartite-iff-each-cycle-in-g-has-even-length.}}

=\textgreater) --Proof by Contradiction-- Let \(G\) be a bipartite graph
which contains a cycle \(c\) with odd length. By assumption, \(c\) is
defined by \(v_1 \rightarrow v_2 \leadsto v_k \rightarrow v_1\) and
\(k\) is odd, i.e.~c has odd length. Then by \(G\) being bipartite we
can divide \(V\) into \(V_1,V_2\) such that \(V = V_1 \cup V_2\) and
\(\forall e \in E\), \(e\) connects a vertex in \(V_1\) to some vertex
in \(V_2\). Thus, w.l.o.g., by \((v_1,v_2) \in E\), \(v_1 \in V_1\) and
\(v_2 \in V_2\), which means every second vertex on \(c\) is in \(V_2\)
and every other vertex on \(c\) in \(V_1\). We know,
\(v_1 \leadsto v_{k-1}\) is of even length, thus \(v_{k-1} \in V_2\) and
by \((v_{k-1}, v_k) \in E\), \(v_k \in V_1\), but then \(v_1 \in V_2\)
would have to hold, by \((v_k,v_1) \in E\). Contradiction! Therefore,
every cycle in \(G\) has even length, assuming \(G\) is bipartite.

\textless=) Let \(G\) be a graph such that each cycle in \(G\) has even
length. W.l.o.g. assume \(G\) is connected. We can then split \(V\) into
\(V_1\) and \(V_2\) such that \(V = V_1 \cup V_2\) as follows:

We take some \(v_1 \in V\) and say \(v_1 \in V_1\), then forall other
vertices in \(v \in V\):

\begin{itemize}
\tightlist
\item
  let \(V_1 = \{v \ | \ v \in V\) and the shortest path from \(v\) to
  \(v_1\) is of even length\(\}\), and
\item
  let \(V_2 = \{v \ | \ v \in V\) and the shortest path from \(v\) to
  \(v_1\) is of odd length\(\}\)
\end{itemize}

We show, no vertex in \(V_1\) is connected to some other vertex in
\(V_1\). (Analogous proof for \(V_2\))

--Proof by Contradiction-- Assume there exist \(v_i,v_j \in V_1/V_2\)
such that \((v_i,v_j) \in E\), then there exists a cycle
\(c: v_1 \leadsto v_i \rightarrow v_j \leadsto v_1\) by construction of
\(V_1/V_2\). We distinguish: * \(v_i,v_j \in V_1\): then the length of
\(c\) is defined by \(v_1 \leadsto v_i + 1 + v_j \leadsto v_1\), by
construction of \(V_1\), \(v_1 \leadsto v_i\) and \(v_j \leadsto v_1\)
are each even, meaning the length of \(c\) is odd. Contradiction! *
\(v_i, v_j \in V_2\): then the length of
\(c: v_1 \leadsto v_i + 1 + v_j \leadsto v_1\), where
\(v_1 \leadsto v_i, v_j \leadsto v_1\) are each odd, thus their sum is
even, if we now add 1 for the edge between \(v_i\) and \(v_j\), the
length of \(c\) is odd. Contradiction!

In both cases \(c\) is of odd length, contradicting the assumption!

\hypertarget{let-g-be-a-eularian-graph-and-h-be-a-subdivision-of-g.}{%
\paragraph{\texorpdfstring{25) Let \(G\) be a Eularian graph and \(H\)
be a subdivision of
\(G\).}{25) Let G be a Eularian graph and H be a subdivision of G.}}\label{let-g-be-a-eularian-graph-and-h-be-a-subdivision-of-g.}}

\begin{itemize}
\tightlist
\item
  \textbf{Is \(H\) Eularian?} --Yes-- Theorem: An undirected, connected
  graph is Eularian iff all its vertices have even degree.
\end{itemize}

Since \(G\) is Eularian, all \(v \in V\) have even degree. Now, all
vertices in \(H\) are either

\begin{itemize}
\item
  vertices from \(G\): then either their edges are equivalent to the
  ones in \(G\), meaning their degree remains unchanged from the one in
  the original graph or some edges connecting to the vertice were
  subdivided then their degree also remains unchanged as for each
  removed edge, a new one is added.
\item
  Or they are some new vertice \(w\) which was added by a subdividing
  operation. Then their degree is exactly two and thus even.
\item
  \textbf{Suppose that \(G\) is Hamiltonian. Does this imply that \(H\)
  is Hamiltonian as well?} --Counter example-- \(G\) is Hamiltonian.
\end{itemize}

\begin{Shaded}
\begin{Highlighting}[]
\NormalTok{graph \{}
\NormalTok{\{rank=same;1,2\}}
\NormalTok{\{rank=same;3,4\}}

\NormalTok{1{-}{-}2{-}{-}3}
\NormalTok{4{-}{-}1}
\NormalTok{4{-}{-}3}
\NormalTok{4{-}{-}2}

\NormalTok{\}}
\end{Highlighting}
\end{Shaded}

\(H\) is not Hamiltonian.

\begin{Shaded}
\begin{Highlighting}[]
\NormalTok{graph \{}
\NormalTok{\{rank=same;1,2\}}
\NormalTok{\{rank=same;3,4\}}

\NormalTok{1{-}{-}2{-}{-}3}
\NormalTok{4{-}{-}1}
\NormalTok{4{-}{-}3}
\NormalTok{2{-}{-}5}
\NormalTok{5{-}{-}4}

\NormalTok{\}}
\end{Highlighting}
\end{Shaded}

\hypertarget{section}{%
\paragraph{26)}\label{section}}

\begin{enumerate}
\def\labelenumi{(\alph{enumi})}
\tightlist
\item
\end{enumerate}

\begin{enumerate}
\def\labelenumi{\arabic{enumi})}
\tightlist
\item
  \textbf{Prove that every simple connected planar graph with at least 3
  vertices satisfies \(\alpha_1(G) \leq 3\alpha_0(G) - 6\).}
\end{enumerate}

We know that each edge contributes to exactly two (not necessarily)
different faces. A face touches at least 3 edges but maybe more.

Therefore, we can define the following inequality: \[
3 \alpha_2(G) \leq 2 \alpha_1(G)
\] Now, by Euler's Formula:
\(\alpha_0(G) - \alpha_1(G) + \alpha_2(G) = 2\)
i.e.~\(\alpha_2(G) = 2- \alpha_0(G) + \alpha_1(G)\)

We can then substitute this into the defined inequality: \[
3(2 - \alpha_0(G) + \alpha_1(G)) \leq 2\alpha_1(G) \Leftrightarrow
\] \[
6 - 3\alpha_0(G) + 3\alpha_1(G) \leq 2\alpha_1(G) \Leftrightarrow
\] \[
6 + 3 \alpha_1(G) \leq 2 \alpha_1(G) + 3\alpha_0(G) \Leftrightarrow
\] \[
6 + \alpha_1(G) \leq 3 \alpha_0(G) \Leftrightarrow
\] \[
\alpha_1(G) \leq 3\alpha_0(G) - 6
\]

\begin{enumerate}
\def\labelenumi{\arabic{enumi})}
\setcounter{enumi}{1}
\tightlist
\item
  \textbf{Show that this implies that \(K_5\) is not planar:}
\end{enumerate}

\(\alpha_1(K_5) = 4 * 5 = 20\) and \(\alpha_0(K_5) = 5\), then we put
these numbers into the inequality:
\(20 \leq 3 \times 5 - 6 = 15 - 6 = 9\), therefore \(K_5\) is not
planar.

\begin{enumerate}
\def\labelenumi{(\alph{enumi})}
\setcounter{enumi}{1}
\tightlist
\item
  \textbf{Prove the following statement or find a counter example: For
  all \(n - m + f = 2\) and for which there exists a simple graph with
  \(\alpha_0 = n, \alpha_1 = m\) there exists also a simple planar graph
  with \(\alpha_0 = n, \alpha_1 = m\) and \(\alpha_2 = f\).}
\end{enumerate}

--Counter example-- \(K_5\) with \(n=5\), \(m=20\), there exists a
simple graph with \(\alpha_0 = 5\) and \(\alpha_1 = 20\) but there does
not exist a planar simple graph for \(n=5\) and \(m=20\) as shown above.

\hypertarget{let-n-in-mathbbn-and-g-v_1cup-v_2e-be-a-bipartite-graph-with-min_x-in-v-dx-geq-n2-and-v_1-v_2-n.-use-halls-theorem-to-prove-that-g-has-a-perfect-matching.}{%
\paragraph{\texorpdfstring{27) Let \(n \in \mathbb{N}\) and
\(G = (V_1\cup V_2,E)\) be a bipartite graph with
\(min_{x \in V} d(x) \geq n/2\) and \(|V_1| = |V_2| = n\). Use Hall's
theorem to prove that \(G\) has a perfect
matching.}{27) Let n \textbackslash in \textbackslash mathbb\{N\} and G = (V\_1\textbackslash cup V\_2,E) be a bipartite graph with min\_\{x \textbackslash in V\} d(x) \textbackslash geq n/2 and \textbar V\_1\textbar{} = \textbar V\_2\textbar{} = n. Use Hall's theorem to prove that G has a perfect matching.}}\label{let-n-in-mathbbn-and-g-v_1cup-v_2e-be-a-bipartite-graph-with-min_x-in-v-dx-geq-n2-and-v_1-v_2-n.-use-halls-theorem-to-prove-that-g-has-a-perfect-matching.}}

W.l.o.g. we argue over an arbitrary subset \(S \subseteq V_1\). We
denote the set of all neighbours of all vertices in \(S\) by:

\[
N(S) = \{v_2 \ | \ (v_1,v_2) \in E \text{ and } v_1 \in S\}
\]

We distinguish two cases:

\begin{enumerate}
\def\labelenumi{\arabic{enumi})}
\item
  \(|S| \leq n/2\): then for some vertex \(v \in S\), we know that
  \(|N(v)| \geq n/2\). Additionally, \(|N(S)| \geq |N(v)|\) by
  construction of \(N(S)\). Then, by assumption,
  \(|N(S)| \geq |N(v)| \geq n/2 \geq |S|\), i.e.~\(|N(S)| \geq |S|\).
\item
  \(|S| > n/2\): then for any vertex \(v_2 \in V_2\), we know that is is
  connected to some vertex in \(S\) as \(d(v_2) \geq n/2\) and \(V_2\)
  is only connected to vertices in \(V_1\) of which more than \(n/2\)
  vertices are in \(S\). Thus, \(\forall v_2 \in V_2\),
  \(N(v_2) \cap S \neq \emptyset\), which means \(\forall v_2 \in V_2\),
  \(v_2 \in N(S)\). We conclude \(|N(S)| = n\), thus
  \(|S| \leq |N(S)|\), since \(|S| \leq |V_1| = n = |N(S)|\).
\end{enumerate}

\emph{Hall's Marriage Theorem}: There is a matching of size \(|A|\) iff
every set \(S \subseteq A\) of vertices is connected to at least \(|S|\)
vertices in \(B\).

Since we showed \(|S| \leq |N(S)|\) for all possible subsets
\(S \subseteq V_1\). We can apply Hall's Marriage theorem and derive
that there is a matching of size \(|V_1| = n\) which is a perfect
matching.

\hypertarget{let-g-be-a-graph-with-alpha_0g-n-and-chig-k.-prove-that-alpha_1g-geq-binomk2.}{%
\paragraph{\texorpdfstring{28) Let \(G\) be a graph with
\(\alpha_0(G) = n\) and \(\chi(G) = k\). Prove that
\(\alpha_1(G) \geq \binom{k}{2}\).}{28) Let G be a graph with \textbackslash alpha\_0(G) = n and \textbackslash chi(G) = k. Prove that \textbackslash alpha\_1(G) \textbackslash geq \textbackslash binom\{k\}\{2\}.}}\label{let-g-be-a-graph-with-alpha_0g-n-and-chig-k.-prove-that-alpha_1g-geq-binomk2.}}

We know that any \(k\)-chromatic graph has at least \(k\) vertices of
degree \(k-1\) each, otherwise it would not be \(k\)-chromatic.

Then, by Handshaking lemma: \[
k (k-1) \leq \sum_{v \in V} deg(v) = 2 |E| \Leftrightarrow 
\] \[
\frac{k (k-1)}{2} \leq |E|
\]

Which is equivalent to \(|E| \geq \binom{k}{2}\) as: \[
\binom{k}{2}  = \frac{k!}{2(k-2)!} = \frac{k(k-1)(k-2)!}{2(k-2)!} = \frac{k(k-1)}{2}
\]

\hypertarget{let-g_1-v-e_1-and-g_2-v-e_2-be-two-graphs.-set-g-v-e_1-cup-e_2-and-prove-that-chig-leq-chig_1chig_2.}{%
\paragraph{\texorpdfstring{29) Let \(G_1 = (V, E_1)\) and
\(G_2 = (V, E_2)\) be two graphs. Set \(G = (V, E_1 \cup E_2)\) and
prove that
\(\chi(G) \leq \chi(G_1)\chi(G_2)\).}{29) Let G\_1 = (V, E\_1) and G\_2 = (V, E\_2) be two graphs. Set G = (V, E\_1 \textbackslash cup E\_2) and prove that \textbackslash chi(G) \textbackslash leq \textbackslash chi(G\_1)\textbackslash chi(G\_2).}}\label{let-g_1-v-e_1-and-g_2-v-e_2-be-two-graphs.-set-g-v-e_1-cup-e_2-and-prove-that-chig-leq-chig_1chig_2.}}

Since \(G_1\) is \(\chi(G_1)\)-colourable, there exists some colouring
\(c_1\) such that \(c_1\) \(\chi(G_1)\)-colours \(G_1\). Analogously,
there exists some colouring \(c_2\) such that \(c_2\)
\(\chi(G_2)\)-colours \(G_2\).

We can then construct a colouring \(c\) which colours an arbitrary
vertex \(v \in G\) by: \[
c(v) = (c_1(v), c_2(v)),
\] then \(\chi(G) \leq \chi(G_1) \times \chi(G_2)\) as the number of
colours which \(c\) uses are all combinations of the colours
\(c_1, c_2\) use, thus \(\chi(G_1) \times \chi(G_2)\), but there could
be a smaller colouring, e.g.~if \(G_1 = G_2\). We now show that \(c\)
colours \(G\) correctly, i.e.~no connected vertices in \(G\) are
coloured the same colour.

W.l.o.g. for an arbitrary edge \(e \in E_1\), where \(e = (v_1,v_2)\) we
know by \(G_1\) being \(\chi(G_1)\)-colourable by function \(c_1\), that
\(c_1(v_1) \neq c_2(v_2)\), thus
\((c_1(v_1), c_2(v_1)) \neq (c_1(v_2),c_2(v_2))\).

Therefore, \(c\) is a correct colouring of \(G\).

\hypertarget{show-the-following-inequality-for-ramsey-numbers-if-r-geq-3-then}{%
\paragraph{\texorpdfstring{30) Show the following inequality for Ramsey
numbers: If \(r \geq 3\)
then:}{30) Show the following inequality for Ramsey numbers: If r \textbackslash geq 3 then:}}\label{show-the-following-inequality-for-ramsey-numbers-if-r-geq-3-then}}

\[
R(n_1, \dots, n_{r-2}, n_{r-1}, n_r) \leq R(n_1, \dots, n_{r-2}, R(n_{r-1},n_r))
\] \emph{Hint}: Let \(n= R(n_1, \dots, n_{r-2}, R(n_{r-1}, n_r))\) and
consider an edge colouring of \(K_n\) with \(r\) colours, say
\(c_1, \dots, c_r\). Identify the colours \(c_{r-1}\) and \(c_r\) and
apply the Ramsey property for \(r-1\) colours.

Let \(n= R(n_1, \dots, n_{r-2}, R(n_{r-1}, n_r))\), \(n_i' = n_i\) for
\(i \in \{1, \dots, r-2\}\) and \(n_{r-1}' = R(n_{r-1},n_r))\), we
consider an edge colouring \(C\) of \(K_n\) with \(r\) colours,
\(c_1, \dots, c_r\). For each edge, coloured in \(c_{r-1}\) or \(c_r\),
we colour it in \(c_{r-1}\) to get a new colouring \(C'\). Let \(C'\) be
a colouring derived from \(C\), where all edges coloured \(c_r\) are
coloured \(c_{r-1}\).

By definition of \(n = R(n_1, \dots, n_{r-2}, R(n_{r-1},n_r))\), \(C'\)
contains some \(K_{n_{i}'}\) coloured in \(c_i\) and either:

\begin{itemize}
\tightlist
\item
  \(i \in \{1, \dots, r-2\}\), then \(K_{n_{i}}\) is coloured \(c_i\) in
  \(C\) by construction.
\item
  \(i = r-1\), then let \(x = R(n_{r-1}, n_r)\), the edges in \(K_x\)
  are coloured either \(c_{r-1}\) or \(c_r\) in \(C\) by construction of
  \(C'\). By definition of \(x\), \(K_x\) coloured in \(C\) contains
  either \(K_{n_{r-1}}\) with all edges coloured in \(c_{r-1}\) or
  \(K_{n_r}\) with all edges coloured in \(c_r\).
\end{itemize}

Thus, \(K_n\) coloured in \(C\) contains some
\(K_{n_i}, i \in \{1, \dots, r\}\) with all edges coloured in \(c_i\).
Since \(C\) was arbitrary, \(R(n_1, \dots, n_r) \leq n\).

\hypertarget{exercise-sheet-4}{%
\subsection{Exercise Sheet 4}\label{exercise-sheet-4}}

\hypertarget{let-a-be-a-non-empty-set.-show-that-a-has-as-many-subsets-with-an-odd-number-of-elements-as-subsets-with-an-even-number-of-elements.}{%
\paragraph{\texorpdfstring{31) Let \(A\) be a non-empty set. Show that
\(A\) has as many subsets with an odd number of elements as subsets with
an even number of
elements.}{31) Let A be a non-empty set. Show that A has as many subsets with an odd number of elements as subsets with an even number of elements.}}\label{let-a-be-a-non-empty-set.-show-that-a-has-as-many-subsets-with-an-odd-number-of-elements-as-subsets-with-an-even-number-of-elements.}}

We proceed by induction on the size of \(A\) and show that \(A\) has
\(2^{|A|-1}\) odd-sized subsets.

\begin{itemize}
\item
  Base case: \(|A| = n = 1\), then \(A\) has two subsets, namely \(A\)
  and \(\emptyset\), thus we have \(2^{0} = 2^{n-1}\) odd-sized subsets.
\item
  Induction hypothesis: Let \(A\) be a set of size \(n\), then A has
  \(2^{n-1}\) odd-sized subsets.
\item
  Induction step: Let \(A\) be a set, such that \(|A| = n+1\) and let
  \(x \in A\). We construct a set \(A'\) by
  \(A' = A \backslash \{ x \}\).

  Then by construction of \(A'\), \(|A'| = n\) and thus by IH, \(A'\)
  has \(2^{n-1}\) odd-sized subsets. All possible subsets of \(A\) can
  now be constructed through all subsets of \(A'\) joined with
  \(\{S \cup \{x\} | S \subseteq A'\}\). Now, for each odd-sized subset
  \(A'_{1} \subseteq A'\), \(A'_{1} \cup \{x\}\) is even sized, and on
  the other hand for each even-sized, subset \(A'_{2} \subseteq A'\),
  \(A'_{2} \cup \{x\}\) is odd-sized.

  The amount of odd-sized subsets of \(A\) can thus be counted by taking
  all odd-sized subsets of \(A'\), of which there are \(2^{n-1}\)
  combined with all even-sized subsets of \(A'\) which are combined with
  \(x\) of which there are also \(2^{n-1}\) thus we have
  \(2^{n-1} + 2^{n-1} = 2^n\) odd-sized subsets.

  Since all subsets of \(A\) are either even or odd, by the sum
  principle \(|A_{even} \cup A_{odd}| = |A_{even}| + |A_{odd}|\),
  i.e.~\(2^{n+1} = 2^n + |A_{even}|\), thus
  \(|A_{odd}| = 2^n = |A_{even}|\).
\end{itemize}

\hypertarget{find-the-number-of-ways-to-place-n-rooks-on-a-n-times-n-chessboard-such-that-no-two-of-them-attack-each-other.}{%
\paragraph{\texorpdfstring{32) Find the number of ways to place \(n\)
rooks on a \(n \times n\) chessboard such that no two of them attack
each
other.}{32) Find the number of ways to place n rooks on a n \textbackslash times n chessboard such that no two of them attack each other.}}\label{find-the-number-of-ways-to-place-n-rooks-on-a-n-times-n-chessboard-such-that-no-two-of-them-attack-each-other.}}

We observe that each of the \(n\) rooks occupies one row, otherwise two
rooks would attack each other.

A column can then be chosen for each rook as follows: The first rook can
be placed on any of the \(n\) columns, the second can be placed on any
but the column the first rook occupies, etc.

The \(n\)-th rook has only a single column left since \(n-1\) columns
are already occupied.

Thus, there are \(n \times n-1 \times \dots \times 1 = n!\) ways to
place the \(n\) rooks on the \(n \times n\) checkerboard.

\hypertarget{let-p_nk-be-the-number-of-permutations-of-12-dots-n-having-exactly-k-fixed-points.-use-the-method-of-double-counting-to-prove-the-identity-sum_k0n-kp_nk-n.}{%
\paragraph{\texorpdfstring{33) Let \(p_n(k)\) be the number of
permutations of \(\{1,2, \dots, n\}\) having exactly \(k\) fixed points.
Use the method of double counting to prove the identity
\(\sum_{k=0}^n kp_n(k) = n!\).}{33) Let p\_n(k) be the number of permutations of \textbackslash\{1,2, \textbackslash dots, n\textbackslash\} having exactly k fixed points. Use the method of double counting to prove the identity \textbackslash sum\_\{k=0\}\^{}n kp\_n(k) = n!.}}\label{let-p_nk-be-the-number-of-permutations-of-12-dots-n-having-exactly-k-fixed-points.-use-the-method-of-double-counting-to-prove-the-identity-sum_k0n-kp_nk-n.}}

We define a matrix with \(n\) rows and \(n\) columns such that an entry
\(i,j\) is defined by the number of permutations with a fixed point of
size \(j\) which contain \(i\) as part of their fixed point.

Then each row \(i\) counts the number of permutations which have \(i\)
as a fixed point. The amount of permutations which have \(i\) as a fixed
point is \((n-1)!\) as \(i\) is fixed while all other spots are
permuted. Thus \(\sum_{k=1}^n (n-1)! = n*(n-1)! = n!\) when we count the
sum of the entries of all rows.

On the other hand, each column \(j\) counts the possible permutations
with a fixed point of size \(j\), but it is important to note that for
each fixed point of size \(j\) there are \(j\) entries in the column,
i.e.~an entry for each element of the fixed point. Therefore the sum
over columns is defined by \(\sum_{k=1}^n k \times p_n(k)\).

Example with \(n=3\), i.e.~\(\{1,2,3\}\): Here we have \(3\) fixed
points of size \(1\) (in column 1), namely each of the elements, and one
fixed point of size \(3\), where all elements are contained (column 3).

\begin{longtable}[]{@{}llll@{}}
\toprule()
& 1 & 2 & 3 \\
\midrule()
\endhead
\textbf{1} & 1 & 0 & 1 \\
\textbf{2} & 1 & 0 & 1 \\
\textbf{3} & 1 & 0 & 1 \\
\bottomrule()
\end{longtable}

\hypertarget{let-a-be-a-set-of-11-positive-integers-such-that-for-all-x-in-a-we-have-20-nmid-x.-prove-that-there-are-two-distinct-integers-ab-in-a-such-that-20-ab-or-20-a-b.}{%
\paragraph{\texorpdfstring{34) Let \(A\) be a set of 11 positive
integers such that for all \(x \in A\) we have \(20 \nmid x\). Prove
that there are two distinct integers \(a,b \in A\) such that
\(20 | (a+b)\) or
\(20 | (a-b)\).}{34) Let A be a set of 11 positive integers such that for all x \textbackslash in A we have 20 \textbackslash nmid x. Prove that there are two distinct integers a,b \textbackslash in A such that 20 \textbar{} (a+b) or 20 \textbar{} (a-b).}}\label{let-a-be-a-set-of-11-positive-integers-such-that-for-all-x-in-a-we-have-20-nmid-x.-prove-that-there-are-two-distinct-integers-ab-in-a-such-that-20-ab-or-20-a-b.}}

We can create \(10\) boxes such that we distribute the numbers into
those boxes based on their remainder modulo \(20\) as follows: \[
\{1,19\}, \{2,18\}, \{3,17\}, \{4,16\}, \{5,15\}, \{6,14\}, \{7,13\}, \{8,12\}, \{9,11\},\{10\}
\]

For each of these boxes, one can see that each of them surmount to
\(20\) when it contains two different numbers and their difference
surmounts to \(0\) when equal. As we have \(11\) integers, by the
pigeonhole principle at least one class will have two numbers.

Since, for each class, if there are two numbers \(a,b\) assigned to the
same class, we know either \(20 | (a + b)\) if \(a \neq b\) or
\(20 | (a-b)\) if \(a = b\).

Thus, every set \(A\) of \(11\) integers such that \(20 \nmid x\) for
all \(x \in A\) contains some \(a,b\) such that \(20 | (a + b)\) or
\(20 | (a-b)\).

\hypertarget{let-n-in-mathbbn-prove-the-identities}{%
\paragraph{\texorpdfstring{35) Let \(n \in \mathbb{N}\), prove the
identities:}{35) Let n \textbackslash in \textbackslash mathbb\{N\}, prove the identities:}}\label{let-n-in-mathbbn-prove-the-identities}}

\[
    \sum_{k=0}^{n} 2^k = 2^{n+1}-1
\]

We count the number of bitstrings of length \(n+1\) containing at least
one `1'.

\begin{itemize}
\tightlist
\item
  \emph{RHS:} We can construct the bitstrings of length \(n+1\)
  containing at least one `1', by fixing a `1' at each position \(k+1\),
  where \(0 \leq k \leq n\) in the string and fixing zeros in front of
  the `1', then for the bits behind the `1' there are \(2^k\)
  possibilities of setting them one or zero each. We count each string
  containing at least one `1', since, if there is some `1' in the
  bitstring, there also exists a rightmost `1' and since we fix each
  possible frontmost `1' exactly once, no string is counted twice.
\item
  \emph{LHS:} There are \(2^{n+1}\) possible bitstrings of length
  \(n+1\), but the string containing only zeroes contains no `1' thus we
  subtract \(1\) from the possible bitstrings.
\end{itemize}

\textbf{and} \[
\sum_{k=1}^n (n-k)2^{(k-1)} = 2^n - n - 1
\] We count the number of bitstrings of length \(n+1\) containing at
least two '1's.

\begin{itemize}
\item
  \emph{LHS}: For each possible position \(k\), \(1 \leq k \leq n\) in
  the bitstring, we can place a `1' at position \(k\) and place a second
  `1' at some position to the left of \(k\) while fixing the rest of the
  bits toi the left of \(k\) to zero guaranteeing two ones in the
  string, meaning there are \(n-k\) possibilities to place that other
  `1' and for each position of that `1' to the left of \(k\), we have
  \(2^{k-1}\) possibilites to choose the bits to the right of \(k\).
  Thus, we have \((n-k)2^{k-1}k\) possibilites for each fixed `1' at
  each position \(k\).
\item
  \emph{RHS}: There are a total of 2\^{}n bitstrings of length \(n\),
  from which we subtract all strings with no ones, i.e.~the string of
  only zeroes, and all strings with exactly one `1' of which there are
  \(n\) different ones, as we can place the `1' in \(n\) different
  positions and fix `0' in the other ones. Thus, the number of
  bitstrings with at least two '1's, are \(2^n - n -1\).
\end{itemize}

\hypertarget{let-d_n-subset-s_n-be-the-fixed-point-free-permutations-fo-12-dots-n-i.e.-all-permutations-pi-in-s_n-with-pii-neq-i-for-all-i.-the-derangement-numbers-are-defined-as-d_n-d_n.-prove-the-recurrence}{%
\paragraph{\texorpdfstring{36) Let \(D_n \subset S_n\) be the
fixed-point-free permutations fo \(\{1,2, \dots, n\}\), i.e.~all
permutations \(\pi \in S_n\) with \(\pi(i) \neq i\) for all \(i\). The
\emph{derangement numbers} are defined as \(d_n := |D_n|\). Prove the
recurrence:}{36) Let D\_n \textbackslash subset S\_n be the fixed-point-free permutations fo \textbackslash\{1,2, \textbackslash dots, n\textbackslash\}, i.e.~all permutations \textbackslash pi \textbackslash in S\_n with \textbackslash pi(i) \textbackslash neq i for all i. The derangement numbers are defined as d\_n := \textbar D\_n\textbar. Prove the recurrence:}}\label{let-d_n-subset-s_n-be-the-fixed-point-free-permutations-fo-12-dots-n-i.e.-all-permutations-pi-in-s_n-with-pii-neq-i-for-all-i.-the-derangement-numbers-are-defined-as-d_n-d_n.-prove-the-recurrence}}

\[
    \begin{cases}
        d_n = (n-1)(d_{n-1} + d_{n-2}), & \text{for $n \geq 2$}.\\
        d_0 = 1, d_1 = 0.
    \end{cases}
\] \textbf{using a combinatorial interpretation.}

We argue about the size of \(D_n\) by counting its elements. We know
that forall derangements \((\pi(1), \dots, \pi(n)) \in D_n\) it holds
that \(\pi(i) \neq i\) forall \(i \in \{1, \dots, n\}\).

Thus, it also holds that \(\pi(n) \neq n\), i.e.~we know that
\(\pi(n) = i\) for some \(i \in \{1, \dots, n-1\}\), we now distinguish
based on the values of \(\pi(i)\):

\begin{itemize}
\tightlist
\item
  case 1: \(\pi(i) = n\), then \(i\) and \(n\) are mapped to each other
  and the number of derangements surmounts to \(d_{n-2}\).
\item
  case 2: \(\pi(i) \neq n\): we can count the number of derangements by
  setting \(\pi(n) = \pi(i)\) and derangement over \(n-1\) elements,
  therefore the number of derangements here is \(d_{n-1}\).
\end{itemize}

Since we have \(n-1\) options for choosing \(i\), we count: \[
d_n = (n-1) (d_{n-1} + d_{n_2})
\]

\textbf{Furthermore, prove that this recurrence relation implies} \[
d_n = n d_{n-1} + (-1)^n
\] and \[
d_n = n! \sum_{k=0}^n \frac{(-1)^k}{k!}
\]

We first show that \(d_n = n! \sum_{k=0}^n \frac{(-1)^k}{k!}\). We count
the number \(d_n\) of fixed point free permutations by subtracting the
number of permutations which have some fixed point from all
permutations. The former are defined as follows: \[
\sum_{k=1}^n \binom{n}{k} (n-k)! (-1)^{k-1}
\] i.e.~we choose \(k\) elements to fix, permute the remaining elements
and apply the inclusion exclusion principle.

Then the number of derangements can be derived as follows.

\[
d_n = n! - \sum_{k=1}^n \binom{n}{k} (n-k)! (-1)^{k-1} = n!- \sum_{k=1}^n \frac{n!}{k!(n-k)!} (n-k)! (-1)^k-1 =
\] \[
=n! - \sum_{k=1}^n \frac{n!}{k!} (-1)^{k-1} = \sum_{k=0}^n \frac{n!}{k!} (-1)^k = n! \sum_{k=0}^n \frac{(-1)^k}{k!}
\]

We then show \(n! \sum_{k=0}^n \frac{(-1)^k}{k!} = n d_{n-1} + (-1)^n\)
by induction.

\begin{itemize}
\item
  Base case: \(n=1\):
  \(1!(\frac{(-1)^0}{1!}) + 1!(\frac{(-1)^1}{1!}) = 1 \times 1 + (-1)^1 = 0 = 0\)
\item
  Induction hypothesis:
  \(n \times d_{n-1} + (-1)^n = n! \sum_{k=0}^n \frac{(-1)^k}{k!}\)
\item
  Induction step: \[
  \begin{aligned}
  (n+1)d_{n} + (-1)^{n+1} &= (n+1)! \sum_{k=0}^{n+1} \frac{(-1)^k}{k!} \Leftrightarrow\\
  (n+1) d_n + (-1)^{n+1} &= (-1)^{n+1} + (n+1)!\sum_{k=0}^{n} \frac{(-1)^k}{k!}
  \Leftrightarrow \\ (n+1) d_n + (-1)^{n+1} &= 
  (-1)^{n+1} + (n+1) n! \sum_{k=0}^n \frac{(-1)^k}{k!} \Leftrightarrow\\ (n+1) d_n + (-1)^{n+1} &= (-1)^{n+1} +(n+1) d_n
  \end{aligned}
  \]
\end{itemize}

\hypertarget{compute-the-number-of-elements-of-the-set-x-in-mathbbn-1-leq-x-leq-100-000-and-x-is-neither-a-square-nor-a-3rd-4th-or-5th-power-of-some-y-in-mathbbn.}{%
\paragraph{\texorpdfstring{37) Compute the number of elements of the set
\(\{x \in \mathbb{N} \ | \ 1 \leq x \leq 100 000\) and \(x\) is neither
a square nor a 3rd, 4th or 5th power of some
\(y \in \mathbb{N}\}\).}{37) Compute the number of elements of the set \textbackslash\{x \textbackslash in \textbackslash mathbb\{N\} \textbackslash{} \textbar{} \textbackslash{} 1 \textbackslash leq x \textbackslash leq 100 000 and x is neither a square nor a 3rd, 4th or 5th power of some y \textbackslash in \textbackslash mathbb\{N\}\textbackslash\}.}}\label{compute-the-number-of-elements-of-the-set-x-in-mathbbn-1-leq-x-leq-100-000-and-x-is-neither-a-square-nor-a-3rd-4th-or-5th-power-of-some-y-in-mathbbn.}}

We define the following sets:

\begin{itemize}
\item
  \(S_2 = \{x \in \mathbb{N} \ | \ 1 \leq x \leq 100 000 \land \exists y \in \mathbb{N}, x = y^2\} = \{1,2,4, \dots 99856\}\),
  where \(|S_2| = 316\).
\item
  \(S_3 = \{x \in \mathbb{N} \ | \ 1 \leq x \leq 100 000 \land \exists y \in \mathbb{N}, x = y^3\} = \{1,8,27, \dots 97336\}\),
  where \(|S_3| = 46\).
\item
  \(S_4\) does not need to be considered as all its elements are
  included in \(S_2\) already.
\item
  \(S_5 = \{x \in \mathbb{N} \ | \ 1 \leq x \leq 100 000 \land \exists y \in \mathbb{N}, x = y^5\} = \{1,32, \dots 100 000\}\),
  where \(|S_5| = 10\).
\end{itemize}

We observe the intersections of the considered sets in order to apply
the Inclusion-Exclusion Principle:

\begin{itemize}
\item
  \(S_2 \cap S_3 = \{x \in \mathbb{N} . \ | \ 1 \leq x \leq 100 000 \land \exists y \in \mathbb{N}, x = y^6\}\),
  where \(|S_2 \cap S_3| = 6\).
\item
  \(S_3 \cap S_5 = \{x \in \mathbb{N} . \ | \ 1 \leq x \leq 100 000 \land \exists y \in \mathbb{N}, x = y^{15}\}\),
  where \(|S_3 \cap S_5| = 2\).
\item
  \(S_2 \cap S_5 = \{x \in \mathbb{N} \ | \ 1 \leq x \leq 100 000 \land \exists y \in \mathbb{N}, x = y^{10}\}\),
  where \(|S_2 \cap S_5| = 3\).
\end{itemize}

The size of the set is then defined by the Inclusion-Exclusion-Principle
as follows:

\(|S| - |S_2| - |S_3| - |S_5| + |S_2 \cap S_3| + |S_3 \cap S_5| + |S_2 \cap S_5| - |S_2 \cap S_3 \cap S_5|\)
\(= 100 000 - 316 - 46 - 10 + 6 + 2 + 3 - 1 = 99638\)

\hypertarget{section-1}{%
\paragraph{38)}\label{section-1}}

\textbf{Let \(n \in \mathbb{N}\). Prove Pascal's recurrence \[
\binom{n}{k} = \binom{n-1}{k-1} + \binom{n-1}{k}
\] algebraically using the closed \(\binom{n}{k} = \frac{n!}{k!(n-k)!}\)
of the binomial coefficient}

\[
\frac{n!}{k!(n-k)!} = \frac{(n-1)!n}{k!(n-k)!} = (n-1)! \frac{n}{k!(n-k)!} =
\]
\[= (n-1)! \left( \frac{n-k}{k!(n-k)!} + \frac{k}{k!(n-k)!} \right)^\dagger = \frac{(n-1)!(n-k)}{k!(n-k)!} + \frac{(n-1)!k}{k!(n-k)!} =\]
\[
\frac{(n-1)!(n-k)}{k!(n-k)(n-k-1)!} + \frac{(n-1)!k}{k(k-1)!(n-k)!} = \frac{(n-1)!}{k!(n-k-1)!} + \frac{(n-1)!}{(k-1)!(n-k)!}=
\] \[
= \binom{n-1}{k} + \binom{n-1}{k-1}
\] Note that at \(\dagger\) we simply calculated \(+k-k\) which is
equivalent to \(0\) and thus a valid transformation.

\textbf{and prove \[
\sum_{k=0}^n \binom{n}{k}^2 = \binom{2n}{n}
\] using a combinatorial interpretation of the binomial coefficients.}

We formulate the question \emph{How many different subsets of size \(n\)
does a set of size \(2n\) have?}

\begin{itemize}
\item
  \(\binom{2n}{n}\): There are \(\binom{2n}{n}\) possibilites to choose
  \(n\) elements from a set of \(2n\) elements.
\item
  \(\sum_{k=0}^n \binom{n}{k}^2\): We can answer the question by
  dividing the \(2n\) elements into two sets \(A,B\) of size \(n\) each
  for which \(A \cap B = \emptyset\). When we now choose \(n\) elements
  from the original set, we can do this by choosing some from \(A\) and
  some from \(B\).

  The options for choosing are as follows: \(0\) from \(A\), \(n\) from
  \(B\): \(\binom{n}{0} \binom{n}{n}\) \(1\) from \(A\), \(n-1\) from
  \(B\): \(\binom{n}{1} \binom{n}{n-1}\) . . . \(n\) from \(A\), \(0\)
  from \(B\): \(\binom{n}{n}\binom{n}{0}\)

  Therefore, the total number of ways to choose sets of size \(n\) from
  \(A\) and \(B\) is: \[
    \binom{n}{0}\binom{n}{n} + \binom{n}{1} \binom{n}{n-1} \dots + \binom{n}{n}\binom{n}{0}
    \] We observe that \(\binom{n}{0} = \binom{n}{n}\),
  \(\binom{n}{1} = \binom{n}{n-1}\), etc., therefore we can write: \[
    \binom{n}{0}^2+\binom{n}{1}^2 + \dots + \binom{n}{n}^2,
    \] i.e.~\[\sum_{k=0}^2 \binom{n}{k}^2\]

  Since both formulas answer the same question they are equivalent
  according to combinatorial interpretation.
\end{itemize}

\hypertarget{let-nm-in-mathbbn.-give-two-proofs-of-the-identity}{%
\paragraph{\texorpdfstring{Let \(n,m \in \mathbb{N}\). Give two proofs
of the
identity}{Let n,m \textbackslash in \textbackslash mathbb\{N\}. Give two proofs of the identity}}\label{let-nm-in-mathbbn.-give-two-proofs-of-the-identity}}

\[
\sum_{k=0}^{n} \binom{m+k}{k} = \binom{n+m+1}{m+1}
\] \textbf{one by induction:}

\begin{itemize}
\tightlist
\item
  Base case: \(n=1\)
\end{itemize}

\[
\binom{m}{0} + \binom{m+1}{1} = \binom{m+2}{m+1} \Leftrightarrow
\] by Pascal's rule: \[
\Leftrightarrow 1 + \binom{m+1}{1} = \binom{m+1}{m} + \binom{m+1}{m+1} \Leftrightarrow 1+ \binom{m+1}{1} = \binom{m+1}{m} + 1 \Leftrightarrow
\] \[
\Leftrightarrow \binom{m+1}{1} = \binom{m+1}{m}
\] which holds by symmetry (\(\binom{n}{k} = \binom{n}{n-k}\))

\begin{itemize}
\item
  Induction hypothesis: \[
  \sum_{k=0}^{n} \binom{m+k}{k} = \binom{n+m+1}{m+1}
  \] holds for \(n=i, i \in \mathbb{Z}_{+}\).
\item
  Induction step: we start with the LHS: \[
  \sum_{k=0}^{n+1} \binom{m+k}{k} =
  \] we extract the first summand and apply the induction hypothesis: \[
  = \binom{n+m+1}{n+1} + \binom{n+m+1}{m+1} =
  \] by symmetry (\(n = n+m+1, k = n+1\)) \[
  = \binom{n+m+1}{m} + \binom{n+m+1}{m+1}=
  \] by Pascal's rule \[
  = \binom{n+m+2}{m+1}
  \]
\end{itemize}

\textbf{and one by combinatorial interpretation:} \emph{Hint: Consider
0-1 sequences and group them according to the position of the last 1.}

We count the number of bitstrings of length \(n+m+1\) with exactly
\(m+1\) \('1'\)s.

\begin{itemize}
\tightlist
\item
  \emph{RHS:} We choose \(m+1\) positions out of \(n+m+1\) possible ones
  to place the \(m+1\) \('1'\)s.
\item
  \emph{LHS:} We consider the position of the last \('1'\) in the
  bitstring at \(m+k+1\), then \(n-k\) bits are \('0'\) after \(m+k+1\).
  \(k\) bits out of the possible \(m+k\) positions can be \('0'\),
  i.e.~\(\binom{m+k}{k}\) possibilities to set the remaining bits exist.
  Taking the sum for all possible positions of the last \('1'\) from
  \(m+1\) to \(m+n+1\) yields all bitstrings of length \(m+n+1\) with
  \(m+1\).
\end{itemize}

\hypertarget{prove-that-for-all-complex-numbers-x-and-all-k-in-mathbbn-we-have}{%
\paragraph{\texorpdfstring{40) Prove that for all complex numbers \(x\)
and all \(k \in \mathbb{N}\) we
have:}{40) Prove that for all complex numbers x and all k \textbackslash in \textbackslash mathbb\{N\} we have:}}\label{prove-that-for-all-complex-numbers-x-and-all-k-in-mathbbn-we-have}}

\[
\binom{-x}{k} = (-1)^k \binom{x+k-1}{k}
\]

We know for complex numbers \(x\): \[
\binom{-x}{k} = \frac{-x^{\underline{k}}}{k!} =
\] where \(x^{\underline{k}}\) denotes the falling factorials, defined
by: \(x^{\underline{k}} = x * (x-1)* \dots * (x-k+1)\). \[
= \frac{(-x)(-x-1)\times \dots \times (-x-k+1)}{k!} =
\] we can extract \(-1\) from each factor, meaning we extract \(k\)
times \$(-1). \[
= \frac{(-1)^k (x) (x+1) \times \dots \times (x+k-1)} = 
\] we now expand by \(\frac{(x-1)!}{(x-1)!}\), which is equivalent to
\(1\). \[
= (-1)^k \frac{(x)(x+1) \times \dots \times (x+k-1)(x-1)!}{k!(x-1)!} =
\] we observe that the numerator is equivalent to \((x+k-1)!\) which is
then equivalent to the right hand side by definition of the binomial
coefficient. \[
(-1)^k \frac{(x+k-1)!}{k!(x-1)!} = (-1)^k \binom{x+k-1}{k}
\]

\hypertarget{exercise-sheet-5}{%
\subsection{Exercise Sheet 5}\label{exercise-sheet-5}}

\hypertarget{let-s_nk-be-the-stirling-numbers-of-the-first-kind.-prove-s_n2-n-1-h_n-1-for-n-geq-2-where-h_n-sum_k1n-frac1k.}{%
\paragraph{\texorpdfstring{41) Let \(s_{n,k}\) be the Stirling numbers
of the first kind. Prove \(s_{n,2} = (n-1) H_{n-1}\) for \(n \geq 2\),
where
\(H_n = \sum_{k=1}^n \frac{1}{k}\).}{41) Let s\_\{n,k\} be the Stirling numbers of the first kind. Prove s\_\{n,2\} = (n-1) H\_\{n-1\} for n \textbackslash geq 2, where H\_n = \textbackslash sum\_\{k=1\}\^{}n \textbackslash frac\{1\}\{k\}.}}\label{let-s_nk-be-the-stirling-numbers-of-the-first-kind.-prove-s_n2-n-1-h_n-1-for-n-geq-2-where-h_n-sum_k1n-frac1k.}}

By definition: \(s_{n,k} = s_{(n-1, k-1)} + (n-1) s_{(n-1,k)}\)

We proceed by induction on \(n\): * Base case: \(n=2\), then
\(s_(2,2) = 1! H_1 \Leftrightarrow 1 = 1\) * Induction Hypothesis:
\(s_{(n,2)} = (n-1)! H_{n-1}\) * Induction step: \[
        s_{(n+1,2)} = n! H_n \Leftrightarrow s_{(n,1)} + n s_{(n,2)} = n! H_n \Leftrightarrow
        \] by IH and \(s_{(n,1)} = (n-1)(n-2) \times \dots \times 1\) \[
        \Leftrightarrow (n-1)! + n (n-1)! H_{n-1} = n! H_n \Leftrightarrow (n-1)! + n! H_{n-1} = n! H_n \Leftrightarrow
        \] now we divide by \(n!\) \[
        \Leftrightarrow \frac{1}{n} + H_{n-1} = H_n \Leftrightarrow \frac{1}{n} + \sum_{k=1}^{n-1} \frac{1}{k} = H_n \Leftrightarrow \sum_{k=1}^{n} \frac{1}{k} = H_n \Leftrightarrow H_n = H_n
        \]

\hypertarget{let-s_nk-be-the-stirling-numbers-of-the-second-kind-that-is-the-number-of-partitions-of-the-set-12-dots-n-into-k-non-empty-subsets.-set-f_kx-sum_n-geq-k-s_nk-xn.-prove-that}{%
\paragraph{\texorpdfstring{42) Let \(S_{n,k}\) be the Stirling numbers
of the second kind, that is, the number of partitions of the set
\(\{1,2, \dots, n\}\) into \(k\) (non-empty subsets). Set
\(F_k(x) := \sum_{n \geq k} S_{n,k} x^n\). Prove
that:}{42) Let S\_\{n,k\} be the Stirling numbers of the second kind, that is, the number of partitions of the set \textbackslash\{1,2, \textbackslash dots, n\textbackslash\} into k (non-empty subsets). Set F\_k(x) := \textbackslash sum\_\{n \textbackslash geq k\} S\_\{n,k\} x\^{}n. Prove that:}}\label{let-s_nk-be-the-stirling-numbers-of-the-second-kind-that-is-the-number-of-partitions-of-the-set-12-dots-n-into-k-non-empty-subsets.-set-f_kx-sum_n-geq-k-s_nk-xn.-prove-that}}

\[
F_1(x) = \frac{x}{1-x}
\]

First, \(S_{(n,1)} = 1\) since there is only one way to partition a set
into size \(1\) subsets. Thus: \[
F_1 (x) = \sum_{n \geq 1} x^n = x \sum_{n \geq 0} x^n = x \frac{1}{1-x} = \frac{x}{1-x}
\]

\textbf{and} \[
F_2(x) = \frac{x^2}{(1-x)(1-2x)}
\]

\[
F_2(x) = \sum_{n \geq 2} S_{n,2} x^n = \sum_{n \geq 2} (2^{n-1} - 1) x^n = \sum_{n \geq 2} 2^{n-1} x^n - x^n =
\] we apply an index shift \[
= \sum_{n \geq 1} 2^nx^{n+1} - x^{n+1} = (\sum_{n \geq 1} 2^n x^{n+1})^\dagger - (\sum_{n \geq 1} x^{n+1})^\heartsuit =
\] \[
= \frac{2x^2}{1-2x} - \frac{x^2}{1-x} = \frac{2x^2(1-x)}{(1-2x)(1-x)} - \frac{2x^2-2x^3-x^2+2x^3}{(1-2x)(1-x)} = \frac{x^2}{(1-x)(1-2x)},
\] where \[
\dagger = \sum_{n \geq 1} x^{n+1} = x^2 \sum_{n \geq 0} x^n = x^2 \frac{1}{1-x} = \frac{x^2}{1-x}
\] and \[
\heartsuit =  \sum_{n \geq 1} 2^n x^{n+1} = 2x^2 \sum_{n \geq 0} 2^n x^n = 2x^2 \frac{1}{1-2x} = \frac{2x^2}{1-2x}
\] \textbf{Furthermore, show that the functions \(F_k(x)\) satisfy the
recurrence relation \(F_k(x) = \frac{x}{1-kx}F_{k-1}(x)\)} we start with
the right hand side: \[
\frac{x}{1-kx}F_{k-1}(x) = \frac{x}{1-kx} \sum_{n \geq k-1} S_{n,k-1} x^n =
\] we replace by the series for the generating function: \[
x \sum_{n \geq 0} k^n x^n \sum_{n \geq k-1} S_{n,k-1} x^n =
\] we apply the cauchy product with \(a_i = S_{i,k-1}, b_j = k^j\) \[
x \sum_{n \geq k-1} \sum_{i=k-1}^n S_{i, k-1} k^{n-i} x^n =
\] drag the x into the sum and use the lemma:
\(S_{n+1,k+1} = \sum_{j=k}^n (k+1)^{n-j} S_{j,k}\) \[
\sum_{n\geq k-1} S_{n+1,k} x^{n+1}=
\] we apply an index shift in the sum \[
\sum_{n \geq k} S_{n,k} x^n
\]

\textbf{and solve the recurrence.} Starting with: \[
F_k(x) = \frac{x}{1-kx} F_{k-1}(x)
\] we substitute repeatedly:

\[
F_k(x) = \frac{x}{1-kx} \frac{x}{1-(k-1)x}  F_{k-2}(x)
\] we apply the substitution until the last term \(F_1\) \[
F_k(x) = \frac{x^{k-1}}{(1-kx)(1-(k-1)x) \cdots (1-2x)} F_1(x)
\] we substitute the definition of \(F_1 = \frac{x}{1-x}\) and multiply
it into the term. \[
F_k(x) = \frac{x^{k}}{(1-kx)(1-(k-1)x)\cdots (1-2x) (1-x)} 
\] \#\#\#\# 43) Prove the following identity: \[
x^n = \sum_{k=0}^{n} S_{n,k} (x)_{k} \quad (n \geq 0)
\] We proceed by induction:

\begin{itemize}
\item
  Base case: \(n=1\) \[
  x^1 = S_{1,0} (x)_0 + S_{1,1} (x)_1 = 0 + 1 \times x
  \]
\item
  Induction hypothesis: For any case \(n=1, i \in \mathbb{N}\): \[
  x^n = \sum_{k=0}^n S_{n,k}(x)_k
  \]
\item
  Induction step: We show that the statement holds by starting with the
  left hand side \[
  x^n = x \times x^{n-1} =
  \] Let \(i = n-1\), we apply the induction step for \(i+1=n\): \[
  = x \sum_{k=0}^{n-1} S_{n-1,k} (x)_{k} = 
  \] we start the sum at 1 since \(S_{n-1,0}\), thus the first summand
  does not affect the sum's value
\end{itemize}

\[
= \sum_{k=1}^{n-1}S_{n-1,k} (x)_k x =
\]

by \(x_{k+1} = x_k (x-k) = x \times x_k - kx_k\) we get
\(x \times x_k = x_{k+1} + k x_k\) by transforming the equation, which
we can apply as follows:

\[
= \sum_{k=1}^{n-1} S_{n-1,k}( (x)_{k+1} + k(x)_k ) = \sum_{k=1}^{n-1} S_{n-1,k} (x)_{k+1} +  S_{n-1,k} k(x)_k  = \sum_{k=1}^{n-1} S_{n-1,k} (x)_{k+1} + \sum_{k=1}^{n-1} S_{n-1,k} k (x)_k =
\] let \(k' = k+1\), we set \(k+1 = k'\): \[
= \sum_{k'=2}^{n} S_{n-1,k'-1} (x)_{k'} + \sum_{k=1}^{n-1} S_{n-1,k} k (x)_k =
\] since \(S_{n-1,0} = S_{n-1,n} = 0\): \[
= \sum_{k=1}^{n} S_{n-1,k-1} (x)_{k} + \sum_{k=1}^{n} S_{n-1,k} k (x)_k = \sum_{k=1}^{n} S_{n-1,k-1} (x)_{k} + k S_{n-1,k} (x)_k = \sum_{k=1}^{n} (S_{n-1,k-1} + k S_{n-1,k}) (x)_k =
\] this is the definition of the recurrence relation of the Stirling
numbers of the second kind: \[
= \sum_{k=1}^{n} S_{n,k} (x)_k = \sum_{k=0}^{n} S_{n,k} (x)_k
\]

\hypertarget{let-ab-be-two-finite-sets-with-a-n-and-b-k.-how-many-injective-mappings-f-a-rightarrow-b-are-there}{%
\paragraph{\texorpdfstring{44) Let \(A,B\) be two finite sets with
\(|A| = n\) and \(|B| = k\). How many injective mappings
\(f: A \rightarrow B\) are
there?}{44) Let A,B be two finite sets with \textbar A\textbar{} = n and \textbar B\textbar{} = k. How many injective mappings f: A \textbackslash rightarrow B are there?}}\label{let-ab-be-two-finite-sets-with-a-n-and-b-k.-how-many-injective-mappings-f-a-rightarrow-b-are-there}}

We assume \(k \geq n\) since otherwise no mapping is possible. For the
first element in \(n\), there are \(k\) possibilities. For the second
one, \(k-1\), etc\ldots{} For the last element in \(n\) there are still
\(k-(n-1)\) choices, thus we have \(k(k-1)\times \dots \times (k-n+1)\)
mappings, this is the falling factorial \(k^{\underline{n}}\) which
surmounts to \[
\frac{k!}{(k-n)!}
\] \textbf{Furthermore, show that the number of surjective mappings
\(f: A \rightarrow B\) equals \(k!S_{n,k}\).}

Assume \(n \geq k\), otherwise no such mapping is possible. We want all
possible partitions of \(A\) into \(k\) non-empty subsets which are
calculated by \(S(n,k)\). For each of these non-empty partitions of
elements in \(A\) we select one item in \(B\) for the elements in the
subset of \(A\) to map to. For the first partition we have \(k\) options
from \(B\), \(k-1\) for the second, etc. and for the last element in
\(B\), 1 option remains. Thus \(k!\) options per partitioning exist,
i.e.~\(k!S_{n,k}\).

\hypertarget{compute}{%
\paragraph{46) Compute:}\label{compute}}

\[
a_n = \sum_{k=0}^{n} k^2
\] using generating functions: Firstly, we know: \[
\sum_{n \geq 0} z^n = \frac{1}{1-z}
\] we take the derivative, applying the quotient rule and the chain rule
\[
\sum_{n \geq 0} n z^{n-1} = \frac{1'(1-z) - 1(1-z)'}{(1-z)^2}
\] \[
\sum_{n \geq 0} n z^{n-1} = \frac{0 - (-1)}{(1-z)^2}
\] we multiply by \(z\) in order to preserve \(z^n\) on the left side \[
\sum_{n \geq 0} n z^{n} = \frac{z}{(1-z)^2}
\] we take the derivative, again applying quotient and chain rule \[
\sum_{n \geq 0} n^2 z^{n-1} = \frac{z'(1-z)^2 - z ((1-z)^2)'}{(1-z)^4}
\] \[
\sum_{n \geq 0} n^2 z^{n-1} = \frac{(1-z)^2 +2z(1-z)}{(1-z)^4}
\] \[
\sum_{n \geq 0} n^2 z^{n-1} = \frac{1-z+2z}{(1-z)^3}
\] \[
\sum_{n \geq 0} n^2 z^{n-1} = \frac{1+z}{(1-z)^3}
\] we multiply by \(z\) \[
\sum_{n \geq 0} n^2 z^{n} = \frac{z+z^2}{(1-z)^3}
\] Now, let \(\tilde F(z)\) be the generating function for coefficient
\(n^2\), of which we just calculated the generating function. We take
the sum with coefficient \(1\) of which we know the GF
(\(\frac{1}{1-z}\)) and apply the cauchy product: \[
\sum_{i \geq 0} i^2z^n \times \sum_{j \geq 0} z^n = \sum_{n \geq 0} \sum_{k=0}^{n} k^2 z^n 
\] Therefore: \[
\tilde F(z) \times \frac{1}{1-z} = F(z)
\] we substitute the previously calculated GF of \(\tilde F(z)\) \[
F(z) = \frac{z+z^2}{(1-z)^3}\times \frac{1}{1-z} = \frac{z+z^2}{(1-z)^4}
\] Now we want to compute \([z^n]F(z)\), i.e. \[
[z^n]\frac{z+z^2}{(1-z)^4}
\] \[
[z^n]\frac{z}{(1-z)^4} + [z^n]\frac{z^2}{(1-z)^4}
\] we apply an index shift, to remove the \(z\) and \(z^2\) respectively
\[
[z^{n-1}]\frac{1}{(1-z)^4} + [z^{n-2}]\frac{1}{(1-z)^4}
\] \[
[z^{n-1}] (1-z)^{-4} + [z^{n-2}] (1-z)^{-4}
\] we apply the generalised binomial theorem \[
[z^{n-1}] \sum_{n \geq 0} \binom{-4}{n}(-1)^{-4-n}+z^n + [z^{n-2}] \sum_{n \geq 0} \binom{-4}{n}(-1)^{-4-n}+z^n
\] now for the coefficient extraction we put in the respective indices
\[
\binom{-4}{n-1}(-1)^{-4-n-1} + \binom{-4}{n-2}(-1)^{-4-n-2}
\] we apply some arithmetic equalities regarding the powers of \(-1\) \[
\binom{-4}{n-1}(-1)^{n-1} + \binom{-4}{n-2}(-1)^{n-2}
\] when writing out the respective quotients, we can drag the factors
\(-1\) into the quotients \[
\frac{(-4)(-5) \dots (-4-(n-2)) (-1)^{n-1}}{(n-1)! } + \frac{(-4)(-5) \dots (-4-(n-3)) (-1)^{n-2}}{(n-2)!} =
\] \[
= \frac{(4)(5) \dots (n - 2 + 4)}{(n-1)!} + \frac{(4)(5) \dots (n - 3 + 4)}{(n-2)!} = \frac{(4)(5) \dots (n + 2)}{(n-1)!} + \frac{(4)(5) \dots (n + 1)}{(n-2)!} = 
\]

\[
= \binom{n+2}{3} \binom{n+1}{3} = \frac{(n+2)(n+1)n + (n+1)n(n-1)}{3!} = \frac{(2n+1)(n+1)n}{3!}
\]

\hypertarget{prove-the-following-identity}{%
\paragraph{48) Prove the following
identity:}\label{prove-the-following-identity}}

\[
\sum_{n \geq 0} \binom{2n}{n}z^n = \frac{1}{\sqrt{1-4z}}.
\]

We start with the right hand side:

\[
\frac{1}{\sqrt{1-4z}} = (1-4z)^{-\frac{1}{2}} = 
\] by the generalised binomial theorem with \(x = 1\) and \(y=-4z\),
since \(x\) is \(1\), only \(y\) is considered. \[
= \sum_{n=0}^{\infty} \binom{-\frac{1}{2}}{n} (-4z)^n = \sum_{n=0}^{\infty} \binom{-\frac{1}{2}}{n} (-4)^n z^n =
\] again, by the theorem to handle \(-\frac{1}{2}\) in the upper index
\[
= \sum_{n=0}^{\infty} \frac{-\frac{1}{2} * (-\frac{1}{2} - 1) \times \dots \times (-\frac{1}{2} - n +1)}{n!} (-4^n) z^n =
\] we extract \((-\frac{1}{2}^n)\) from each factor \[
= \sum_{n=0}^{\infty} (-4)^n (-\frac{1}{2}^n) \frac{1 \times 3 \times (2n - 1)}{n!} z^n = 
\] all odds as factorials are \(!!\) \[
= \sum_{n=0}^{\infty} 2^n \frac{(2n-1)!!}{n!} z^n = 
\] since \((2n-1)!! = \frac{(2n)!}{2^n n!}\) by definition \[
\sum_{n=0}^{\infty} 2^n \frac{\frac{(2n)!}{2^n n!}}{\frac{n!}{1}} z^n = \sum_{n=0}^{\infty} 2^n \frac{(2n)!}{2^n n! n!} z^n = \sum_{n=0}^{\infty} \binom{2n}{n} z^n
\]

\hypertarget{compute-1}{%
\paragraph{49) Compute}\label{compute-1}}

\[
[z^n] \frac{2+5z}{\sqrt{1-8z}}
\] where \([z^n]\sum_{n \geq 0} a_n z^n := a_n\) is the coefficient
extraction operator. \[
\frac{2+5z}{\sqrt{1-8z}} = (2+5z^2) \frac{1}{\sqrt{1-8z}} =
\] we apply the generalised binomial theorem for \((1-8z)^{-0.5}\) \[
= (2+5z^2) \sum_{k=0}^{\infty} \binom{-0.5}{k} (-8z)^k = (2+5z^2) \sum_{k\geq 0} \binom{-0.5}{k} (-1)^k 8^k z^k =
\] \[
= 2 \sum_{k\geq 0} \binom{-0.5}{k} (-1)^k 8^k z^k + 5z^2 \sum_{k\geq 0} \binom{-0.5}{k} (-1)^k 8^k z^k =
\]

\[
= 2 \sum_{k\geq 0} \binom{-0.5}{k} (-1)^k 8^k z^k + 5 \sum_{k\geq 0} \binom{-0.5}{k} (-1)^k 8^k z^{k+2} =
\] but we want \(z^k\) for coefficient extraction

\[
= 2 \sum_{k\geq 0} \binom{-0.5}{k} (-1)^k 8^k z^k + 5 \sum_{k\geq 2} \binom{-0.5}{k-2} (-1)^{k-2} 8^{k-2} z^{k} =
\] we want the same indices in both sums so we extract the first two
elements on the left sum \[
= 2 + 8z + (2 \sum_{k\geq 2} \binom{-0.5}{k} (-1)^k 8^k z^k + 5 \sum_{k\geq 2} \binom{-0.5}{k-2} (-1)^{k-2} 8^{k-2} z^{k}  =
\]

\[
= 2 + 8z + \sum_{k\geq 2} \left(\binom{-0.5}{k} (-1)^k 8^k 2 \right) + \left( \binom{-0.5}{k-2} (-1)^{k-2} 8^{k-2} 5 \right) z^{k}  =
\] therefore (since \((-1)^k = (-1)^{k-2}\)): \[
[z^n]= \begin{cases}
      2, & \text{if}\ k=0 \\
      8, & \text{if}\ k=1 \\
      \left(\binom{-0.5}{k} (-1)^k 8^k 2 \right) + \left( \binom{-0.5}{k-2} (-1)^{k} 8^{k-2} 5 \right)  &  \text{if}\ k \geq 2
\end{cases}
\] \#\#\#\# 50) Solve the following recurrence using generating
functions.

\(a_{n+1} = 3 a_n -2\), for \(n \geq 0, a_0 = 2\)

\[
\sum_{n=0}^{\infty} a_{n+1} z^{n+1} = 3 \sum_{n=0}^{\infty} a_{n} z^{n+1} - 2 \sum_{n=0}^{\infty} z^{n+1} \Leftrightarrow
\] we want \(z^n\) in all of our sums \[
\Leftrightarrow \sum_{n=1}^{\infty} a_{n} z^{n} = 3z \sum_{n=1}^{\infty} a_{n} z^{n} - 2z \sum_{n=0}^{\infty} z^{n} \Leftrightarrow
\] we can transform the sums to their respective generating functions on
the right side, the first sum is simple, for the second we need to
consider that \(a_n = 1\) so the generating function is
\(\frac{1}{1-z}\) \[
\Leftrightarrow (\sum_{n=1}^{\infty} a_{n} z^{n}) + a_0z^0 - a_0z^0 = 3z F(z) - 2 \frac{z}{1-z} \Leftrightarrow
\] now we want each sum to start at \(0\), so we add the \(0\)th summand
to the left sum \[
\Leftrightarrow (\sum_{n=1}^{\infty} a_{n} z^{n}) + a_0z^0 - a_0z^0 = 3z F(z) - 2 \frac{z}{1-z} \Leftrightarrow
\] we can drag the \(0\)-th summand into the sum, to shift the index \[
\Leftrightarrow (\sum_{n=0}^{\infty} a_{n} z^{n}) - a_0z^0 = 3z F(z) - 2 \frac{z}{1-z} \Leftrightarrow
\] we again replace by the generating function \[
\Leftrightarrow F(z) - a_0 = 3z F(z) - 2 \frac{z}{1-z} \Leftrightarrow  \]

\[
\Leftrightarrow F(z) - 3z F(z) =  - 2 \frac{z}{1-z} + a_0 
\]

\[
\Leftrightarrow F(z) - 3z F(z) = - \frac{2z}{(1-z)(1-3z)} + \frac{a_0}{1-3z} \Leftrightarrow
\] we apply partial fraction decomposition \[
\Leftrightarrow F(z) =  \frac{1}{1-z} - \frac{1}{1-3z} + \frac{a_0}{(1-3z)} \Leftrightarrow
\] \[
\Leftrightarrow F(z) =  (a_0  - 1) \frac{1}{(1-3z)} + \frac{a_0}{(1-3z)} \Leftrightarrow
\] \[
\Leftrightarrow F(z) =  (a_0  - 1) \sum_{n=0}^{\infty} 3^n z^n + \sum_{n=0}^{\infty} z^n \Leftrightarrow
\] \[
\Leftrightarrow F(z) = \sum_{n=0}^{\infty} (((a_0  - 1)3n)+1) z^n
\]

\hypertarget{exercise-sheet-6}{%
\subsection{Exercise Sheet 6}\label{exercise-sheet-6}}

\hypertarget{a-t-ary-tree-is-a-plane-rooted-tree-such-that-every-node-has-either-t-or-0-successors.-a-node-with-t-successors-is-called-internal-node.}{%
\paragraph{\texorpdfstring{51) A \(t\)-ary tree is a plane rooted tree
such that every node has either \(t\) or \(0\) successors. A node with
\(t\) successors is called internal
node.}{51) A t-ary tree is a plane rooted tree such that every node has either t or 0 successors. A node with t successors is called internal node.}}\label{a-t-ary-tree-is-a-plane-rooted-tree-such-that-every-node-has-either-t-or-0-successors.-a-node-with-t-successors-is-called-internal-node.}}

\begin{itemize}
\item
  How many leaves does a \(t\)-ary tree with \(n\) internal nodes have?
  We can count the number of nodes in our tree by counting all nodes
  with a parent and the nodes without a parent, the latter simply being
  the root. Each internal node has to have \(t\) children, thus we have:
  \[
    t \times n +1
    \] nodes in total, meaning we have: \[
    (t n + 1) - n = t (n-1) + 1
    \] leaves, by removing the internal nodes from the total nodes.
\item
  Moreover, let \(a_n\) be the number of \(t\)-ary trees with \(n\)
  internal nodes and \(A(z)\) the generating function of this sequence.
  Find a functional equation for \(A(z)\)! We can construct a \(t\)-ary
  tree with \(n\) internal nodes by taking \(t\) \(t\)-ary trees which
  have \(n-1\) internal nodes all together and attaching them to a root
  node.
\end{itemize}

\[
a_n = \sum_{n_1, \dots, n_t} a_{n_1} \times \dots \times a_{n_t},
\] where \((n_1, \dots, n_t)\) are all possible combinations such that
\(n_1 + \dots n_t = n-1\)

We define \(a_0=1\) since there exists exactly one \(t-ary\) tree with
no internal nodes, namely the one consisting of only the root. Then,
since \(A(z) = \sum_{n \geq 0} a_n z^n\), by definition of generating
functions.

Furthermore, we know \[
(A(z))^t = (\sum_{n \geq 0} a_{n+1}z^n)^t = \sum_{n \geq 0} c_n z^n,
\] where \[
c_n = \sum_{n_1 + \dots + n_t = n} a_{n_1} \times \dots \times a_{n_t}
\] which can be observed from the definition of the cauchy product.

We observe that \(c_n=a_{n+1}\), hence

\[
(A(z))^t = \sum_{n \geq 0} a_{n+1}z^n
\] and \[
\sum_{n \geq 1} a_n z^{n-1} = \sum_{n \geq 1} a_n z^{n-1} + a_0 - a_0 = \frac{1}{z} \sum_{n \geq 0} a_n z^n - a_0
\]

Thus, our functional equation for \(A(z)\) is defined by: \[
(A(z))^t = \frac{A(z)-a_0}{z} \Leftrightarrow (A(z))^t \times z + 1 = A(z)
\]

\hypertarget{compute-the-number-t_n-of-plane-rooted-trees-with-n-nodes-which-can-be-described-by-the-equation}{%
\paragraph{\texorpdfstring{52) Compute the number \(t_n\) of plane
rooted trees with \(n\) nodes which can be described by the
equation:}{52) Compute the number t\_n of plane rooted trees with n nodes which can be described by the equation:}}\label{compute-the-number-t_n-of-plane-rooted-trees-with-n-nodes-which-can-be-described-by-the-equation}}

\begin{Shaded}
\begin{Highlighting}[]
\NormalTok{graph graphname \{ }
\NormalTok{rank=same}
\NormalTok{y[label="T =", shape=plaintext]}
\NormalTok{subgraph cluster\_0 \{}
\NormalTok{style=invis;}
\NormalTok{a[label="",shape=circle]}
\NormalTok{b[label="",shape=circle]}
\NormalTok{c[label="",shape=circle]}
\NormalTok{d[label="",shape=circle]}
\NormalTok{        a {-}{-} b; }
\NormalTok{        a {-}{-} c;}
\NormalTok{        c {-}{-} d;}
\NormalTok{    \}}
\NormalTok{x[label="+", shape=plaintext]}
\NormalTok{subgraph cluster\_1 \{}
\NormalTok{style=invis}
\NormalTok{e[label="", shape=circle]}
\NormalTok{f[label="", shape=circle]}
\NormalTok{g[label="T", shape=circle]}
\NormalTok{h[label="T", shape=circle]}
\NormalTok{        e {-}{-} g;}
\NormalTok{        e {-}{-} f; }
\NormalTok{        f {-}{-} h;}
\NormalTok{    \}}
\NormalTok{\} }
\end{Highlighting}
\end{Shaded}

Then \(t_4 = 1\) and \(t_n = \sum_{i+j=n-2} t_i \times t_j\), for
\(n \geq 4\).

We observe that \[
t_{n+2} = \sum_{i+j=n} t_i \times t_j = c_n, \quad n > 2
\] where, since,

\[
(A(z))^2 = \sum_{n > 2} t_{n+2} z^n= A(z) \times A(z) = \sum_{n \geq 0} t_n z^n \times \sum_{n \geq 0} t_n z^n = \sum_{n \geq 0} (\sum_{l=0}^{n} t_l \times t_{n-l})z^n =
\] \[
= \sum_{n \geq 0} (\sum_{i+j=n} t_i \times t_j)z^n = \sum_{n > 2} t_{n+2} z^n =
\] \[
(\sum_{n > 2} t_{n+2} z^n) + t_4 z^2 - t_4 z^2 = (\sum_{n \geq 2} t_{n+2} z^n) - z^2 =
\] \[
= z^{1/2} \sum_{n \geq 2} t_{n+2} z^{n+2}) - z^2 =
\] since for \(n = 0,1,2,3\), \(t_n = 0\) \[
= z^{1/2} \sum_{n \geq 0} t_{n} z^{n}) - z^2 
\]

Then, we have \((A(z))^2 = z^2 A(z) - z^2\), we solve the quadratic
equation:

\[
A(z) = \frac{1}{2z^2} \pm \sqrt{\frac{1}{4z^4} - z^2}
\] refactor to get the same enumerator: \[
A(z) = \frac{1-\sqrt{1- 4z^6}}{2z^2}
\] where the negative square root denotes a common generating function,
thus we substitute \[
A(z) = \frac{1}{2z^2} \sum_{n \geq 1} \binom{\frac{1}{2}}{n} (-4^n) z^{6n} =
\]

we know the \(0\)th series number is \(1\): \[
\sum_{n \geq 1} -\frac{1}{2} \binom{1/2}{n} (-4^n) z^{6n -2}
\] we can then define \(t_n\):

\[  
t_n =
    \begin{cases}
      - \binom{\frac{1}{2}}{k} (-4^k) \frac{1}{2} & \text{if } n = 6k-2 \text{ for } k \in \mathbb{N}\\
      0 & \text{otherwise}
    \end{cases}       
\] \#\#\#\# 53) Compute the number \(t_n\) of plane rooted trees with
\(n\) nodes. Since we are dealing with plane-rooted trees, left-right
order matters, these trees can be described by the root combined with a
sequence of subtrees: \[
P = \{\circ \} \times seq(P)
\] We apply the relevant generating functions: \[
P(z) = \frac{z}{1- P(z)}
\] we multiply by the denominator \[
P(z) - (P(z))^2 = z
\] and solve the quadratic equation, taking the positive result as
\(z_0=0\) \[
P(z) = \frac{1-\sqrt{1-4z}}{2}
\] when comparing this to binary trees, only factor \(1/z\) is missing
for binary trees, thus we can state that \(1/zP(z)\) is equivalent to
the generating function for the Catalan numbers.

The number of plane-rooted trees with \(n\) nodes can thus be defined by
coefficient extraction of the Catalan numbers: \[
[z^n]P(z) = [z^{n-1}] C(z) = \frac{1}{n} \binom{2(n-1)}{n-1} \quad \text{, for } n\geq 1 
\] since \[
C_n = \frac{1}{n+1} \binom{2n}{n}
\]

\hypertarget{consider-the-following-context-free-grammar-s-rightarrow-asbepsilon.-this-defines-a-formal-language-mathcall-which-consists-of-all-words-w-over-the-alphabet-sigma-ab-such-that-either-a-w-starts-with-a-followed-by-a-word-from-mathcall-then-a-b-follows-which-is-itself-followed-by-another-word-of-mathcall-or-b-w-is-the-empty-word.-compute-the-number-of-words-in-mathcall-that-consist-of-n-letters.-do-this-by-finding-a-combinatorial-structure-that-specifies-mathcall-and-analysing-the-generating-function-of-that-structure.}{%
\paragraph{\texorpdfstring{54) Consider the following context-free
grammar: \(S \rightarrow aSb|\epsilon\). This defines a formal language
\(\mathcal{L}\) which consists of all words \(w\) over the alphabet
\(\Sigma = \{a,b\}\) such that either (a) \(w\) starts with \(a\)
followed by a word from \(\mathcal{L}\), then a \(b\) follows, which is
itself followed by another word of \(\mathcal{L}\), or (b) \(w\) is the
empty word. Compute the number of words in \(\mathcal{L}\) that consist
of \(n\) letters. Do this by finding a combinatorial structure that
specifies \(\mathcal{L}\) and analysing the generating function of that
structure.}{54) Consider the following context-free grammar: S \textbackslash rightarrow aSb\textbar\textbackslash epsilon. This defines a formal language \textbackslash mathcal\{L\} which consists of all words w over the alphabet \textbackslash Sigma = \textbackslash\{a,b\textbackslash\} such that either (a) w starts with a followed by a word from \textbackslash mathcal\{L\}, then a b follows, which is itself followed by another word of \textbackslash mathcal\{L\}, or (b) w is the empty word. Compute the number of words in \textbackslash mathcal\{L\} that consist of n letters. Do this by finding a combinatorial structure that specifies \textbackslash mathcal\{L\} and analysing the generating function of that structure.}}\label{consider-the-following-context-free-grammar-s-rightarrow-asbepsilon.-this-defines-a-formal-language-mathcall-which-consists-of-all-words-w-over-the-alphabet-sigma-ab-such-that-either-a-w-starts-with-a-followed-by-a-word-from-mathcall-then-a-b-follows-which-is-itself-followed-by-another-word-of-mathcall-or-b-w-is-the-empty-word.-compute-the-number-of-words-in-mathcall-that-consist-of-n-letters.-do-this-by-finding-a-combinatorial-structure-that-specifies-mathcall-and-analysing-the-generating-function-of-that-structure.}}

\begin{Shaded}
\begin{Highlighting}[]
\NormalTok{graph graphname \{ }
\NormalTok{splines=false}
\NormalTok{a[label="",shape=point]}
\NormalTok{b[label="",shape=point]}
\NormalTok{c[label="",shape=point]}
\NormalTok{d[label="",shape=point, style=invis]}
\NormalTok{e[label="",shape=point, style=invis]}
\NormalTok{f[label="",shape=point, style=invis]}
\NormalTok{g[label="",shape=point, style=invis]}

\NormalTok{        a {-}{-} b[label="a"]; }
\NormalTok{        a {-}{-} c[label="b"];}
\NormalTok{        c{-}{-}d[shape=none];}
\NormalTok{        c{-}{-}e[shape=none];}
\NormalTok{        b{-}{-}g[shape=none];}
\NormalTok{        b{-}{-}f[shape=none];}

\NormalTok{\} }
\end{Highlighting}
\end{Shaded}

where the words of length \(0\) have one possibility and there exist no
words of odd length, since exerytime we write \(a\), we write a \(b\).

We can easily show that the language represented by this grammar can be
translated to \(D_1\), where \(D_1\) is the Dyck language over one set
of parentheses, we simply translate \(a \rightarrow (\) and
\(b \rightarrow )\). Since we know that \(C_n\) is the number of Dyck
words of length \(2n\), where \(C_n\) is the \(n\)th catalan number, we
can define the number of words in \(\mathcal{L}\) with \(n\) letters by:
\[
[z^{n}] D_z = \frac{1}{\frac{n}{2} +1} \binom{n}{\frac{n}{2}}
\], for even \(n \geq0\) and by \(0\) otherwise.

\hypertarget{consider-a-regular-n2-gon-a-say-with-the-vertices-01-dots-n1.-a-triangulation-is-a-decomposition-of-a-into-n-triangles-such-that-the-3-vertices-of-each-triangle-are-vertices-of-a-as-well.-show-that-the-set-mathcalt-of-triangulations-of-regular-polygons-can-be-described-as-a-combinatorial-construction-satisfying}{%
\paragraph{\texorpdfstring{55) Consider a regular \((n+2)\)-gon \(A\),
say, with the vertices \(0,1, \dots, n+1\). A triangulation is a
decomposition of \(A\) into \(n\) triangles such that the \(3\) vertices
of each triangle are vertices of \(A\) as well. Show that the set
\(\mathcal{T}\) of triangulations of regular polygons can be described
as a combinatorial construction
satisfying}{55) Consider a regular (n+2)-gon A, say, with the vertices 0,1, \textbackslash dots, n+1. A triangulation is a decomposition of A into n triangles such that the 3 vertices of each triangle are vertices of A as well. Show that the set \textbackslash mathcal\{T\} of triangulations of regular polygons can be described as a combinatorial construction satisfying}}\label{consider-a-regular-n2-gon-a-say-with-the-vertices-01-dots-n1.-a-triangulation-is-a-decomposition-of-a-into-n-triangles-such-that-the-3-vertices-of-each-triangle-are-vertices-of-a-as-well.-show-that-the-set-mathcalt-of-triangulations-of-regular-polygons-can-be-described-as-a-combinatorial-construction-satisfying}}

\[
\mathcal{T} = \{\epsilon\} \cup \mathcal{T} \times \Delta \times \mathcal{T}
\] \textbf{where \(\Delta\) denotes a single triangle and \(\epsilon\)
denotes the empty triangulation (consisting of no triangle and
corresponding to the case \(n=0\)).}

Order matters here, since we have labelled vertices. We have \(0\)
triangulations for \(n=0\), since we then only have 2 vertices.
\[\{\epsilon\}\] Otherwise, we can split our \((n+2)\)-gon into a
triangle and the remaining shapes to be triangulated, hence: \[
\mathcal{T} \times \Delta \times \mathcal{T}
\]

\textbf{What is the number of triangulations of \(A\)?} From the
combinatorial construction, we directly get: \[
T(z) = 1 + T(z) \times z \times T(z) \Leftrightarrow
\]

\[
\Leftrightarrow T(z) = 1 + zT(z)^2 \Leftrightarrow
\] we solve the quadratic equation for \(a=z, b = -1, c = 1\): \[
T(z)= \frac{1- \sqrt{1-4z}}{2z}
\] where we chose the negative solution, since for \(n=0\) the number is
0.

This is the closed form of the Catalan numbers. Therefore, the number of
triangulations of \(A\) is computed by: \[
[z^n]\sum_{n \geq 0} \frac{1}{n+1} \binom{2n}{n}
\] which is \[
 \frac{1}{n+1} \binom{2n}{n}
\]

\hypertarget{use-exponential-generating-functions-to-determine-the-number-a_n-of-ordered-choices-of-n-balls-such-that-there-are-2-or-4-red-balls-an-even-number-of-green-balls-and-an-arbitrary-number-of-blue-balls.}{%
\paragraph{\texorpdfstring{56) Use exponential generating functions to
determine the number \(a_n\) of ordered choices of \(n\) balls such that
there are \(2\) or \(4\) red balls, an even number of green balls and an
arbitrary number of blue
balls.}{56) Use exponential generating functions to determine the number a\_n of ordered choices of n balls such that there are 2 or 4 red balls, an even number of green balls and an arbitrary number of blue balls.}}\label{use-exponential-generating-functions-to-determine-the-number-a_n-of-ordered-choices-of-n-balls-such-that-there-are-2-or-4-red-balls-an-even-number-of-green-balls-and-an-arbitrary-number-of-blue-balls.}}

\begin{itemize}
\tightlist
\item
  \(2\) or \(4\) red balls: \((\frac{z^2}{2!} + \frac{z^4}{4!})\)
\item
  even number of green balls: \(\frac{1}{2}(e^z + e^{-z})\)
\item
  arbitrary number of blue balls: \(e^z\)
\end{itemize}

We therefore get: \[
(\frac{z^2}{2!} + \frac{z^4}{4!}) \times \frac{1}{2}(e^z + e^{-z}) \times e^z = 
\]

\[
= (\frac{z^2}{2!} + \frac{z^4}{4!}) \times (\frac{e^{2z}}{2} + \frac{e^{0}}{2})= 
\]

\[
= \frac{1}{2} \frac{z^2}{2} + \frac{1}{2} \frac{z^4}{4!} + \frac{e^{2z}}{2} \frac{z^2}{2} + \frac{e^{2z}} {2}\frac{z^4}{4!} = 
\]

\[
\frac{1}{2} \frac{z^2}{2} + \frac{1}{2} \frac{z^4}{4!} + \frac{1}{4} z ^2 \sum_{n \geq 0} 2^n \frac{z^{n}}{n!} + 
\frac{1}{48} z ^4 \sum_{n \geq 0} 2^n \frac{z^{n}}{n!} = 
\]

\[
\frac{1}{2} \frac{z^2}{2} + \frac{1}{2} \frac{z^4}{4!} + \sum_{n \geq 0} \frac{1}{4}  2^n \frac{z^{n}}{n!} + 
\sum_{n \geq 0} \frac{1}{48}  2^{n} \frac{z^{n}}{n!} = 
\]

\[
\frac{1}{2} \frac{z^2}{2} + \frac{1}{2} \frac{z^4}{4!} + \sum_{n \geq 2} \frac{1}{4}  2^{n-2} \frac{z^{n}}{(n-2)!} + 
\sum_{n \geq 4} \frac{1}{48} 2^{n-4} \frac{z^{n}}{(n-4)!} = 
\]

\[
\frac{1}{2} \frac{z^2}{2} + \frac{1}{2} \frac{z^4}{4!} + \sum_{n \geq 2} \frac{1}{16} n (n-1) 2^n \frac{z^n}{(n-2)!} + 
\sum_{n \geq 4} \frac{1}{768} n (n-1)(n-2) (n-3) 2^n \frac{z^n}{n!} = 
\]

\[
\frac{z^2}{2} + 3 \frac{z^3}{3!} + 13 \frac{z^4}{4!} + \sum_{n \geq 5} \frac{1}{16} n (n-1) 2^n \frac{z^n}{n!} + 
\sum_{n \geq 5} \frac{1}{768} n (n-1) (n-2) (n-3) 2^n \frac{z^n}{n!} = 
\]

\[
\frac{z^2}{2} + 3 \frac{z^3}{3!} + 13 \frac{z^4}{4!} + \sum_{n \geq 5} (\frac{1}{16} + \frac{(n-2)(n-3)}{768}) n (n-1) 2^n \frac{z^n}{n!} = 
\]

\[
\frac{z^2}{2} + 3 \frac{z^3}{3!} + 13 \frac{z^4}{4!} + \sum_{n \geq 5} \frac{1}{768} (48 + n^2 - 5n + 6) n (n-1) 2^n \frac{z^n}{n!} = 
\]

\[
\frac{z^2}{2} + 3 \frac{z^3}{3!} + 13 \frac{z^4}{4!} + \sum_{n \geq 5} \frac{1}{768} (n^2 - 5n + 54) n (n-1) 2^n \frac{z^n}{n!} = 
\] We therefore get the following numbers for \(a_n\): \[
a_n = \begin{cases}
1  &\textit{, if } n = 2 \\
3  &\textit{, if } n = 3 \\
13  &\textit{, if } n = 4 \\
\frac{1}{768} (n^2 - 5n + 54) n (n-1) 2^n  &\textit{, if } n \geq 5 
\end{cases}
\]

\hypertarget{determine-all-solutions-of-the-recurrence-relation}{%
\paragraph{57) Determine all solutions of the recurrence
relation:}\label{determine-all-solutions-of-the-recurrence-relation}}

\[
a_n - 2 n a_{n-1} + n (n-1) a_{n-2} = 2n \times n! \textit{, } n \geq 2 \textit{, } a_0 = a_1 = 1.
\] \emph{Hint: Use exponential generating functions.}

\[
\sum_{ n \geq 2} a_n \frac{z^n}{n!}  - 2 \sum_{ n \geq 2} n a_{n-1} \frac{z^n}{n!} + \sum_{ n \geq 2} n (n-1) a_{n-2} \frac{z^n}{n!} = 2 \sum_{ n \geq 2} n  n! \frac{z^n}{n!} 
\]

\[
(\sum_{ n \geq 0} a_n \frac{z^n}{n!}) - a_0 - za_1  - 2z (\sum_{ n \geq 0} n a_{n} \frac{z^n}{n!}) - a_0 + z^2 (\sum_{ n \geq 0} a_{n} \frac{z^n}{n!}) = 2 ((\sum_{ n \geq 2} n z^n ) - z)
\]

we input the respective generating functions: \[
\hat{A} - a_0 - za_1  - 2z \hat{A} - a_0 + z^2 \hat{A} = 2 \left( \frac{z}{(1-z)^2}  - z \right)
\] \[
\hat{A} - 1 - z  - 2z \hat{A}  + 2z + z^2 \hat{A} = \frac{2z}{(1-z)^2}  - 2z
\] \[
\hat{A} (1-2z+z^2) + 3z - 1 = \frac{2z}{(1-z)^2}
\]

\[
\hat{A} (1-z)^2 + 3z - 1 = \frac{2z}{(1-z)^2}
\]

\[
\hat{A} (1-z)^4 + 3z(1-z)^2 - (1-z)^2 = 2z
\]

\[
\hat{A} (1-z)^4 + 3z - 6z^2 + 3z^3 - 1 + 2z -z^2 = 2z
\]

\[
\hat{A} (1-z)^4 = -3z^3 + 7z^2 -3z + 1 =
\]

\[
\hat{A} = \frac{(1-z)^3}{(1-z)^4} + \frac{4z^2 - 2z^3}{(1-z)^4} =
\]

\[
\hat{A} = \sum_{n \geq 0} z^n + 4z^2 \sum_{n \geq 0} \binom{-4}{n} z^n- 2z^3 \sum_{n \geq 0} \binom{-4}{n} z^n =
\]

\[
\hat{A} = \sum_{n \geq 0} z^n - 2 \sum_{n \geq 3} \binom{-4}{n-3} z^n + 4 \sum_{n \geq 2} \binom{-4}{n-2} z^n =
\]

\[
\hat{A} = 1 + z + z^2 \sum_{n \geq 3} z^n - \sum_{n \geq 3} 2\binom{-4}{n-3} z^n + 4z^2 \sum_{n \geq 3} \binom{-4}{n-2} z^n =
\]

\[
\hat{A} = 1 + z + 5z^2 +  \sum_{n \geq 3} (1 -  (2\binom{-4}{n-3}) + (4 \binom{-4}{n-2})) z^n
\]

Therefore:

\[
a_n = \begin{cases}
1  &\textit{, if } n = 0 \\
1  &\textit{, if } n = 1 \\
5  &\textit{, if } n = 2 \\
1 -  (2\binom{-4}{n-3}) + (4 \binom{-4}{n-2})  &\textit{, if } n \geq 3 
\end{cases}
\]

\hypertarget{an-involution-is-a-permutation-pi-such-that-pi-circ-pi-id_m-where-m-12-dots-n.-let-mathcali-be-the-set-of-involutions.-determine-the-exponential-generating-function-iz-of-mathcali.}{%
\paragraph{\texorpdfstring{58) An involution is a permutation \(\pi\)
such that \(\pi \circ \pi = id_M\), where \(M = \{1,2, \dots, n\}\). Let
\(\mathcal{I}\) be the set of involutions. Determine the exponential
generating function \(I(z)\) of
\(\mathcal{I}\).}{58) An involution is a permutation \textbackslash pi such that \textbackslash pi \textbackslash circ \textbackslash pi = id\_M, where M = \textbackslash\{1,2, \textbackslash dots, n\textbackslash\}. Let \textbackslash mathcal\{I\} be the set of involutions. Determine the exponential generating function I(z) of \textbackslash mathcal\{I\}.}}\label{an-involution-is-a-permutation-pi-such-that-pi-circ-pi-id_m-where-m-12-dots-n.-let-mathcali-be-the-set-of-involutions.-determine-the-exponential-generating-function-iz-of-mathcali.}}

The possible structures of permutations which satisfy the condition
\(\pi \circ \pi = id_M\), are either 1-cycles, where \(\pi(i) = i\) or
2-cycles where \(\pi(i) = j\) for some \(j \neq i\) and \(\pi(j) = i\).

Therefore, \(\mathbb{I}\) can be defined as a combinatorial structure:
\[
\mathcal{I} = set(1cycle(\{\circ\}) + 2cycle(\{\circ\}))
\] Hence, \(I\) can be computed by: \[
exp\left(log(\frac{1}{1-z}) + \frac{1}{2} log(\frac{1}{1-z})^2\right) = 
\]

\[
exp\left(log(\frac{1}{1-z}) + log(((\frac{1}{1-z})^2)^{\frac{1}{2}})\right) = e^z + e^{z^2 \frac{1}{2}} = e^{z + \frac{z^2}{2}}
\]

\hypertarget{let-mathcalt-be-the-class-of-rooted-and-labelled-trees-i.e.-the-n-vertices-of-a-tree-of-size-n-are-labelled-with-the-labels-12-dots-n.-use-the-theory-of-combinatorial-constructions-to-determine-a-functional-equation-for-the-exponential-generating-function-of-mathcalt.}{%
\paragraph{\texorpdfstring{59) Let \(\mathcal{T}\) be the class of
rooted and labelled trees, i.e.~the \(n\) vertices of a tree of size
\(n\) are labelled with the labels \(1,2, \dots, n\). Use the theory of
combinatorial constructions to determine a functional equation for the
exponential generating function of
\(\mathcal{T}\).}{59) Let \textbackslash mathcal\{T\} be the class of rooted and labelled trees, i.e.~the n vertices of a tree of size n are labelled with the labels 1,2, \textbackslash dots, n. Use the theory of combinatorial constructions to determine a functional equation for the exponential generating function of \textbackslash mathcal\{T\}.}}\label{let-mathcalt-be-the-class-of-rooted-and-labelled-trees-i.e.-the-n-vertices-of-a-tree-of-size-n-are-labelled-with-the-labels-12-dots-n.-use-the-theory-of-combinatorial-constructions-to-determine-a-functional-equation-for-the-exponential-generating-function-of-mathcalt.}}

\textbf{Finally, apply the following theorem to prove that the number of
trees in \(\mathcal{T}\) which have \(n\) vertices is equal to
\$n\^{}\{n-1\}. (You are not asked to prove the theorem.)}
\emph{Theorem: Let \(\Phi(w) = \sum_{n \geq 0} \phi_0 \neq 0\) with
\(\phi_0 \neq 0\). If \(z = w/\Phi(w)\), then
\([z^n]w = \frac{1}{n} [w^{n-1}] \Phi(w)^n\)}

\hypertarget{show-the-following-formula-for-stirling-numbers-of-the-second-kind}{%
\paragraph{60) Show the following formula for Stirling numbers of the
second
kind:}\label{show-the-following-formula-for-stirling-numbers-of-the-second-kind}}

\[
\sum_{n \geq 0} \sum_{k = 0}^{n} S_{n,k} \frac{z^n}{n!} u^k = e^{u(e^x -1)}
\]

We start with the left hand side: \[
\sum_{n,k} S_{n,k} \frac{z^n}{n!} u^k
\]

\hypertarget{exercise-sheet-7}{%
\subsection{Exercise Sheet 7}\label{exercise-sheet-7}}

\hypertarget{let-p-be-the-set-of-all-divisors-of-12.-determine-the-muxf6bius-function-of-p-using-the-definition-of-the-muxf6bius-function-and-compare-your-result-with-the-one-from-the-last-example-in-the-lecture.}{%
\paragraph{\texorpdfstring{61) Let \(P\) be the set of all divisors of
\(12\). Determine the Möbius function of \((P,|)\) using the definition
of the Möbius function and compare your result with the one from the
last example in the
lecture.}{61) Let P be the set of all divisors of 12. Determine the Möbius function of (P,\textbar) using the definition of the Möbius function and compare your result with the one from the last example in the lecture.}}\label{let-p-be-the-set-of-all-divisors-of-12.-determine-the-muxf6bius-function-of-p-using-the-definition-of-the-muxf6bius-function-and-compare-your-result-with-the-one-from-the-last-example-in-the-lecture.}}

\begin{Shaded}
\begin{Highlighting}[]
\NormalTok{graph Hasse \{ }
\NormalTok{    label="Hasse diagram for P=\{1,2,3,4,6,12\}"}
\NormalTok{    12[label="12",shape=circle]}
\NormalTok{    4[label="4",shape=circle]}
\NormalTok{    6[label="6",shape=circle]}
\NormalTok{    2[label="2",shape=circle]}
\NormalTok{    3[label="3",shape=circle]}
\NormalTok{    1[label="1",shape=circle]}
    
\NormalTok{    12 {-}{-} 6;}
\NormalTok{    12 {-}{-} 4;}
\NormalTok{    4 {-}{-} 2;}
\NormalTok{    6 {-}{-} 2;}
\NormalTok{    6 {-}{-} 3;}
\NormalTok{    2 {-}{-} 1;}
\NormalTok{    3 {-}{-} 1;}
    
\NormalTok{\} }
\end{Highlighting}
\end{Shaded}

The Möbius function is defined as follows: \[
\forall x,y \in P: \quad \sum_{z \in [x,y]} \varphi(z,y) = 
    \begin{cases}
    &1 \textit{, if } x=y \\
    &0 \textit{, if } x \neq y
    \end{cases}
\]

By definition \(\varphi(12,12) = 1\) and \[
\varphi(6,12) + \varphi(12,12) = 0
\] meaning \(\varphi(6,12) = -1\). Thus, by \[
\varphi(3,12) + \varphi(6,12) + \varphi(12,12) = 0
\] we derive \(\varphi(3,12) = 0\). And by \[
\varphi(4,12) +  \varphi(12,12) = 0
\] we get that \(\varphi(4,12) = -1\). Now we have: \[
\varphi(2,12) +  \varphi(4,12) + \varphi(6,12) +  \varphi(12,12) ) = 0
\] from which we can derive \(\varphi(2,12) = 1\) Therefore, we know by:
\[
 \varphi(1,12) +  \varphi(2,12) + \varphi(3,12) +  \varphi(4,12) +  \varphi(6,12) +  \varphi(12,12) = 0
\] that \(\varphi(1,12) = 0\). The last example from the lecture states:
\[
\varphi(1,n) = \varphi(n)= \varphi_{\geq}(0,e_1) \cdots \varphi_{geq}(0,e_r)
\] where \(n = p_1^{e_1} \cdots p_r^{e_r}\) (prime factorisation of
\(n\)). \[
\varphi(n)=
    \begin{cases} 
        1 &\text{, if } n=1\\
        (-1)^r &\text{, if } n=p_1 \cdots p_r\\
        0 &\text{, if $n$ is not square free}.
    \end{cases}
\] Since \(1 \cdot 2 \cdot 3 \cdot 4 \cdot 6 \cdot 12\) is not square
free \(\varphi(12)=0\), which coincides with our findings.

\hypertarget{let-p-be-the-poset-defined-by-p-01234-and-0-geq-1-geq-4-0-geq-2-geq-4-0-geq-3-geq-4.-compute-all-values-varphixy-for-xy-in-p.}{%
\paragraph{\texorpdfstring{62) Let \((P,|)\) be the poset defined by \$P
=\{0,1,2,3,4\} and \(0 \geq 1 \geq 4\), \(0 \geq 2 \geq 4\),
\(0 \geq 3 \geq 4\). Compute all values \(\varphi(x,y)\) for
\(x,y \in P\).}{62) Let (P,\textbar) be the poset defined by \$P =\{0,1,2,3,4\} and 0 \textbackslash geq 1 \textbackslash geq 4, 0 \textbackslash geq 2 \textbackslash geq 4, 0 \textbackslash geq 3 \textbackslash geq 4. Compute all values \textbackslash varphi(x,y) for x,y \textbackslash in P.}}\label{let-p-be-the-poset-defined-by-p-01234-and-0-geq-1-geq-4-0-geq-2-geq-4-0-geq-3-geq-4.-compute-all-values-varphixy-for-xy-in-p.}}

\begin{Shaded}
\begin{Highlighting}[]
\NormalTok{graph Hasse \{ }
\NormalTok{    label="Hasse diagram for P=\{0,1,2,3,4\}"}
\NormalTok{    0[label="0",shape=circle]}
\NormalTok{    4[label="4",shape=circle]}
\NormalTok{    2[label="2",shape=circle]}
\NormalTok{    3[label="3",shape=circle]}
\NormalTok{    1[label="1",shape=circle]}
    
\NormalTok{    4 {-}{-} 1;}
\NormalTok{    4 {-}{-} 2;}
\NormalTok{    4 {-}{-} 3;}
\NormalTok{    1 {-}{-} 0;    }
\NormalTok{    2 {-}{-} 0;}
\NormalTok{    3 {-}{-} 0;}
    
\NormalTok{\} }
\end{Highlighting}
\end{Shaded}

We know: \[
\varphi(4,4) = \varphi(3,3) = \varphi(2,2) = \varphi(1,1) = \varphi(0,0) = 1
\] and \[
\varphi(0,1) = \varphi(0,3) = \varphi(0,2) = \varphi(1,4) = \varphi(2,4) = \varphi(3,4) = -1
\] and by the calculated values and \[
\varphi(0,4) + \varphi(1,4) + \varphi(2,4) + \varphi(3,4) + \varphi(4,4) = 0
\] we derive \(\varphi(0,4) = 2\).

\hypertarget{let-p_1-leq_1-and-p_2-leq_2-be-two-locally-finite-posets-with-0-element-and-pleq-be-defined-by-p-p_1-times-p_2-and-for-axby-in-p}{%
\paragraph{\texorpdfstring{63) Let \((P_1, \leq_1)\) and
\((P_2, \leq_2)\) be two locally finite posets with \(0\)-element and
\((P,\leq)\) be defined by \(P = P_1 \times P_2\) and for
\((a,x),(b,y) \in P\):}{63) Let (P\_1, \textbackslash leq\_1) and (P\_2, \textbackslash leq\_2) be two locally finite posets with 0-element and (P,\textbackslash leq) be defined by P = P\_1 \textbackslash times P\_2 and for (a,x),(b,y) \textbackslash in P:}}\label{let-p_1-leq_1-and-p_2-leq_2-be-two-locally-finite-posets-with-0-element-and-pleq-be-defined-by-p-p_1-times-p_2-and-for-axby-in-p}}

\[
(a,x) \leq (b,x) \Leftrightarrow a \leq_1 b \land x \leq_2 y.
\] Show that \((P,\leq)\) is a locally finite poset with \(0\)-element.

A poset has to satisfy \emph{reflexivity}, \emph{asymmetry} and
\emph{transitivity}. We first show that \(P\) is a poset.

\begin{itemize}
\tightlist
\item
  Reflexivity (\(\forall (a,x) \in P: (a,x) \leq (a,x)\)) By \(P_1,P_2\)
  being posets and thus reflexive, we know for each \((a,x) \in P\) that
  \(a \leq_1 a\) and \(x \leq_2 x\), thus by definition of \(P\),
  \((a,x) \leq (a,x)\).
\item
  Antisymmetry
  (\(\forall (a,x),(b,y): (a,x) \leq (b,y) \land (b,y) \leq (a,x) \implies (a,x) = (b,y)\))
  We take arbitrary \((a,x), (b,y)\) such that \((a,x) \leq (b,y)\) and
  \((b,y) \leq (a,x)\). Then \(a \leq_1 b\) and \(b \leq_1 a\) by
  definition of \(P\), thus \(a = b\) by \(P_1\) being a poset. And
  \(x \leq_2 y\) and \(y \leq_2 x\) by definition of \(P\), thus
  \(x = y\) by \(P_2\) being a poset. Therefore, \((a,x) = (b,y)\) by
  \(a = b\) and \(x = y\).
\item
  Transitivity
  (\(\forall (a,x), (b,y), (c,z): (a,x) \leq (b,y) \land (b,y) \leq (c,z) \implies (a,x) \leq (c,z)\))
  We take arbitrary \((a,x), (b,y), (c,z)\) such that
  \((a,x) \leq (b,y)\) and \((b,y) \leq (c,z)\). Then \(a \leq_1 b\) and
  \(b \leq_1 c\) by definition of \(P\), thus \(a \leq_1 c\) by \(P_1\)
  being a poset. And \(x \leq_2 y\) and \(y \leq_2 z\) by definition of
  \(P\), thus \(x \leq_2 z\) by \(P_2\) being a poset. It follows, that
  \((a,x) \leq (c,z)\) by definition of \(P\).
\item
  \((P,\leq)\) has a \(0\)-element By \(P_1, P_2\) having
  \(0\)-elements, there exist \(a \in P_1\) and \(x \in P_2\) such that
  \(\forall b \in P_1: a \leq_1 b\) and
  \(\forall y \in P_2: x \leq_2 y\), thus we know \((a,x)\) is the
  \(0\)-element of \(P\), since
  \(\forall (b,y) \in P: (a,x) \leq (b,y)\) by the above and definition
  of \(P\).
\item
  \((P,\leq)\) is locally finite Since \(P_1,P_2\) are locally finite,
  we know: \(\forall a,b \in P_1: |[a,b]| < \infty\) and
  \(\forall x,y \in P_2: |[x,y]| < \infty\) Therefore, for any
  \(a,b \in P_1\) and \(x,y \in P_2\), \(|[a,b]| = k\) for some finite
  \(k\) and \(|[x,y]| = d\) for some finite \(d\). Inspecting the
  intervall \([(a,x),(b,y)]\) of \(P\), by the above
  \(|[(a,x),(b,y)]| = k \cdot d\), where \(k,d\) are respectively
  finite, thus \(k \cdot d\) is finite.
\end{itemize}

\hypertarget{we-use-the-notations-from-exercise-63.-let-the-muxf6bius-functions-of-pp_1p_2-be-denoted-by-varphi_p-varphi_p_1-varphi_p_2-respectively.-prove-that-forall-ax-by-in-p-we-have-varphi_paxby-varphi_p_1-cdot-varphi_p_2xy.}{%
\paragraph{\texorpdfstring{64) We use the notations from exercise 63.
Let the Möbius functions of \(P,P_1,P_2\) be denoted by
\(\varphi_P, \varphi_{P_1}, \varphi_{P_2}\), respectively. Prove that
forall \((a,x) (b,y) \in P\) we have
\(\varphi_P((a,x),(b,y)) = \varphi_{P_1} \cdot \varphi_{P_2}(x,y)\).}{64) We use the notations from exercise 63. Let the Möbius functions of P,P\_1,P\_2 be denoted by \textbackslash varphi\_P, \textbackslash varphi\_\{P\_1\}, \textbackslash varphi\_\{P\_2\}, respectively. Prove that forall (a,x) (b,y) \textbackslash in P we have \textbackslash varphi\_P((a,x),(b,y)) = \textbackslash varphi\_\{P\_1\} \textbackslash cdot \textbackslash varphi\_\{P\_2\}(x,y).}}\label{we-use-the-notations-from-exercise-63.-let-the-muxf6bius-functions-of-pp_1p_2-be-denoted-by-varphi_p-varphi_p_1-varphi_p_2-respectively.-prove-that-forall-ax-by-in-p-we-have-varphi_paxby-varphi_p_1-cdot-varphi_p_2xy.}}

For \(\varphi_P((a,x),(b,y))\) it has to hold that: \[
\sum_{(c,z) \in [(a,x),(b,y)]}
\varphi_{P((c,z),(b,y))} = \begin{cases}
1 &\text{ ,if $a=b$ and $x=y$}\\
0 &\text{ ,if $a \neq b$ or $x \neq y$}
\end{cases}
\]

The first condition holds, since:

\(\varphi_{P_1}(a,b) = 1\), when \(a=b\) and \(varphi_{P_2}(x,y) = 1\)
when \(x = y\) thus \(\varphi_P((a,x),(b,y)) = 1 \cdot 1 = 1\), if
\(a=b\) and \(x = y\).

We show, the second condition holds: \[
\sum_{(c,z) \in [(a,x),(b,y)]} \varphi_P((c,z),(b,y)) = \sum_{(c,z) \in [(a,x),(b,y)]} \varphi_{P_1}(c,b) \cdot \varphi_{P_2}(z,y) =
\] we split the sum \[
\sum_{(c) \in [a,b]} \sum_{(z) \in [x,y]} \varphi_{P_1}(c,b) \cdot \varphi_{P_2}(z,y)
\] we extract the constant factor \(\varphi_{P_1}(c,b)\) \[
\sum_{(c) \in [a,b]} \varphi_{P_1}(c,b) \cdot \sum_{(z) \in [x,y]} \varphi_{P_2}(z,y)
\] Now either \(x \neq y\) or \(a \neq b\), w.l.o.g. we say \(x = y\) \[
\sum_{(c) \in [a,b]} \varphi_{P_1}(c,b) \cdot 0 = 0
\]

\hypertarget{draw-the-hasse-diagram-of-2123-supseteq-and-redo-the-proof-of-the-principle-of-inclusion-and-exclusion-for-the-special-case-of-three-sets-a_1-a_2-a_3-subseteq-m.-carry-out-every-step-in-detail.}{%
\paragraph{\texorpdfstring{65) Draw the Hasse diagram of
\((2^{1,2,3}, \supseteq)\) and redo the proof of the principle of
inclusion and exclusion for the special case of three sets
\(A_1, A_2, A_3 \subseteq M\). Carry out every step in
detail.}{65) Draw the Hasse diagram of (2\^{}\{1,2,3\}, \textbackslash supseteq) and redo the proof of the principle of inclusion and exclusion for the special case of three sets A\_1, A\_2, A\_3 \textbackslash subseteq M. Carry out every step in detail.}}\label{draw-the-hasse-diagram-of-2123-supseteq-and-redo-the-proof-of-the-principle-of-inclusion-and-exclusion-for-the-special-case-of-three-sets-a_1-a_2-a_3-subseteq-m.-carry-out-every-step-in-detail.}}

\begin{Shaded}
\begin{Highlighting}[]
\NormalTok{graph Hasse \{ }
\NormalTok{    label="Hasse diagram for (2\^{}\{1,2,3\})"}
\NormalTok{    123[label="\{1,2,3\}",shape=circle]}
\NormalTok{    12[label="\{1,2\}",shape=circle]}
\NormalTok{    23[label="\{2,3\}",shape=circle]}
\NormalTok{    13[label="\{1,3\}",shape=circle]}
\NormalTok{    1[label="\{1\}",shape=circle]}
\NormalTok{    2[label="\{2\}",shape=circle]}
\NormalTok{    3[label="\{3\}",shape=circle]}
\NormalTok{    empty[label="\{\}",shape=circle]}
    
\NormalTok{    123 {-}{-} 12;}
\NormalTok{    123 {-}{-} 23;}
\NormalTok{    123 {-}{-} 13;}
\NormalTok{    12 {-}{-} 2;}
\NormalTok{    23 {-}{-} 3;}
\NormalTok{    23 {-}{-} 2;}
\NormalTok{    12 {-}{-} 1;}
\NormalTok{    13 {-}{-} 1;}
\NormalTok{    13 {-}{-} 3;}
\NormalTok{    1 {-}{-} empty;    }
\NormalTok{    2 {-}{-} empty;}
\NormalTok{    3 {-}{-} empty;}
    
\NormalTok{\} }
\end{Highlighting}
\end{Shaded}

\hypertarget{let-pqr-be-three-distinct-prime-numbers-and-m-pqr.-how-many-of-the-numbers-12-dots-m-are-relatively-prime-to-m-two-numbers-x-and-y-are-called-relatively-prime-if-their-greatest-common-divisor-is-1.}{%
\paragraph{\texorpdfstring{66) Let \(p,q,r\) be three distinct prime
numbers and \(m = pqr\). How many of the numbers \(1,2, \dots, m\) are
relatively prime to \(m\)? (Two numbers \(x\) and \(y\) are called
relatively prime if their greatest common divisor is
\(1\).)}{66) Let p,q,r be three distinct prime numbers and m = pqr. How many of the numbers 1,2, \textbackslash dots, m are relatively prime to m? (Two numbers x and y are called relatively prime if their greatest common divisor is 1.)}}\label{let-pqr-be-three-distinct-prime-numbers-and-m-pqr.-how-many-of-the-numbers-12-dots-m-are-relatively-prime-to-m-two-numbers-x-and-y-are-called-relatively-prime-if-their-greatest-common-divisor-is-1.}}

\[
m - ((\frac{m}{p} + \frac{m}{q} + \frac{m}{r}) - (\frac{m}{pq} + \frac{m}{pr} + \frac{m}{qr}) + \frac{m}{pqr}) = 
\] since, \(m = pqr\) \[
= m - (\frac{pqr}{p} + \frac{pqr}{q} + \frac{pqr}{r} - \frac{pqr}{pq} - \frac{pqr}{pr} - \frac{pqr}{qr} + \frac{pqr}{pqr}) = 
\] we cancel the respective factors in each term \[
= m - qr - pr - pq + r + q + p - 1
\]

\hypertarget{prove-if-gcdab-1-then-gcdaba-b-is-either-1-or-2.}{%
\paragraph{\texorpdfstring{67) Prove: If \(gcd(a,b) = 1\) then
\(gcd(a+b,a-b)\) is either \(1\) or
\(2\).}{67) Prove: If gcd(a,b) = 1 then gcd(a+b,a-b) is either 1 or 2.}}\label{prove-if-gcdab-1-then-gcdaba-b-is-either-1-or-2.}}

Let \(d = gcd(a+b,a-b)\), then by \(\dagger\), \(d | (a+b)+(a-b)\), thus
\(d|2a\), and \(d | (a+b)-(a-b)\), thus \(d | 2b\).

And since \(d|2a\) and \(d |2b\), \(d|gcd(2a,2b)\), by definition of the
\(gcd\), and by \(gcd(a,b) = 1\), \(d\) can only either be \(1\) or
\(2\).

\emph{Lemma \(\dagger\):} Let \(z = gcd(x,y)\), then \(\exists m,n\),
such that \(x=zm\) and \(y=zn\). Then \(x+y = zm + zn = z(m+n)\), thus
\(z|(x+y)\) for arbitrary \(x,y\), where \(z = gcd(x,y)\) and
\(x-y = zm - zn = z(m-n)\), thus \(z|(x-y)\) for arbitrary \(x,y\) such
that \(z=gcd(x,y)\).

\hypertarget{let-n-prod_p-in-mathbbp-pv_p-n-be-a-positive-integer-such-that-for-all-p-in-mathbbp-we-have-v_pn-leq-1.-moreover-a-prime-p-divides-n-if-and-only-if-p-1-divides-n-too-dagger.-compute-n.}{%
\paragraph{\texorpdfstring{68) Let
\(n = \prod_{p \in \mathbb{P}} p^{v_p (n)}\) be a positive integer such
that for all \(p \in \mathbb{P}\) we have \(v_p(n) \leq 1\). Moreover, a
prime \(p\) divides \(n\) if and only if \(p-1\) divides \(n\) too
(\(\dagger\)). Compute
\(n\).}{68) Let n = \textbackslash prod\_\{p \textbackslash in \textbackslash mathbb\{P\}\} p\^{}\{v\_p (n)\} be a positive integer such that for all p \textbackslash in \textbackslash mathbb\{P\} we have v\_p(n) \textbackslash leq 1. Moreover, a prime p divides n if and only if p-1 divides n too (\textbackslash dagger). Compute n.}}\label{let-n-prod_p-in-mathbbp-pv_p-n-be-a-positive-integer-such-that-for-all-p-in-mathbbp-we-have-v_pn-leq-1.-moreover-a-prime-p-divides-n-if-and-only-if-p-1-divides-n-too-dagger.-compute-n.}}

We know that forall positive integers, \(1 | n\), thus by \(\dagger\)
\(2 | n\), therefore also \(3|n\). Since \(2\) and \(3\) divide \(n\),
\(6 | n\), hence by \(\dagger\), \(7 |n\). Then,
\(n = 2 \cdot 3 \cdot 7 = 42\).

\hypertarget{use-the-euclidian-algorithm-to-find-two-integers-such-that}{%
\paragraph{69) Use the Euclidian algorithm to find two integers such
that:}\label{use-the-euclidian-algorithm-to-find-two-integers-such-that}}

\[
2863a + 1057b = 42
\]

We first compute the division chain: \[
\begin{aligned}
2863 &= 1057 \cdot 2 + 749 \\
1057 &= 749 \cdot 1 + 308 \\
749 &=308 \cdot 2 + 133 \\
308 &= 133 \cdot 2 + 42 \\
133 &= 42 \cdot 3 + 7 \\
42 &= 7 \cdot 6 + 0
\end{aligned}
\]

We then find the linear combination for \(2863\) and \(1057\): \[
\begin{aligned}
42 &= 308 - (133 \cdot 2) \\
42 &= 308 - ((749 - 308 \cdot 2) \cdot 2) = 308 \cdot 5 - 749 \cdot 2\\
42 &= (1057 - 749 \cdot 1) \cdot 5 - 749 \cdot 2 = 1057 \cdot 5 - 749 \cdot 7
42 &= 1057 \cdot 5 - (2863 - 1057 \cdot 2) \cdot 7 = 1057 \cdot 19 - 2863 \cdot 7
\end{aligned}
\]

Thus we have found \(a = -7\) and \(b = 19\) which satisfy the equation.

\hypertarget{use-the-euclidian-algorithm-to-find-all-the-greatest-common-divisors-of-x3-5x2-7x-3-and-x3-x2---5x-3-in-mathbbqx.}{%
\paragraph{\texorpdfstring{70) Use the Euclidian algorithm to find all
the greatest common divisors of \(x^3 + 5x^2 + 7x + 3\) and
\(x^3 + x^2 - 5x +3\) in
\(\mathbb{Q}[x]\).}{70) Use the Euclidian algorithm to find all the greatest common divisors of x\^{}3 + 5x\^{}2 + 7x + 3 and x\^{}3 + x\^{}2 - 5x +3 in \textbackslash mathbb\{Q\}{[}x{]}.}}\label{use-the-euclidian-algorithm-to-find-all-the-greatest-common-divisors-of-x3-5x2-7x-3-and-x3-x2---5x-3-in-mathbbqx.}}

\[
\begin{aligned}
x^3 + 5x^2 + 7x + 3 = 1 \cdot (x^3 + x^2 - 5x + 3) + (4x^2 + 12x)
x^3 + x^2 - 5x + 3 = (4x^2 + 12x) \cdot (1/4 x - 1/2) + (x+3)
4x^2 + 12x = (x+3) \cdot (4x) + 0
\end{aligned}
\]

Thus, forall \(x \in \mathbb{Q}^x: a \cdot x + 3 \cdot a\) is a gcd of
\(x^3 + 5x^2 + 7x + 3\) and \(x^3 + x^2 - 5x +3\).

\hypertarget{exercise-sheet-8}{%
\subsection{Exercise Sheet 8}\label{exercise-sheet-8}}

\hypertarget{let-f_n-geq-0-be-the-fibonacci-sequence-i.e.-f_0-0-f_1-1-f_n1-f_n-f_n-1.-prove-gcdf_n2-f_n-1.}{%
\paragraph{\texorpdfstring{71) Let \((F)_{n \geq 0}\) be the Fibonacci
sequence, i.e.~\(F_0 = 0, F_1 = 1, F_{n+1} = F_n + F_{n-1}\). Prove
\(gcd(F_{n+2}, F_n) = 1\).}{71) Let (F)\_\{n \textbackslash geq 0\} be the Fibonacci sequence, i.e.~F\_0 = 0, F\_1 = 1, F\_\{n+1\} = F\_n + F\_\{n-1\}. Prove gcd(F\_\{n+2\}, F\_n) = 1.}}\label{let-f_n-geq-0-be-the-fibonacci-sequence-i.e.-f_0-0-f_1-1-f_n1-f_n-f_n-1.-prove-gcdf_n2-f_n-1.}}

We proceed by induction:

\begin{itemize}
\item
  Base case: \(F_2 = F_1 + F_0 = 0 + 1 = 1\) and \(gcd(F_2, F_0) = 1\)
  since \(gcd(1,0) = 1\).
\item
  IH: \(gcd(F_{n+2}, F_n) = 1\) holds for \(n \in \mathbb{N}\).
\item
  Induction step: We are now interested in \(gcd(F_{n+3}, F_{n+1})\)
  which is equivalent to \(gcd(F_{n+2} + F_{n+1}, F_{n+1})\) and since
  \(gcd(a+bm, b) = gcd(a,b)\),
  \(gcd(F_{n+2} + F_{n+1}, F_{n+1}) = gcd(F_{n+2}, F_{n+1})\). This
  again is equivalent to \(gcd(F_{n+1} + F_{n}, F_{n+1})\) and by the
  same argument as before
  \(gcd(F_{n+1} + F_{n}, F_{n+1}) = gcd(F_n, F_{n+1})\) which is
  equivalent to \(gcd(F_n, F_{n+1} + F_n)\) by the same argument again,
  which is \(gcd(F_n, F_{n+2})\) by definition of the Fibonacci
  sequence, thus \(gcd(F_{n+3},F_{n+1}) = gcd(F_n, F_{n+2})\) which is 1
  by IH.
\end{itemize}

\hypertarget{prove-that-there-exist-infinitely-many-primes-which-are-solutions-of-the-equation-p-equiv-3-mod-4.}{%
\paragraph{\texorpdfstring{72) Prove that there exist infinitely many
primes which are solutions of the equation \(p \equiv 3\) mod
\(4\).}{72) Prove that there exist infinitely many primes which are solutions of the equation p \textbackslash equiv 3 mod 4.}}\label{prove-that-there-exist-infinitely-many-primes-which-are-solutions-of-the-equation-p-equiv-3-mod-4.}}

Proof by contradiction: Assume there are only finitely many such primes:
\[
\mathcal{p}_1, \dots \mathcal{p}_n \text{ for some } n \in \mathbb{N}. 
\] We consider the number: \[
x = 4\mathcal{p_1}\cdot \mathcal{p}_2 \cdots \mathcal{p_n} -1
\] We distinguish as follows

\begin{itemize}
\item
  case 1: x is prime by definition of
  \(\mathcal{p}_1, \dots \mathcal{p}_n\),
  \(4 \mathcal{p_1}\cdot \mathcal{p}_2 \cdots \mathcal{p_n} \equiv 0\)
  mod \(4\), therefore
  \(4 \mathcal{p_1}\cdot \mathcal{p}_2 \cdots \mathcal{p_n} -1 \equiv 3\)
  mod \(4\). Now, since \(x\) is prime, there are \(n+1\) primes which
  satisfy the equation. Contradiction!
\item
  case 2: x is not prime Then, there exist \(q_1, \dots , q_m\), such
  that \(q_i \in \mathbb{P}\), where \(1 \leq i \leq m\) and
  \(q_1 \cdot q_2 \cdots q_m = x\). \(x\) has to be odd, since
  \(x \equiv 3\) mode \(4\), thus \(q_i \neq 2\) holds for all
  \(1 \leq i \leq m\). There has to be some \(q_i\) such that
  \(q_i \equiv 3\) mod \(4\), else \(\forall j: q_j \equiv 1\) mod 4
  would have to hold, thus \(x \equiv 1\) mod \(4\) would have to hold.

  Now, \(q_i \neq p_j\) holds forall \(1 \leq j \leq n\), else
  \(\exists j\) such that \(q_i = p_j\), thus \(q_i |x\) (by definition
  of \(q_i\)) and \(q_i | x+1\) since \(pj | x+1\).

  Therefore, there exist \(n+1\) primes which satisfy the equation
  namely \(\mathcal{p}_1, \dots \mathcal{p}_n\) and \(q_i\).
\end{itemize}

\hypertarget{find-without-using-a-computer-the-last-two-digits-of-21000.}{%
\paragraph{\texorpdfstring{73) Find (without using a computer) the last
two digits of
\(2^{1000}\).}{73) Find (without using a computer) the last two digits of 2\^{}\{1000\}.}}\label{find-without-using-a-computer-the-last-two-digits-of-21000.}}

We can find the last two digits by: \[
2^{1000} \text{ mod } 100
\]

We cannot apply Euler's theorem since \(gcd(2,100) \neq 1\). But
\(gcd(25,2) = 1\) and \(\varphi(25) = 20\). So: \[
2^{20} \equiv 1 \text{ mod } 25
\] therefore \[
(2^{20})^{50} \equiv 1^{50} \text{ mod } 25
\] Additionally we know \(2^2 + 2^{998}\) and \(2^2 = 0\) mod \(4\). We
can therefore define the following system of congruence equations: \[
2^{1000} \equiv 0 (4) \\
2^{1000} \equiv 1 (25)
\] We can then apply the Chinese Remainder Theorem to find
\(2^{1000}(100)\), i.e.~\(2^{1000}(4 \cdot 25)\).

We remind ourselves of the Chinese Remainder Theorem: \[
x \equiv \sum_{j=1}^{r} \frac{m}{m_j} \cdot bj a_j (m) \text{, with } b_j = (\frac{m}{m_j})^{-1} (m_j)
\]

We apply \(x = 2^{1000}, a_1 = 0, a_2 = 1, m = 100, m_1 = 4, m_2 = 25\):
\[
2^{1000} \equiv \frac{100}{25} \cdot 19 \cdot 1 (100) + \frac{100}{4} \cdot 1 \cdot 0 (100)
\] \[
2^{1000} \equiv 4 \cdot 19 (100) 
\] \[
2^{1000} \equiv 76 (100) 
\] Hence the last two digits are \(76\).

\hypertarget{consider-the-ring-mathbbzi-a-b_i-ab-in-mathbbz-addition-and-multiplication-taken-from-mathbbc-and-determine-a-gcd19-5i-16-6i.}{%
\paragraph{\texorpdfstring{74) Consider the ring
\(\mathbb{Z}[i] = \{a + b_i | a,b \in \mathbb{Z} \}\) (addition and
multiplication taken from \(\mathbb{C}\)) and determine a
\(gcd(19 + 5i, 16-6i)\).}{74) Consider the ring \textbackslash mathbb\{Z\}{[}i{]} = \textbackslash\{a + b\_i \textbar{} a,b \textbackslash in \textbackslash mathbb\{Z\} \textbackslash\} (addition and multiplication taken from \textbackslash mathbb\{C\}) and determine a gcd(19 + 5i, 16-6i).}}\label{consider-the-ring-mathbbzi-a-b_i-ab-in-mathbbz-addition-and-multiplication-taken-from-mathbbc-and-determine-a-gcd19-5i-16-6i.}}

\emph{Hint: you may assume without proof, that \(\mathbb{Z}[i]\) with
\(n(a+bi) = a^2 + b^2\) is a Euclidian ring. Now find \(q,r\) in
\(u = qv + r\) by determining \(\frac{u}{v}\) in \(\mathbb{Z}[i]\) and
rounding real and imaginary part.}

We apply the Euclidian algorithm, since \(\mathbb{Z}[i]\) is a Euclidian
ring.

\[
\begin{aligned}
19 + 5i &= (1+i) (16 -6i) + (-3-5i)\\
16 - 6i &= (-1 + 3i) \cdot (-3i -5i) + (-2-2i)\\
-3i - 5i &= (2+i) \cdot (-2-2i) + (-1+i)\\
-2-2i &= (0 + 2i) \cdot (-1 + i) + 0
\end{aligned}
\]

Thus \(-1+i\) is a gcd of \(19+5i\) and \(16-6i\).

\hypertarget{which-of-the-following-mappings-is-well-defined}{%
\paragraph{75) Which of the following mappings is well
defined?}\label{which-of-the-following-mappings-is-well-defined}}

\begin{enumerate}
\def\labelenumi{\alph{enumi})}
\item
  \(f : \mathbb{Z}_m \rightarrow \mathbb{Z}_m, \bar x \mapsto \bar{x^2}\)
  We show \(f\) is well defined by showing that for any \(x,y\) with
  \(\bar x \equiv \bar y\), \(\bar{x^2} \equiv \bar{y^2}\) holds. We
  thus assume \(\bar x \equiv \bar y\), i.e.~\(\exists k (x = km +y)\).
  We compute an \(l\) such that \(x^2 = lm + y^2\). Since we know
  \(x = km +y\) for some \(k\), we can write: \[
   (km + y)^2 = lm +y^2\\
   (km)^2 + 2kmy + y^2 = lm +y \\
   (km)^2 + 2kmy = lm \\
   k^2m+2ky = l
   \] Since we computed an \(l\) such that \(x^2 = lm+y\), we have shown
  \(\bar{x^2} \equiv \bar{y^2}\) for any \(x,y\), such that
  \(\bar x \equiv \bar y\).
\item
  \(g: \mathbb{Z}_m \rightarrow \mathbb{Z}_m\),
  \(\bar x \mapsto \bar{2^x}\) For all \(m > 1\), \(\bar0 = \bar m\) but
  \(\bar{2^0} = \bar 1\) and \(\bar{2^m} \neq \bar 1\), thus
  \(\bar{2^0} = \bar 1\) and \(\bar{2^m} \neq \bar 1\), thus
  \(\bar{2^0} \neq \bar{2^m}\), but \(\bar0 = \bar m\).
\end{enumerate}

\hypertarget{use-the-chinese-remainder-theorem-to-solve-the-following-system-of-congruence-relations}{%
\paragraph{76) Use the Chinese remainder theorem to solve the following
system of congruence
relations:}\label{use-the-chinese-remainder-theorem-to-solve-the-following-system-of-congruence-relations}}

\[
3x \equiv 12(13) \mathit{\dagger}, \quad 5x \equiv 7(22)\clubsuit, \quad 2x \equiv 3 (7) \heartsuit.
\]

\begin{itemize}
\tightlist
\item
  \(\dagger\): we can divide \(3\) and \(12\) by \(3\), \(13\) remains
  as \(ggT(13,3) = 1\).
\item
  \(\clubsuit\): we calculate times \(9\) since \(5^{-1} = 9\) in
  \(\mathbb{Z}_{22}\)
\item
  \(\heartsuit\): times \(4\) since \(2^{-1} = 4\) in \(\mathbb{Z}_7\)
  Thus we get: \[
  x \equiv 4(13) \mathit{\dagger}, \quad x \equiv 19(22)\clubsuit, \quad x \equiv 5 (7) \heartsuit.
  \] Now since, \(ggT(13,22) = ggT(13,7) = ggT(22,7) = 1\), we can apply
  the Chinese Remainder Theorem: \[
  \begin{aligned}
  x \equiv &\frac{2002}{13} \cdot b_1 \cdot 4 \ (2002) \ + \\
  &\frac{2002}{22} \cdot b_2 \cdot 19 \ (2002) \ + \\
  &\frac{2002}{7} \cdot b_3 \cdot 5 \ (2002)
  \end{aligned}
  \] where \[
  \begin{aligned}
  b_1 &= (\frac{2002}{13})^{-1} (13) = 154^-1 (13)  = 6\\
  b_2 &= (\frac{2002}{22})^{-1} (22) = 91^{-1} (22) = 15\\
  b_3 &= (\frac{2002}{7})^{-1} (7) = 286^{-1} (7) = 6
  \end{aligned}
  \] thus \[
  \begin{aligned}
  x &\equiv 154 \cdot 6 \cdot 4 + 91 \cdot 15 \cdot \cdot 19 + 286 \cdot 6 \cdot 5 \ (2002) \\
  x &\equiv 173 \ (2002)
  \end{aligned}
  \]
\end{itemize}

\hypertarget{use-the-chinese-remainder-theorem-to-solve-the-following-system-of-congruence-relations-1}{%
\paragraph{76) Use the Chinese Remainder Theorem to solve the following
system of congruence
relations:}\label{use-the-chinese-remainder-theorem-to-solve-the-following-system-of-congruence-relations-1}}

\[
5x \equiv 8(32) \mathit{\dagger}, 
\quad 14x \equiv 2(22)\clubsuit, 
\quad 9x \equiv 3 (15) \heartsuit.
\]

\begin{itemize}
\tightlist
\item
  \(\dagger\): we calculate times \(13\) since \(5^{-1} = 13\) in
  \(\mathbb{Z}_{32}\)
\item
  \(\clubsuit\): we divide by \(2\), then calculate times \(8\) since
  \(7^{-1} = 7\) in \(\mathbb{Z}_{11}\)
\item
  \(\heartsuit\): we divide by \(3\), then calculate times \(2\) since
  \(3^{-1} = 2\) in \(\mathbb{Z}_5\)
\end{itemize}

\[
x \equiv 8(32) \mathit{\dagger}, 
\quad x \equiv 8(11)\clubsuit, 
\quad x \equiv 2 (5) \heartsuit.
\]

Since \(gcd(32,11) = gcd(11,5) = gcd(32,5) = 1\), we apply the CRT with
\(m = 32 \cdot 11 \cdot 5 = 1760\).

\[
\begin{aligned}
x \equiv &\frac{1760}{32} \cdot b_1 \cdot 8 \ (1760) \ + \\
&\frac{1760}{11} \cdot b_2 \cdot 8 \ (1760) \ + \\
&\frac{1760}{5} \cdot b_3 \cdot 2 \ (1760)
\end{aligned}
\] where \[
\begin{aligned}
b_1 &= (\frac{1760}{32})^{-1} (32) = 55^-1 (32)  = 7\\
b_2 &= (\frac{1760}{11})^{-1} (11) = 160^{-1} (11) = 2\\
b_3 &= (\frac{1760}{5})^{-1} (5) = 352^{-1} (5) = 3
\end{aligned}
\] thus \[
\begin{aligned}
x \equiv &55 \cdot 7 \cdot 8 \ (1760) \ + \\
&160 \cdot 2 \cdot 8 \ (1760) \ + \\
&352 \cdot 3 \cdot 2 \ (1760)
\end{aligned}
\] \[
\begin{aligned}
x \equiv 3080 + 2560 + 2112 \equiv 712 \ (1760)
\end{aligned}
\]

\hypertarget{let-ne-323349-be-a-public-rsa-key.-compute-the-decryption-key-d.}{%
\paragraph{\texorpdfstring{78) Let \((n,e) = (3233,49)\) be a public RSA
key. Compute the decryption key
\(d\).}{78) Let (n,e) = (3233,49) be a public RSA key. Compute the decryption key d.}}\label{let-ne-323349-be-a-public-rsa-key.-compute-the-decryption-key-d.}}

We know \(d = 49^{-1} (v)\), where
\(v = lcm(p-1, q-1) = \frac{(p-1)(q-1)}{gcd(p-1,q-1)}\). And we know
\(n = q\cdot p= 3233\), where \(p,q \in \mathbb{P}, q \neq 2, p \neq 2\)
and \(p \neq q\).

We simply try primes and divide \(3233\) by them, checking whether the
resulting number is prime.

We get: \(3233 = 53 \cdot 61\).

We then calculate \(gcd(52,60)\) by the Euclidian algorithm: \[
\begin{aligned}
60 &= 52 \cdot 1 + 8\\
52 &= 8 \cdot 6 + 4\\
8 = 4 \cdot 2 + 0
\end{aligned}
\]

Hence, \(\frac{52 \cdot 60}{4} = 780\). Therefore,
\(d = 49^{-1} (780)\), meaning \[
d \cdot 49 \equiv 1 (780)
\] we conclude \(d = 589\).

\hypertarget{use-the-key-of-exercise-78-to-encrypt-the-string-mathttcomputer.-decompose-the-string-into-blocks-of-length-2-and-apply-the-mapping-mathtta-mapsto-01-mathttb-mapsto-02-dots-mathttz-mapsto-26.}{%
\paragraph{\texorpdfstring{79) Use the key of exercise 78) to encrypt
the string \(\mathtt{"COMPUTER"}\). Decompose the string into blocks of
length \(2\) and apply the mapping
\(\mathtt{A} \mapsto 01, \mathtt{B} \mapsto 02, \dots \mathtt{Z} \mapsto 26\).}{79) Use the key of exercise 78) to encrypt the string \textbackslash mathtt\{"COMPUTER"\}. Decompose the string into blocks of length 2 and apply the mapping \textbackslash mathtt\{A\} \textbackslash mapsto 01, \textbackslash mathtt\{B\} \textbackslash mapsto 02, \textbackslash dots \textbackslash mathtt\{Z\} \textbackslash mapsto 26.}}\label{use-the-key-of-exercise-78-to-encrypt-the-string-mathttcomputer.-decompose-the-string-into-blocks-of-length-2-and-apply-the-mapping-mathtta-mapsto-01-mathttb-mapsto-02-dots-mathttz-mapsto-26.}}

\[
\mathtt{CO} \mapsto 0315, \quad 0315^{49} \ (3233) = 2701 \\
\mathtt{MP} \mapsto 1316, \quad 1316^{49} \ (3233) = 2593 \\
\mathtt{UT} \mapsto 2120, \quad 2120^{49} \ (3233) = 0371 \\
\mathtt{ER} \mapsto 0518, \quad 0518^{49} \ (3233) = 1002 \\
\] Thus the encrypted string is: \(2701 2593 0371 1002\).

\hypertarget{let-en-and-dn-be-bobs-public-and-private-rsa-key-respectively.-suppose-that-bob-receives-an-encrypted-message-c-which-is-also-caught-by-an-eavesdropper.-the-eavesdropper-wants-to-find-out-the-original-message-m-and-has-the-idea-to-send-bob-a-message-and-ask-him-to-sign-it.-how-can-this-idea-be-used-to-find-out-m}{%
\paragraph{\texorpdfstring{80) Let \((e,n)\) and \((d,n)\) be Bob's
public and private RSA key, respectively. Suppose that Bob receives an
encrypted message \(c\) which is also caught by an eavesdropper. The
eavesdropper wants to find out the original message \(m\) and has the
idea to send Bob a message and ask him to sign it. How can this idea be
used to find out
\(m\)?}{80) Let (e,n) and (d,n) be Bob's public and private RSA key, respectively. Suppose that Bob receives an encrypted message c which is also caught by an eavesdropper. The eavesdropper wants to find out the original message m and has the idea to send Bob a message and ask him to sign it. How can this idea be used to find out m?}}\label{let-en-and-dn-be-bobs-public-and-private-rsa-key-respectively.-suppose-that-bob-receives-an-encrypted-message-c-which-is-also-caught-by-an-eavesdropper.-the-eavesdropper-wants-to-find-out-the-original-message-m-and-has-the-idea-to-send-bob-a-message-and-ask-him-to-sign-it.-how-can-this-idea-be-used-to-find-out-m}}

\emph{Hint: Pick a random integer \(r\) and consider the message
\(r^{e}c\) mod \(n\).}

We choose an arbitrary integer \(r\), such that \(gcd(r,n) = 1\)
consider the message \(x = r^{e} \cdot c \ (n)\).

We know for any \(a_j\), \(a_j \equiv a_j^{e \cdot d} \ (n)\),
\(E(a_j) = (a_j \text{ mod } n)\), and
\(D(b_j) = (b_j^d \text{ mod } n)\).

The signed message from bob is then of the form: \[
x^d \equiv (r^{e} \cdot c)^d \equiv (r^{e})^d \equiv c^d \equiv r \cdot c^d \ (n)
\] we simply need to calculate \(x \cdot r^{-1}\) to get \(c^d\), which
in turn we can use to calculate \(m(n)\). Note, that \(x \cdot r^{-1}\)
exists since we chose \(r\) such that \(gcd(r,n) = 1\).

\hypertarget{exercise-sheet-9}{%
\subsection{Exercise Sheet 9}\label{exercise-sheet-9}}

\hypertarget{let-g-be-a-finite-abelian-group-and-a-in-g-an-element-for-which-ord_ga-is-maximal.-prove-that-for-all-b-in-g-the-order-ord_gb-is-a-divisor-of-ord_ga.}{%
\paragraph{\texorpdfstring{81) Let \(G\) be a finite, abelian group and
\(a \in G\) an element for which \(ord_G(a)\) is maximal. Prove that for
all \(b \in G\) the order \(ord_G(b)\) is a divisor of
\(ord_G(a)\).}{81) Let G be a finite, abelian group and a \textbackslash in G an element for which ord\_G(a) is maximal. Prove that for all b \textbackslash in G the order ord\_G(b) is a divisor of ord\_G(a).}}\label{let-g-be-a-finite-abelian-group-and-a-in-g-an-element-for-which-ord_ga-is-maximal.-prove-that-for-all-b-in-g-the-order-ord_gb-is-a-divisor-of-ord_ga.}}

Proof by contradiction:

Assume \(a \in G\) for which \(ord_G(a)\) is maximal and \(b \in G\)
where \(ord_G(b) \nmid ord_G(a)\).

We denote \(k = ord_G(a)\) and \(l = ord_G(b)\), and can write \[
k = \prod_{p \in \mathbb{P}} p^{\varphi_p(k)} \text{ ,and } l = \prod_{p \in \mathbb{P}} p^{\varphi_p(l)}
\] By \(l \nmid k\), there exist \(p_1, \dots, p_n \in \mathbb{P}\) such
that \(\varphi_{p_i}(l) > \varphi_{p_i}(k)\).

Then, by definition:
\(lcm(k,l) = \prod_{p \in \mathbb{P}} p^{max(\varphi_p(k), \varphi_p (l))}\)
which means \(lcm(k,l)>k\).

This means \(\exists c \in G\) such that \(ord(c) = lcm(k,l)\) by

\[
\forall a,b \in G (ord(a) = r \land ord(b) = s) \Rightarrow \exists c \in G (ord(c) = lcm(r,s))
\]

We know \(lcm(k,l) > k\), i.e.: \[
lcm(ord_G(a), ord_G(b)) > ord_G(a)
\]

thus \(ord_G(c) > ord_G(a)\) but \(ord_G(a)\) is maximal. Contradiction!

\hypertarget{let-lambda-and-varphi-denote-the-carmicheal-function-and-eulers-totient-function-respectively.-compute-lambda172872-and-varphi172872.}{%
\paragraph{\texorpdfstring{82) Let \(\lambda\) and \(\varphi\) denote
the Carmicheal function and Euler's totient function respectively.
Compute \(\lambda(172872)\) and
\(\varphi(172872)\).}{82) Let \textbackslash lambda and \textbackslash varphi denote the Carmicheal function and Euler's totient function respectively. Compute \textbackslash lambda(172872) and \textbackslash varphi(172872).}}\label{let-lambda-and-varphi-denote-the-carmicheal-function-and-eulers-totient-function-respectively.-compute-lambda172872-and-varphi172872.}}

We recall the definitions of the Carmicheal and totient functions: \[
\varphi(n) = lcm(\lambda(p_1^{r_1}) \cdot \lambda(p_2^{r_2}) \cdots \varphi(p_k^{r_k})),\\
\varphi(p^r) = 
    \begin{cases} \varphi(p^r) &\text{, if } p^r = 2,3^r,4,5^r,7^r,11^r,13^r, \dots \\
                            \frac{1}{2} \varphi(p^r) &\text{, if } p^r = 8,16,32,64, \dots
    \end{cases}\\
\varphi(n) = n \cdot \prod_{p | n} (1-\frac{1}{p})
\] We apply integer factorisation to \(172872\) and get \[
172872 = 2^3 \cdot 3^2 \cdot 7^4
\] thus \[
\lambda(172872) = \lambda(2^3 \cdot 3^2 \cdot 7^4) = lcm(\lambda(2^3), \lambda(3^2), \lambda(7^4))
\] we calculate the respective Carmicheal values: \[
\begin{aligned}
\lambda(2^3) &= \frac{1}{2} \varphi(2^3) = \frac{1}{2} 4 = 2\\
\lambda(3^2) &= \varphi(3^2) = 6\\
\lambda(7^4) &= 7^3 \cdot 6 = 2058
\end{aligned}
\] \[
lcm(2,6,2058) = 2058
\]

\[
\varphi(172872) = 172872 \cdot (1- \frac{1}{2}) \cdot (1- \frac{1}{3}) \cdot (1- \frac{1}{7}) = \\
= 172872 \cdot \frac{1}{2} \cdot \frac{2}{3} \cdot \frac{6}{7} = 
172872 \cdot \frac{2}{7} = 49392
\]

\hypertarget{consider-an-rsa-cryptosystem-with-n-pq-p-and-q-being-odd-primes-with-p-neq-q.-let-n_pq-denote-the-number-of-possible-pairs-ed-such-that-en-and-dn-are-the-public-and-private-key-respectively.-show-that-n_pq-varphilambdan-where-lambda-is-the-carmicheal-function-and-varphi-eulers-totient-function-and-that-2-leq-n_pq-leq-lambdan2.}{%
\paragraph{\texorpdfstring{83) Consider an RSA cryptosystem with
\(n = pq\), \(p\) and \(q\) being odd primes with \(p \neq q\). Let
\(N_{p,q}\) denote the number of possible pairs \((e,d)\) such that
\((e,n)\) and \((d,n)\) are the public and private key respectively.
Show that \(N_{p,q} = \varphi(\lambda(n))\), where \(\lambda\) is the
Carmicheal function and \(\varphi\) Euler's totient function, and that
\(2 \leq N_{p,q} \leq \lambda(n)/2\).}{83) Consider an RSA cryptosystem with n = pq, p and q being odd primes with p \textbackslash neq q. Let N\_\{p,q\} denote the number of possible pairs (e,d) such that (e,n) and (d,n) are the public and private key respectively. Show that N\_\{p,q\} = \textbackslash varphi(\textbackslash lambda(n)), where \textbackslash lambda is the Carmicheal function and \textbackslash varphi Euler's totient function, and that 2 \textbackslash leq N\_\{p,q\} \textbackslash leq \textbackslash lambda(n)/2.}}\label{consider-an-rsa-cryptosystem-with-n-pq-p-and-q-being-odd-primes-with-p-neq-q.-let-n_pq-denote-the-number-of-possible-pairs-ed-such-that-en-and-dn-are-the-public-and-private-key-respectively.-show-that-n_pq-varphilambdan-where-lambda-is-the-carmicheal-function-and-varphi-eulers-totient-function-and-that-2-leq-n_pq-leq-lambdan2.}}

By definition we need to choose \(e\), coprime to \(lcm(p-1,q-1)\).
(Since \(d\) is the inverse of \(e\) in \(\mathbb{Z}_{\lambda(pq)}\),
each \(e\) has a unique \(d\), thus we only count possible \(e\)s.) Now
\(\lambda (pq) = lcm(\lambda(p), \lambda(q))\) since \(p,q\) are primes.
And by \(p,q\) prime, \(\lambda(p) = \varphi(p) = p(1- \frac{1}{p})\) =
p-1, the same holds for \(q\). Hence, \(\lambda(pq) = lcm(p-1,q-1)\) and
\(\varphi(lcm(p-1,q-1))\) counts all integers coprime to
\(lcm(p-1, q-1)\). Thus \(N_{p,q} = \varphi(\lambda(n))\).

Show that \(2 \leq N_{p,q} < \frac{\lambda(pq)}{2}\).
\emph{Counterexample:
\(p=3, q=5 \Rightarrow \varphi(\lambda (3 \cdot 5)) < \frac{\lambda(3 \cdot 5)}{2} \Rightarrow \varphi(4) < \frac{4}{2} \Rightarrow 2 < 2\)}

We instead show: \(2 \leq N{p,q} \leq \frac{\lambda(pq)}{2}\), i.e. \[
2 \leq \varphi(\lambda(pq)) \leq \frac{lcm(p-1,q-1)}{2}
\]

\(\varphi(\lambda(pq)) \leq \frac{\lambda(pq)}{2}\) since
\(\lambda(pq) = lcm(p-1, q-1)\) and \(p,q\) are odd primes,
\(lcm(p-1, q-1)\) is even and has at least prime factor \(2\), thus by
3) \(\varphi(\lambda(pq)) \leq \lambda(pq) \cdot \frac{1}{2}\).

\(2 \leq \varphi(\lambda(pq))\) Assuming \(\varphi(\lambda(pq)) = 1\)
then by definition of \(\varphi\), \(\lambda(p,q) = 2\). Hence, by
\(\lambda(pq) = lcm(p-1, q-1)\), \(p-1, q-1 = 2\), i.e.~\(p = q = 3\).
Contradiction, we conclude
\(2 \leq \varphi(\lambda(pq)) \leq \frac{\lambda(pq)}{2}\).

\hypertarget{show-that-mn-implies-lambdam-lambda-n-where-lambda-denotes-the-carmicheal-function.}{%
\paragraph{\texorpdfstring{84) Show that \(m|n\) implies
\(\lambda(m) | \lambda (n)\) where \(\lambda\) denotes the Carmicheal
function.}{84) Show that m\textbar n implies \textbackslash lambda(m) \textbar{} \textbackslash lambda (n) where \textbackslash lambda denotes the Carmicheal function.}}\label{show-that-mn-implies-lambdam-lambda-n-where-lambda-denotes-the-carmicheal-function.}}

\emph{Hint: Prove first that \(a_i | b_i\) for \(i = 1, \dots k\)
implies \(lcm(a_1,a_2,\dots , a_k) | lcm(b_1,b_2, \dots, b_k)\).}

We assume \(a_i | b_i\), with \(i = 1, \dots, k\). Thus
\(\forall i: b_i | lcm(b_1, \dots, b_k)\), and thus by \(a_i|b_i\),
\(a_i|lcm(b_1, \dots, b_k)\) follows.

Let \[
\dagger := a|b \Leftrightarrow \forall p : v_p(a) \leq v_p(b) \\
\heartsuit := lcm(a,b) = \prod_{p \in \mathbb{P}} \\
\text{,where: } v_p(a) \text{ is the multiplicity of } p \text{ in the factorisation of } n. 
\]

Since, forall \(i\), \(a_i | b_i\), by \(\dagger\) every prime factor of
\(a_i\)'s exponent is smaller or equal than \(b_i\)'s. By
\(\heartsuit\), \(lcm(a_1, \dots a_k)\) contains the largest of all
prime exponents which as shown before are still smaller of equal than
the maximum of all primes' exponents in \(lcm(b_1, \dots, b_k)\).

Hence, \(lcm(a_1, \dots, a_k) | lcm(b_1, \dots, b_k\). Let
\(n = \prod_{i = 1}^{r} p_i^{n_i}\) and \(m = \prod_{i=1}^r p_i^{n_i}\)
and \(m | n\). Then, \(p_i^{m_i} | p_i^{n_i}\) forall
\(i = 1, \dots, r\), now by the hint we showed to hold and by definition
of the Carmicheal function:
\(\lambda(p_i^{n_i}) = p_i^{n_i-1} (p_i -1)\) all exponents in the prime
factorisation of \(m\) as well as \(n\) are reduced , hence forall
\(p_i^{m_i}, p_i^{n_i}\) where \(p_i^{m_i} | p_i{n_i}\),
\(\lambda(p_i^{n_i}) | \lambda(p_i^{m_i})\) also holds.

Thus,
\(lcm(\lambda(p_1^{m1}), \dots, \lambda(p_r^{mr})) | lcm(\lambda(p_1^{n1}) , \dots , \lambda(p_r^{nr}))\),
holds by the hint which we proved.

\hypertarget{let-mathbbzi-a-bi-ab-in-mathbbz-where-i-sqrt-1.-show-that-mathbbzi-is-a-subring-of-mathbbccdot.}{%
\paragraph{\texorpdfstring{85) Let
\(\mathbb{Z}[i] = \{a + bi | a,b \in \mathbb{Z}\}\) where
\(i = \sqrt(-1)\). Show that \(\mathbb{Z}[i]\) is a subring of
\((\mathbb{C},+,\cdot)\).}{85) Let \textbackslash mathbb\{Z\}{[}i{]} = \textbackslash\{a + bi \textbar{} a,b \textbackslash in \textbackslash mathbb\{Z\}\textbackslash\} where i = \textbackslash sqrt(-1). Show that \textbackslash mathbb\{Z\}{[}i{]} is a subring of (\textbackslash mathbb\{C\},+,\textbackslash cdot).}}\label{let-mathbbzi-a-bi-ab-in-mathbbz-where-i-sqrt-1.-show-that-mathbbzi-is-a-subring-of-mathbbccdot.}}

\(\mathbb{Z}[i]\) is a subring of \((\mathbb{C},+,\cdot)\) if it is a
nonempty subset of \(\mathbb{C}\) and

\(\forall z_1, z_2 \in \mathbb{Z}[i] z_1 - z_2 \in \mathbb{Z}[i]\), and
\(z_1 \cdot z_2 \in \mathbb{Z}[i]\), where
\(z_1 = a_1 + b_1i, z_2 = a_2 + b_2i\).

\begin{itemize}
\tightlist
\item
  \(\mathbb{Z}[i]\) is trivially nonempty
\item
  Assume \(z_1,z_2 \in \mathbb{Z}[i]\), then

  \begin{itemize}
  \tightlist
  \item
    \(z_1 - z_2\) = \$(a\_1 + b\_1i) - (a\_2 + b\_2i) =
    \(a_1-a_2 + (b_1 - b_2 i)\) thus the resulting coefficients are both
    in \(\mathbb{Z}[i]\).
  \item
    \(z_1 \cdot z_2\) =
    \((a_1 + b_1i) \cdot (a_2 +b_2i) = a_1a_2 - b1b2 + (a_1b_2 + a_2b1)i\),
    where, again the resulting coefficients are both in
    \(\mathbb{Z}[i]\).
  \end{itemize}
\end{itemize}

\textbf{Determine its group of units \((\mathbb{Z}[i]^*, \cdot)\)} Let
\(a+bi\) be a unit in \(\mathbb{Z}[i]\), then trivially \(a\) and \(b\)
cannot both be zero. Additionally \(\exists c,d \in \mathbb{Z}\) such
that \((a+bi)(c+di) = 1+0i\), meaning \[
ac + adi + cbi -bd = 1 + 0i
\] thus \(ac-bd = 1\) and \(ad +cb = 0\). Hence: \[
c = \frac{ad}{b} \Rightarrow a(-\frac{ad}{b}) - bd = 1 \Rightarrow a^2d+b^2d = -b \Rightarrow  d = \frac{-b}{a^2+b^2}
\] now \[
\frac{a(-\frac{b}{a^2+b^2})}{b} = \frac{\frac{ab}{a^2+b^2}}{\frac{b}{1}} = \frac{ab}{a^2b + b^3} = \frac{a}{a^2+b^2} = c
\]

Now let \(n = a^2+b^2\), we know \(n >0\) and \(n \in \mathbb{Z}\),
since \(c,d \in \mathbb{Z}\), \(n |a\) and \(n |b\), there are
\(x,y \in \mathbb{Z}\) such that \(a = nx\) and \(b = ny\): \[
n = (nx)^2 + (ny)^2 \Leftrightarrow n = n2(x^2+y^2) \Leftrightarrow 1 = n(x^2+y^2)
\] This means, either

\begin{itemize}
\tightlist
\item
  \(n = 1\) , or
\item
  (x\^{}2 + y\^{}2) = 1\$, w.l.o.g. we assume \(x = 0, y = 1\), then
  \(a = 0\) by \(nx = a\) and \(b = n\) by \(ny = b\) but
  \(n = a^2 +b^2 = 0 +n^2\), thus \(n = n^2\), meaning \(n\) can only be
  \(1\).
\end{itemize}

Hence, in any case \(a^2+b^2 = 1\), meaning either \(a^2 = 1, b^2 = 0\)
or \(a^2 = 0, b^2 = 1\). We conclude, the units of \(\mathbb{Z}[i]\) are
\(\{1, -1, i, -\}\).

\textbf{Is \(\mathbb{Z}[i]\) an integral domain?} Yes, as it is a
subring of \((\mathbb{C},+,\cdot)\) and \((\mathbb{C},+,\cdot)\) is an
integral domain.

\hypertarget{let-rcdot-be-an-integral-domain.-two-elements-a-and-b-are-associated-if-there-is-a-unit-rin-r-such-that-a-br.-show-that-two-elements-xy-in-r-are-associated-iff-x-y-and-y-x.}{%
\paragraph{\texorpdfstring{86) Let \((R,+,\cdot)\) be an integral
domain. Two elements \(a\) and \(b\) are associated if there is a unit
\(r\in R^*\) such that \(a = br\). Show that two elements \(x,y \in R\)
are associated iff \(x | y\) and
\(y | x\).}{86) Let (R,+,\textbackslash cdot) be an integral domain. Two elements a and b are associated if there is a unit r\textbackslash in R\^{}* such that a = br. Show that two elements x,y \textbackslash in R are associated iff x \textbar{} y and y \textbar{} x.}}\label{let-rcdot-be-an-integral-domain.-two-elements-a-and-b-are-associated-if-there-is-a-unit-rin-r-such-that-a-br.-show-that-two-elements-xy-in-r-are-associated-iff-x-y-and-y-x.}}

\begin{itemize}
\item
  \(\Rightarrow\): We assume arbitrary \(x,y \in R\) that are
  associated: Then \(x | y\) trvially, since \(rx = y\) by definition.
  Additionally since \(r \in R^*\), we take the multiplicative inverse
  \(r^{-1}\) to show \(y|x\), as \(x = y \cdot r^{-1}\).
\item
  \(\Leftarrow\): We assume arbitrary \(x,y \in R\) such that \(x|y\)
  and \(y|x\): Then, \(\exists c_1: x \cdot c_1 = y\) and
  \(\exists c_2: y \cdot c_2 = x\) thus \(y = c_1 \cdot c_2\). Since
  \(R\) is an integral domain, we can cancel \(y\) and get
  \(1 = c1 \cdot c_2\) therefore \(c_1\) and \(c_2\) are units as they
  are each other's multiplicative inverses.
\end{itemize}

\hypertarget{let-rcdot-be-an-integral-domain.-show-that-x-in-r-is-a-unit-iff-it-is-a-divisor-of-every-a-in-r.}{%
\paragraph{\texorpdfstring{87) Let \((R,+,\cdot)\) be an integral
domain. Show that \(x \in R\) is a unit iff it is a divisor of every
\(a \in R\).}{87) Let (R,+,\textbackslash cdot) be an integral domain. Show that x \textbackslash in R is a unit iff it is a divisor of every a \textbackslash in R.}}\label{let-rcdot-be-an-integral-domain.-show-that-x-in-r-is-a-unit-iff-it-is-a-divisor-of-every-a-in-r.}}

\begin{itemize}
\tightlist
\item
  \(\Rightarrow\): We assume an arbitrary \(x \in R^*\). Then
  \$\exists \(x^{-1}: x \cdot x^{-1} = 1\). We know \(x |y\) in \(R\)
  iff \(\exists t \in R : tx = y\). Now, forall
  \(a \in R: a = a \cdot x \cdot x^{-1}\). Therefore we can satisfy
  \(\exists t \in R : tx = a\), namely \(t = a \cdot x^{-1}\), meaning
  \(x|a\) holds for all \(a \in R\).
\item
  \(\Leftarrow\): We assume an arbitrary \(x \in R\) such that
  \(\forall a \in R: x |a\) This holds specifically for \(a = 1\),
  meaning \(x | 1\), this means \(\exists t: t \cdot x = 1\) but by
  definition \(t = x^{-1}\). Thus \(x\) is a unit.
\end{itemize}

\hypertarget{let-rcdot-be-a-euclidian-ring-and-let-its-euclidian-function-be-denoted-by-n.-show-that-nx-n1-for-all-units-x-of-r.}{%
\paragraph{\texorpdfstring{88) Let \((R,+,\cdot)\) be a Euclidian ring
and let its Euclidian function be denoted by \(n\). Show that
\(n(x) = n(1)\) for all units \(x\) of
\(R\).}{88) Let (R,+,\textbackslash cdot) be a Euclidian ring and let its Euclidian function be denoted by n. Show that n(x) = n(1) for all units x of R.}}\label{let-rcdot-be-a-euclidian-ring-and-let-its-euclidian-function-be-denoted-by-n.-show-that-nx-n1-for-all-units-x-of-r.}}

We recall the properties of \(n\) in \(R\):

\[
\forall a,b \in R: b \neq 0: \exists q,r \in R:\\
\heartsuit: a = bq+r \text{ with } n(r) < n(b) \text{ or } r = 0\\
\dagger: \quad \forall a,b \in R\backslash \{0\}: n(a) \leq n(ab) 
\]

Let \(x\) be a unit, then \(1 = x \cdot x^{-1}\). As \(1\) divides any
element in \(R\), \(n(1) \leq n(x) = n(1x)\), by \(\dagger\).

\(x|1\) since \(1 = x \cdot x^{-1}\), thus
\(n(x) \leq n(x \cdot x ^{-1}) = n(1)\) by \(\dagger\) thus
\(n(x) = n(1)\).

\textbf{Prove, moreover, if \(x,y \in R\) and \(y\) is a unit, then
\(n(xy) = n(x)\).} \[
n(x) \leq n(xy)
\] holds trivially by \(\dagger\).

By \(y\) unit \(y \cdot y^{-1} \cdot x = x\). By \(\dagger\): \[
n(y \cdot x) \leq n(y\cdot x \cdot x^{-1}) = n(x)
\] Thus \(n(xy) = n(x)\).

\hypertarget{list-all-irreducible-polynomials-up-to-degree-4-over-mathbbz_3.}{%
\paragraph{\texorpdfstring{89) List all irreducible polynomials up to
degree \(4\) over
\(\mathbb{Z}_3\).}{89) List all irreducible polynomials up to degree 4 over \textbackslash mathbb\{Z\}\_3.}}\label{list-all-irreducible-polynomials-up-to-degree-4-over-mathbbz_3.}}

We recall: \emph{A nonzero, nonunit \(a \in R\) is irreducible iff
\(a=bc\) with \(b,c \in R\) means that either \(b\) is unit or \(c\) is
unit.}

All linear polynoms \(p \in \mathbb{Z}_3\) are irreducible since one of
the factors of \(p\) has to be constant and the possible constant
factors in \(\mathbb{Z}_3\) are \(1,2\) which are both unit
(\(1 \cdot 1 = 1\), \(2 \cdot 2 = 1\)).

Thus the irreducible polynoms in \(\mathbb{Z}_3\) of degree \(1\) are:
\(x, 2x, x+1, x+2, 2x+1, 2x+2\).

For a polynomial \(p \in \mathbb{Z}_3\) with degree \(>1\), \(p\) cannot
have a constant factor as we argued before, thus we need to check
whether \(p\) has a linear factor, i.e.~if \(p\) has no zeroes.

A polynomial in \(\mathbb{Z}_3\) has no zeroes iff
\(ax^3 +bx^2 + cx + d \neq 0\), which in turn holds if:

\begin{itemize}
\tightlist
\item
  \(d \neq 0\) for the case of \(x = 0\)
\item
  \(a+b+c+d \neq 0\) for the case of \(x = 1\)
\item
  \(2a + b + 2c + d \neq 0\) for the case of \(x = 2\)
\end{itemize}

Now, we can generate all irreducible polynomials in \(\mathbb{Z}_3\)
with degree \(geq 2\) by setting all possible values for \(a,b,c,d\)
such that the conditions are satisfied:

\begin{itemize}
\tightlist
\item
  a = 0, b = 1, c = 0, d = 1: \(x^2 + 1\)
\item
  a = 0, b = 1, c = 1, d = 2: \(x^2 + x + 2\)
\item
  a = 0, b = 1, c = 2, d = 2: \(x^2 + 2x + 2\)
\item
  a = 0, b = 2, c = 1, d = 1: \(2x^2 + x + 1\)
\item
  a = 1, b = 0, c = 2, d = 1: \(x^3 + 2x + 1\)
\item
  a = 1, b = 0, c = 2, d = 2: \(x^3 + 2x + 2\)
\item
  a = 1, b = 1, c = 0, d = 2: \(x^3 + x^2 + 2\)
\item
  a = 1, b = 1, c = 2, d = 1: \(x^3 + x^2 + 2x + 1\)
\item
  a = 1, b = 2, c = 0, d = 1: \(x^3 + 2x^2 + 1\)
\item
  a = 1, b = 2, c = 1, d = 1: \(x^3 + 2x^2 + x + 1\)
\item
  a = 1, b = 2, c = 2, d = 2: \(x^3 + 2x^2 + 2x + 2\)
\item
  a = 2, b = 0, c = 1, d = 1: \(2x^3 + x + 1\)
\item
  a = 2, b = 0, c = 1, d = 2: \(2x^3 + x + 2\)
\item
  a = 2, b = 1, c = 0, d = 2: \(2x^3 + x^2 + 2\)
\item
  a = 2, b = 1, c = 1, d = 1: \(2x^3 + x^2 + x + 1\)
\item
  a = 2, b = 2, c = 0, d = 1: \(2x^3 + 2x^2 + 1\)
\item
  a = 2, b = 2, c = 1, d = 2: \(2x^3 + 2x^2 + x + 2\)
\item
  a = 2, b = 2, c = 2, d = 2: \(2x^3 + 2x^2 + 2x + 2\)
\end{itemize}

\hypertarget{decompose-x4-x3-1-into-irreducible-factors-over-mathbbz_2.}{%
\paragraph{\texorpdfstring{90) Decompose \(x^4 + x^3 + 1\) into
irreducible factors over
\(\mathbb{Z}_2\).}{90) Decompose x\^{}4 + x\^{}3 + 1 into irreducible factors over \textbackslash mathbb\{Z\}\_2.}}\label{decompose-x4-x3-1-into-irreducible-factors-over-mathbbz_2.}}

We try to find non-unit, i.e.~non-constant, factors of
\(p := x^4 + x^3 + 1\). Since \(p\) has degree \(4\) it can either be a
product of two quadratic factors or one linear and one cubic factor.

\begin{itemize}
\tightlist
\item
  case 1: \(x^4 +x^3 +1 = b \cdot d\) and w.l.o.g. \(b\) is linear, we
  know either

  \begin{itemize}
  \tightlist
  \item
    \(b = x\), but \(\frac{x^4 +x^3 +1}{x}\) does not result in an
    element from \(\mathbb{Z}_2\), or
  \item
    \(b = x+1\) but again \(\frac{x^4 +x^3 +1}{x}\) does not result in
    an element from \(\mathbb{Z}_2\).
  \end{itemize}
\item
  case 2: \(x^4 +x^3 +1 = b \cdot d\) where \(b\) and \(d\) are
  quadratic, irreducible polynomials, these are

  \begin{itemize}
  \tightlist
  \item
    \(x^2+1\) but \(x^2+1\) does not divide \(p\) cleanly in
    \(\mathbb{Z}_2\).
  \item
    \(x^2+x+1\) but this does not divide \(p\) either.
  \end{itemize}
\end{itemize}

We conclude, there exists no factorisation of \(x^4+x^3+1\) in
\(\mathbb{Z_2}\).

\hypertarget{exercise-sheet-10}{%
\subsection{Exercise Sheet 10}\label{exercise-sheet-10}}

\hypertarget{let-k-be-a-field-and-px-in-kx-a-polynomial-of-degree-m.-prove-that-px-cannot-have-more-than-m-zeros-in-k-counted-with-multiplicities.}{%
\paragraph{\texorpdfstring{91) Let \(K\) be a field and
\(p(x) \in K[x]\) a polynomial of degree \(m\). Prove that \(p(x)\)
cannot have more than \(m\) zeros in \(K\) (counted with
multiplicities).}{91) Let K be a field and p(x) \textbackslash in K{[}x{]} a polynomial of degree m. Prove that p(x) cannot have more than m zeros in K (counted with multiplicities).}}\label{let-k-be-a-field-and-px-in-kx-a-polynomial-of-degree-m.-prove-that-px-cannot-have-more-than-m-zeros-in-k-counted-with-multiplicities.}}

\emph{Hint: Use the fact that \(K[x]\) is a factorial ring.}

Firstly we recall: \[
\begin{aligned}
p(a) = 0 &\ \Leftrightarrow (x-a) | p\\
\text{"$a$ has an $n$-fold zero"} &:\Leftrightarrow (x-a)^n | p
\end{aligned}
\]

By \(K[x]\) being a factorial ring, \(p\) can be written as: \[
p = u \prod_i p_i, \quad i \geq 0
\] We can partition this product as follows: \[
p = u \prod_i (x-a_i)^{ni} \cdot \prod q
\] where the left product represents the product of all \(n\)-fold zeros
of \(p\), where each \((x-a_i)\) is irreducible and the right product
represents the remaining factors of \(p\). Additionally, \[
\prod_i (x-a_i)^{ni} | p 
\] by construction. Hence, the degree of the left side has to be smaller
or equal to the degree of the right side, otherwise \(p\) would be zero.
This means the degree of the left side is smaller or equal to \(m\).
Since the left side represents the amount of zeros in \(p\) this means
the number of zeros in \(p\) has to be less or equal to \(m\).

\hypertarget{let-r-be-a-ring-and-i_j_j-in-j-be-a-family-of-ideals-of-r.-prove-that-bigcap_j-in-j-i_j-is-an-ideal-of-r.}{%
\paragraph{\texorpdfstring{92) Let \(R\) be a ring and
\((I_j)_{j \in J}\) be a family of ideals of \(R\). Prove that
\(\bigcap_{j \in J} I_j\) is an ideal of
\(R\).}{92) Let R be a ring and (I\_j)\_\{j \textbackslash in J\} be a family of ideals of R. Prove that \textbackslash bigcap\_\{j \textbackslash in J\} I\_j is an ideal of R.}}\label{let-r-be-a-ring-and-i_j_j-in-j-be-a-family-of-ideals-of-r.-prove-that-bigcap_j-in-j-i_j-is-an-ideal-of-r.}}

We recall: I is an ideal of \(R\) iff:

\begin{enumerate}
\def\labelenumi{\arabic{enumi})}
\item
  \(I \neq \emptyset\): Since \(0\) is contained in any ideal by
  definition, it will also be in the cut of ideals.
\item
  \(\forall x,y \in I: x -y \in I\): Since \(\forall j \in J: I_j\) is
  an ideal: by definition of the cut operation \(\forall a,b \in I_j\),
  \(a +(-b) \in I_j\) as \(I_j\) is an ideal. As this holds for any
  \(I_j\), \(a+ (-b) \in I\).
\item
  \(\forall i \in I, r \in R: i \cdot r \in I \land r \cdot i \in I\)
  Assume an arbitrary \(r \in R\) and \(i \in I\), then forall
  \(j \in J, x \in I_j\) and \(r \cdot x \in I_j\) by \(I_j\) being an
  ideal. Thus \(r \cdot i \in I\) by definition of the cut operation.
  The same argument holds for \(I \cdot r\).
\end{enumerate}

\hypertarget{let-r-be-a-ring-and-ij-two-ideals-of-r.-does-this-imply-that-i-cup-j-is-an-ideal-of-r}{%
\paragraph{\texorpdfstring{93) Let \(R\) be a ring and \(I,J\) two
ideals of \(R\). Does this imply that \(I \cup J\) is an ideal of
\(R\)?}{93) Let R be a ring and I,J two ideals of R. Does this imply that I \textbackslash cup J is an ideal of R?}}\label{let-r-be-a-ring-and-ij-two-ideals-of-r.-does-this-imply-that-i-cup-j-is-an-ideal-of-r}}

No, we provide a counterexample:

Let \(R = (\mathbb{Z}, +)\), \(I = 2 \mathbb{2}\) and
\(J = 3 \mathbb{Z}\), then \(2 \in I\) thus \(2 \in I \cup J\), and
\(3 \in J\), thus \(3 \in I \cup J\).

But \(2+3 \not \in 2\mathbb{Z} \cup 3\mathbb{Z}\) since
\(5 \not \in 2\mathbb{Z}\) and \(5 \not \in 3\mathbb{Z}\).

\hypertarget{let-r-be-a-ring-and-i-be-an-ideal-of-r.-then-r-i-is-the-quotient-group-of-r-over-i.-define-a-multiplication-on-ri-by}{%
\paragraph{\texorpdfstring{94) Let \(R\) be a ring and \(I\) be an ideal
of \(R\). Then \((R/ I,+)\) is the quotient group of \((R,+)\) over
\((I,+)\). Define a multiplication on \(R/I\)
by}{94) Let R be a ring and I be an ideal of R. Then (R/ I,+) is the quotient group of (R,+) over (I,+). Define a multiplication on R/I by}}\label{let-r-be-a-ring-and-i-be-an-ideal-of-r.-then-r-i-is-the-quotient-group-of-r-over-i.-define-a-multiplication-on-ri-by}}

\[
(a+I)\cdot (b+I) := (ab)+I
\] \textbf{Prove that this operation is well-defined, i.e.~that}

\[
\left.
  \begin{array}{lr}
    &a+I = c+I \\
    \text{and } &b+I  = d+I
  \end{array}
\right\} \Longrightarrow (ab) + I = (cd) +I.
\]

Assume, \(a,b,c,d\) such that \(a + I = c + I\) and \(b + I = d + I\).

By \(a + I = c + I\), we know \((a+I) - (c+I) = I\) (since \(I\) is the
neutral element in the quotient group).

But additionally by definition of addition in \(R/I\),
\((a+I) - (c+I) = (a-c) + I\) and
\((a-c) + I = I \Leftrightarrow (a-c) \in I\). Thus \(a-c \in I\) and
similarly \((b-d) \in I\). Now by ideal properties: \(b(a-c) \in I\) and
\(c(b-d) \in I\) and by subgroup properties: \(b(a-c) + c(b-d) \in I\),
i.e.~\(ba-bc+bc-bd = ab - bd \in I\). Therefore, \(ab + I = cd + I\).
\textbf{Furthermore, show that \((R/I,+,\cdot)\) is a ring.}

\begin{itemize}
\tightlist
\item
  \textbf{\((R/I, +)\) is abelian:} This follows directly from \((R,+)\)
  being abelian and \(I\) being a normal subgroup.
\item
  \textbf{\((R/I, \cdot)\) is a semigroup, i.e.~satisfies
  associativity:} We show associativity by repeatedly applying the
  definitions of the respective \(+\) and \(\cdot\) operations:
  \(((a+I) \cdot (b+I)) \cdot (c+I) =\)
  \((ab + I) \cdot (c+I) = abc + I = (a+I) \cdot (bc + I) =\)
  \((a+I) \cdot ((b+I) \cdot (c+I))\)
\item
  \textbf{distributivity:} We show distributivity by repeatedly applying
  the definitions of the respective \(+\) and \(\cdot\) operations:
  \((a+I) \cdot ((b+I) + (c +I )) =\)
  \((a+I) \cdot ((c+b) + I) = a(c+b) + I = (ac +ab) + I =(ac + I) + (ab + I) =\)
  \((a+I) \cdot (c + I) + (a + I) \cdot (b+I)\) \#\#\#\# 95) Let
  \(U = \{\bar0, \bar2, \bar4\} \subseteq \mathbb{Z}_6\). Show that
  \(U\) is an ideal \(\mathbb{Z}_6\). Let
  \(R = (\mathbb{Z}_6,+,\cdot)\): We recall, \(U\) is an ideal of \(R\)
  iff
\end{itemize}

\begin{enumerate}
\def\labelenumi{\arabic{enumi})}
\tightlist
\item
  \(U \neq \emptyset\) \(U \neq \emptyset\)
\item
  \(\forall x,y \in U: x + (-y) \in U\)
\end{enumerate}

\begin{longtable}[]{@{}llll@{}}
\toprule()
\(-\) & \(\bar 0\) & \(\bar 2\) & \(\bar 4\) \\
\midrule()
\endhead
\(\bar 0\) & \(\bar 0\) & \(\bar 4\) & \(\bar 2\) \\
\(\bar 2\) & \(\bar 2\) & \(\bar 0\) & \(\bar 4\) \\
\(\bar 4\) & \(\bar 4\) & \(\bar 2\) & \(\bar 0\) \\
\bottomrule()
\end{longtable}

\begin{enumerate}
\def\labelenumi{\arabic{enumi})}
\setcounter{enumi}{2}
\tightlist
\item
  \(\forall u \in U, r \in R: u \cdot r \in U \land r \cdot u \in U\)
\end{enumerate}

\begin{longtable}[]{@{}llll@{}}
\toprule()
\(\cdot\) & \(\bar 0\) & \(\bar 2\) & \(\bar 4\) \\
\midrule()
\endhead
\(\bar 0\) & \(\bar 0\) & \(\bar 0\) & \(\bar 0\) \\
\(\bar 1\) & \(\bar 0\) & \(\bar 2\) & \(\bar 4\) \\
\(\bar 2\) & \(\bar 0\) & \(\bar 2\) & \(\bar 0\) \\
\(\bar 3\) & \(\bar 0\) & \(\bar 0\) & \(\bar 0\) \\
\(\bar 4\) & \(\bar 0\) & \(\bar 2\) & \(\bar 4\) \\
\(\bar 5\) & \(\bar 0\) & \(\bar 4\) & \(\bar 2\) \\
\bottomrule()
\end{longtable}

All conditions are satisfied, thus \(U\) is an ideal of \(R\).

\textbf{Is it a subring as well?} Conditions 1) and 2) show \(U\) is a
subgroup of \(R\). In 3) we showed \(U\) is closed under multiplication.

\textbf{Does it have a \(1\)-element?} In 3) one can see the
multiplicative identity, namely \(\bar 4\).

\hypertarget{show-that-mathbbzx-cdot-is-a-ring-and-that-1-neq-x-x2.}{%
\paragraph{\texorpdfstring{96) Show that \((\mathbb{Z}[x], +, \cdot)\)
is a ring and that
\(1 \neq (\{x, x+2\})\).}{96) Show that (\textbackslash mathbb\{Z\}{[}x{]}, +, \textbackslash cdot) is a ring and that 1 \textbackslash neq (\textbackslash\{x, x+2\textbackslash\}).}}\label{show-that-mathbbzx-cdot-is-a-ring-and-that-1-neq-x-x2.}}

Let \(i\) be an arbitrary element such that \(i \in (\{x, x+2\})\), we
can rewrite \(i = i_1 x + i_2(x+2)\). Now let \(i_2 = i_2'x + c\), where
\(c\) is some constant factor, this means \[
p = p_1 x + (p_2'x+c)(x+2) = p_1 x + p_2'x^2 + cx + 2p_2'x + 2c
\] we can partition this product into factors of \(x\) and constant
factors: \[
p = x(p_1 + p_2' + 2p_2' x + c) + 2c
\]

Now, assume \(p = 1\), then \((p_1 + p_2' + 2p_2' x + c) = 0\) has to
hold, but \(2c \neq 1\). Contradiction!

\emph{Remark: It can be showen that a principal deal which is generated
by \(a_1,a_2,\dots,a_k\) can be alternatively generated by
\(gcd(a_1,a_2, \dots, a_k)\). Therefore this example shows that
\(\mathbb{Z}[x]\) is a ring where not every ideal is a principal ideal.
As a consequence, \$\mathbb{Z}{[}x{]} cannot be a Euklidian ring.}

\hypertarget{prove-if-rcdot-is-a-ring-and-i_1-i_2-two-of-its-ideals-then}{%
\paragraph{\texorpdfstring{97) Prove: If \((R,+,\cdot)\) is a ring and
\(I_1\), \(I_2\) two of its ideals,
then}{97) Prove: If (R,+,\textbackslash cdot) is a ring and I\_1, I\_2 two of its ideals, then}}\label{prove-if-rcdot-is-a-ring-and-i_1-i_2-two-of-its-ideals-then}}

\[
\begin{aligned}
I_1 + I_2 &:= \{a+b| a \in I_1, b \in I_2\} \text{, and} \\
I_1 * I_2 &:= \{a_1 b_1 + a_2 b_2 + \cdots + a_n b_n | n \geq 1, a_i \in I_1, b_i \in I_2 \text{ for } 1 \leq i \leq n\}
\end{aligned}
\] \textbf{are ideals as well.} Let \(I' = I_1 + I_2\), then:

\begin{enumerate}
\def\labelenumi{\arabic{enumi})}
\item
  \(I' \neq \emptyset\) by \(I_1 \neq \emptyset\) and
  \(I_2 \neq \emptyset\), \(I' \neq \emptyset\) by construction.
\item
  \(\forall x,y \in I': x + (-y) \in I'\) we can write \(x\) as
  \(a_1+b_1\) and \(y\) as \(a_2 + b_2\), then
  \(x + (-y) = (a_1+b_1) - (a_2 + b_2) = a_1 -a_2 + b_1 - b_2\), and
  since \(a_1,a_2 \in I_1\) , \(a_1 - a_2 \in I_1\) by \(I_1\) being an
  ideal. Similarly, since \(b_1,b_2 \in I_2\) , \(b_1 - b_2 \in I_2\) by
  \(I_2\) being an ideal. Therefore, by definition
  \((a_1 - a_2) + (b_1 - b_2) \in I'\).
\item
  \(\forall i \in I', r \in R: i \cdot r \in I' \land r \cdot i \in I'\)
  Assume arbitrary \(r \in R, i \in I'\), we can rewrite \(i\) by
  \((a+b)\) for some \(a \in I_1, b \in I_2\). Then
  \(r \cdot i = r \cdot (a+b) = r \cdot a + r \cdot b\) and since
  \(a \in I_1, r \in R\), \(r \cdot a \in I_1\) by \(I_1\) being an
  ideal, similarly, since \(b \in I_2, r \in R\), \(r \cdot b \in I_2\)
  by \(I_2\) being an ideal. Therefore, \(r \cdot a + r \cdot b \in I'\)
  by definition.
\end{enumerate}

Since all necessary conditions hold, we conclude \(I'\) is an ideal.

Let \(I'' = I_1 * I_2\), then:

\begin{enumerate}
\def\labelenumi{\arabic{enumi})}
\tightlist
\item
  \(I' \neq \emptyset\) by \(I_1 \neq \emptyset\) and
  \(I_2 \neq \emptyset\), \(I' \neq \emptyset\) by construction.
\item
  \(\forall x,y \in I'': x + (-y) \in I''\) we can write \(x\) as
  \(a_1 b_1+ \cdots + a_n b_n\) and \(y\) as
  \(a'_1b'_1 + \cdots + a'_m b'_m\). W.l.o.g. assume \(n > m\): then
  \(x-y = (a_1 b_1 - a'_1b'_1 + \cdots + a'_m b'_m - a_m b_m - \dots - a_n b_n)\)
  then each summand takes one of two forms:

  \begin{itemize}
  \tightlist
  \item
    \(a_ib_i\), where \(a_i \in I_1\) and \(b_i \in I_2\), or
  \item
    \((-a_i)b_i\), where since \(a_i \in I_1\) and \(0 \in I_1\) (by
    definition), \(-a_i \in I_1\) and \(b_i \in I_2\). Thus,
    \((a_1 b_1 - a'_1b'_1 + \cdots + a'_m b'_m - a_m b_m - \dots - a_n b_n)\)
    is in \(I''\).
  \end{itemize}
\item
  \(\forall i \in I'', r \in R: i \cdot r \in I'' \land r \cdot i \in I''\)
  Assume arbitrary \(r \in R, i \in I'\), we can rewrite \(i\) by
  \(a_1 b_1 + a_2 b_2 + \cdots + a_n b_n\) and \(r \cdot i\) by
  \(r(a_1 b_1) + \cdots + r(a_n b_n)\). Now, for any \(a_ib_i\),
  \(1 \leq i \leq n\): \(r \cdot a_i\) has to hold for any \(r \in R\)
  by \(I_1\) being an ideal, therefore
  \(r a_1 b_1 + \dots + ra_n b_n \in I''\). Since all necessary
  conditions hold, we conclude \(I''\) is an ideal.
\end{enumerate}

\hypertarget{show-that-the-set-r-absqrt2-ab-in-mathbbz-with-the-usual-addition-and-multiplication-is-an-integral-domain-but-not-a-field.}{%
\paragraph{\texorpdfstring{98) Show that the set
\(R = \{a+b\sqrt{2} | a,b \in \mathbb{Z}\}\) with the usual addition and
multiplication is an integral domain but not a
field.}{98) Show that the set R = \textbackslash\{a+b\textbackslash sqrt\{2\} \textbar{} a,b \textbackslash in \textbackslash mathbb\{Z\}\textbackslash\} with the usual addition and multiplication is an integral domain but not a field.}}\label{show-that-the-set-r-absqrt2-ab-in-mathbbz-with-the-usual-addition-and-multiplication-is-an-integral-domain-but-not-a-field.}}

\emph{An integral domain is a commutative ring with \(1\)-element and no
zero divisor}

\begin{itemize}
\item
  \textbf{Commutative ring:} We show \(R \leq \mathbb{R}\): *
  \(R \neq \emptyset\) trivially by \(\mathbb{Z} \neq \emptyset\) *
  \(\forall x,y \in R: x-y \in R\): we can write \(x = a + b \sqrt{2}\)
  and \(y = a' + b' \sqrt{2}\) then:
  \(x-y = a + b \sqrt{2} - (a' + b' \sqrt{2}) = a + b\sqrt{2} - a' - b'\sqrt{2} = a-a' + (b-b')\sqrt{2}\),
  where \(a-a' \in \mathbb{Z}\) and \(b-b'\in \mathbb{Z}\). *
  \(\forall x,y \in R: x \cdot y \in R\): we again can write
  \(x = a + b \sqrt{2}\) and \(y = a' + b' \sqrt{2}\) then:
  \(x \cdot y = (a + b \sqrt{2}) \cdot (a' + b'\sqrt{2}) = aa' + ab'\sqrt{2} + a'b\sqrt{2} + 2bb' = (aa'+2bb') + (ab' + a'b)\sqrt{2}\),
  where \((aa'+2bb') \in \mathbb{Z}\) and
  \((ab' + a'b) \in \mathbb{Z}\). Thus, \(R\) is a ring, and \(R\) is
  commutative by commutativity of \(\mathbb{R}\).
\item
  \textbf{\(1\)-element:} The one element is inherited from
  \(\mathbb{R}\), namely \(1 + 0 \sqrt{2} = 1\).
\item
  \textbf{No zero divisor:} Assume there exists a non-zero zero divisor
  \(a \in R\), meaning there is some \(x \in R\) such that \(ax = 0\)
  and \(x \neq 0\). We define a homomorphism
  \(\varphi: R \mapsto \mathbb{Z}\), where
  \(\varphi(a+b\sqrt{2}) = a^2 -2b^2\). We show
  \(\varphi(x) \cdot \varphi(y) = \varphi(xy)\): Let
  \(x = a'^2 - 2b'^2\) and \(y = a''^2 - 2b''^2\), then:
  \(\varphi(x) \cdot \varphi(y) = a'^2a''^2 - 2a'^2b''^2 - 2a''^2b'^2 + 4b'^2b''^2\)
  and: \(\varphi(xy) = (a'a'' + 2b'b'')^2 - 2(a'b'' + a''b')^2 =\)
  \(a'^2a''^2 + 4a'a''b'b'' + 4 b'^2b''^2 - 2a'^2b''^2 - 4a'b''a''b' - 2a''^2b'^2 =\)
  \(a'^2a''^2 - 2a'^2b''^2 - 2a''^2b'^2 + 4b'^2b''^2 = \varphi(x) \cdot \varphi(y)\)

  Then \(ax = 0\) has to hold, and \(\varphi(ax) = \varphi(0)\) has to
  be satisfied, now by \(\varphi\) being a homomorphism:
  \(\varphi(a) \cdot \varphi(x) = \varphi(0) = 0\), but
  \(\varphi(y) = 0\) only holds for \(y = 0\), thus either \(a = 0\) or
  \(x = 0\). Contradiction!
\item
  \textbf{But not a field, i.e.: \(\exists a \neq 0\) such that \(a\)
  has no multiplicative inverse.} \(\varphi(2) = 4\), \(\varphi(1) = 1\)
  \(\varphi(2x)\) = \(\varphi(2) \cdot \varphi(x) = 4 \varphi(x)\) Now
  if \(2\) was unit: \(\varphi(2x) = 1\) would have to hold, but
  \(\varphi(2x) = 4\varphi(x) = 1\) cannot be the case since
  \(\varphi(x) \in \mathbb{Z}\).
\end{itemize}

\textbf{Furthermore, prove that there are infinitely many units in \(R\)
and give three concrete examples.}

\hypertarget{find-all-irreducible-polynomials-of-mathbbrx.}{%
\paragraph{\texorpdfstring{99) Find all irreducible polynomials of
\(\mathbb{R}[x]\).}{99) Find all irreducible polynomials of \textbackslash mathbb\{R\}{[}x{]}.}}\label{find-all-irreducible-polynomials-of-mathbbrx.}}

\emph{Hint: Show first that, if a polynomial \(p(x) \in \mathbb{R}[x]\)
has a complex zero \(a+bi\), then its conjugate \(a-bi\) is a zero of
\(p(x)\), too. Use this fact to conclude that every polynomial of degree
at least \(3\) is reducible in \(\mathbb{R}[x]\).}

\hypertarget{show-that-sqrt2-sqrt3-is-algebraic-over-mathbbq-and-determine-its-minimal-polynomial.}{%
\paragraph{\texorpdfstring{100) Show that \(\sqrt{2} + \sqrt{3}\) is
algebraic over \(\mathbb{Q}\) and determine its minimal
polynomial.}{100) Show that \textbackslash sqrt\{2\} + \textbackslash sqrt\{3\} is algebraic over \textbackslash mathbb\{Q\} and determine its minimal polynomial.}}\label{show-that-sqrt2-sqrt3-is-algebraic-over-mathbbq-and-determine-its-minimal-polynomial.}}

Note the definition of algebraic: \emph{Let \(P(x)\) be monic and
irreducible over \(K\), with degree \(deg(P) = n\) and \(P(a) = 0\).
Then \(a\) is algebraic over \(K\) and \(P(x)\) is a minimal polynomial
of \(a\) over \(K\).}

Thus, we determine whether \(a = \sqrt{2} + \sqrt{3}\) is algebraic over
\(\mathbb{Q}\) by:

\[
\begin{aligned}
x   &= \sqrt{2} + \sqrt{3}\\
x^2 &= (\sqrt{2} + \sqrt{3})^2 = 2 + 2 \sqrt{6} + 3\\
x^2 - 5 &= 2 \sqrt{6}\\
(x^2 - 5)^2 &= (2 \sqrt{6})^2 = 4 \cdot 6\\
x^4 -10x^2 + 25 &= 24\\
x^4 -10x^2 + 1 &= 0
\end{aligned}
\] Thus \(f(x) = x^4 -10x^2 + 1\) has \(a\) as a zero in \(\mathbb{Q}\),
meaning \(a\) is algebraic over \(\mathbb{Q}\).

Additionally \(f(x)\) is the minimal polynomial of \(a\) over
\(\mathbb{Q}\), since the minimal polynomial of \(a\) has to have degree
at least \(4\), and it is already monic (the coefficient of the largest
degree component is \(1\)).

\hypertarget{exercise-sheet-11}{%
\subsection{Exercise Sheet 11}\label{exercise-sheet-11}}

\hypertarget{each-element-of-mathbbqsqrt2-sqrt3-can-be-uniquely-represented-in-the-form-a-bsqrt2-c-sqrt3-dsqrt6-where-abcd-in-mathbbq.-therefore-it-can-be-identified-with-a-quadruple-abcd-in-mathbbq4.-determine-the-quadruple-corresponding-to-a-bsqrt2-c-sqrt3-dsqrt6a-b-sqrt2-c-sqrt3-d-sqrt6.}{%
\paragraph{\texorpdfstring{101) Each element of
\(\mathbb{Q}(\sqrt{2}, \sqrt{3})\) can be uniquely represented in the
form \(a + b\sqrt{2} + c \sqrt{3} + d\sqrt{6}\) where
\(a,b,c,d \in \mathbb{Q}\). Therefore it can be identified with a
quadruple \((a,b,c,d) \in \mathbb{Q}^4\). Determine the quadruple
corresponding to
\((a + b\sqrt{2} + c \sqrt{3} + d\sqrt{6})(a' + b' \sqrt{2} + c' \sqrt{3} + d' \sqrt{6})\).}{101) Each element of \textbackslash mathbb\{Q\}(\textbackslash sqrt\{2\}, \textbackslash sqrt\{3\}) can be uniquely represented in the form a + b\textbackslash sqrt\{2\} + c \textbackslash sqrt\{3\} + d\textbackslash sqrt\{6\} where a,b,c,d \textbackslash in \textbackslash mathbb\{Q\}. Therefore it can be identified with a quadruple (a,b,c,d) \textbackslash in \textbackslash mathbb\{Q\}\^{}4. Determine the quadruple corresponding to (a + b\textbackslash sqrt\{2\} + c \textbackslash sqrt\{3\} + d\textbackslash sqrt\{6\})(a\textquotesingle{} + b\textquotesingle{} \textbackslash sqrt\{2\} + c\textquotesingle{} \textbackslash sqrt\{3\} + d\textquotesingle{} \textbackslash sqrt\{6\}).}}\label{each-element-of-mathbbqsqrt2-sqrt3-can-be-uniquely-represented-in-the-form-a-bsqrt2-c-sqrt3-dsqrt6-where-abcd-in-mathbbq.-therefore-it-can-be-identified-with-a-quadruple-abcd-in-mathbbq4.-determine-the-quadruple-corresponding-to-a-bsqrt2-c-sqrt3-dsqrt6a-b-sqrt2-c-sqrt3-d-sqrt6.}}

We mark and sort the summands according to their square root factor: \[
\color{green}{aa'} + \color{brown}{ab'\sqrt{2}} + \color{blue}{ac'\sqrt{3}} + \color{pink}{ad'\sqrt{6}} + \color{brown}{b\sqrt{2}a'} + \color{green}{b\sqrt{2}b'\sqrt{2}} + \color{pink}{b\sqrt{2}c'\sqrt{3}} + \color{blue}{b\sqrt{2}d'\sqrt{6}} + \\
\color{blue}{c\sqrt{3}a'} + \color{pink}{c\sqrt{3}b'\sqrt{2}} + \color{green}{c \sqrt{3} c'\sqrt{3}} + \color{brown}{c\sqrt{3}d'\sqrt{6}} + \color{pink}{d\sqrt{6}a'} + \color{blue}{d\sqrt{6} b'\sqrt{2}} + \color{brown}{d\sqrt{6} c' \sqrt{3}} + \color{green}{d\sqrt{6} d' \sqrt{6}} 
\]

Thus the quadrupel is the following: \[
(\color{green}{aa'} + \color{green}{b\sqrt{2}b'\sqrt{2}} + \color{green}{c \sqrt{3} c'\sqrt{3}} + \color{green}{d\sqrt{6} d' \sqrt{6}},\\
\color{brown}{ab'\sqrt{2}} + \color{brown}{b\sqrt{2}a'} + 
\color{brown}{c\sqrt{3}d'\sqrt{6}} + \color{brown}{d\sqrt{6} c'}, \\
\color{blue}{ac'\sqrt{3}} + \color{blue}{b\sqrt{2}d'\sqrt{6}} + \color{blue}{d\sqrt{6} b'\sqrt{2}} + \color{blue}{c\sqrt{3}a'}, \\
\color{pink}{ad'\sqrt{6}} + \color{pink}{b\sqrt{2}c'\sqrt{3}} + \color{pink}{c\sqrt{3}b'\sqrt{2}} + \color{pink}{d\sqrt{6}a'})
\]

\hypertarget{show-that-the-set-absqrt2ab-in-mathbbq-with-the-usual-addition-and-multiplication-is-a-field.}{%
\paragraph{\texorpdfstring{102) Show that the set
\(\{a+b\sqrt{2}|a,b \in \mathbb{Q}\}\) with the usual addition and
multiplication is a
field.}{102) Show that the set \textbackslash\{a+b\textbackslash sqrt\{2\}\textbar a,b \textbackslash in \textbackslash mathbb\{Q\}\textbackslash\} with the usual addition and multiplication is a field.}}\label{show-that-the-set-absqrt2ab-in-mathbbq-with-the-usual-addition-and-multiplication-is-a-field.}}

Let \(S = \{a+b\sqrt{2}|a,b \in \mathbb{Q}\}\),

\begin{itemize}
\item
  \(S\) is trivially closed under addition by \(\mathbb{Q}\) being a
  field.
\item
  As can be shown analogously to 101), \(S\) is closed under
  multiplication.
\item
  Each element in \(S\) has a multiplicative inverse, i.e.: For any
  \(a+ \sqrt{2}b \neq 0\) there exists \(c + \sqrt{2}d\) such that
  \((a+\sqrt{2}b)(c + \sqrt{2}d = 1\) meaning
  \(ac + 2bd + \sqrt{2}ad + \sqrt{2}bc = 1\) thus we get the set of
  equations: \[
    ad + bc = 0\\
    ac + 2bd = 1
    \]

  We can use matrices to solve this system of equations: \[
    \begin{bmatrix}
    a & 2b\\
    b & a
    \end{bmatrix} \cdot \begin{bmatrix}
    c \\
    d 
    \end{bmatrix} = \begin{bmatrix}
    1 \\
    0
    \end{bmatrix}
    \iff \begin{bmatrix}
    c\\
    d 
    \end{bmatrix}=
    \begin{bmatrix}
    a & 2b\\
    b & a
    \end{bmatrix}^{-1} \cdot
    \begin{bmatrix}
    1 \\
    0
    \end{bmatrix}
    \] and \[
    \begin{bmatrix}
    a & 2b\\
    b & a
    \end{bmatrix}^{-1} = \frac{1}{a^2-2b^2}\begin{bmatrix}
    a & -2b\\
    -b & a
    \end{bmatrix}^{-1}
    \] thus \[
    \begin{bmatrix}
    c \\
    d 
    \end{bmatrix} = \frac{1}{a^2-2b^2} \begin{bmatrix}
    a \\
    -b 
    \end{bmatrix}
    \] meaning \(c = \frac{a}{a^2-2b^2}, d = \frac{-b}{a^2-2b^2}\) which
  are both in \(\mathbb{Q}\) since \(a = b = 0\) is not possible by
  construction.

  \textbf{Compute \((3-5\sqrt{2})^{-1}\).} \((3-5\sqrt{2})^{-1}\) is
  thus \(c = \frac{3}{-41}, d = \frac{-5}{41}\).
\end{itemize}

\hypertarget{let-k-be-a-field-with-chark-p.-prove-that-abp-ap-bp-for-all-ab-in-k.}{%
\paragraph{\texorpdfstring{104) Let \(K\) be a field with
\(char(K) = p\). Prove that \((a+b)^p = a^p + b^p\) for all
\(a,b \in K\).}{104) Let K be a field with char(K) = p. Prove that (a+b)\^{}p = a\^{}p + b\^{}p for all a,b \textbackslash in K.}}\label{let-k-be-a-field-with-chark-p.-prove-that-abp-ap-bp-for-all-ab-in-k.}}

\emph{Hint: Use the binomial theorem and consider the equation
\(\binom{p}{k} = p \cdot \frac{(p-1)!}{k!(p-k)!}\) for \(0 < k < p\).
Show that \(\binom{p}{k} \in \mathbb{N}\) implies that the fraction on
the right-hand side must be an integer, too, since the factors in the
denominator do not divide \(p\).} We apply the binomial theorem: \[
(a+b)^p = \sum_{k=0}^p \binom{p}{k} a^{p-k} b^k 
\] we split the sum in order to be able to apply the next step of the
hint: \[
\binom{p}{0} a^pb^0 + \sum_{k=1}^{p-1} p\cdot \frac{(p-1)!}{k!(p-k)!} a^{p-k}b^k + \binom{p}{p}a^0 b^p = \\
= a^p + \sum_{k=1}^{p-1} p \cdot \frac{(p-1)!}{k!(p-k)!} \cdot a^{p-k} \cdot b^k + b^p
\] Now, we know
\(\binom{p}{k} = p \cdot \frac{(p-1)!}{k!(p-k)!} \in \mathbb{N}\) and we
know \(k!(p-k)!\) does not divide \(p\) as \(k \leq p-1 < p\) and thus
each \(k (k-1) \cdots 1\) does not divide \(p\), analogously this holds
for \((p-k)!\)'s factors. Since \(p\) is prime, no number smaller than
\(p\) divides \(p\), therefore
\(\frac{(p-1)!}{k!(p-k)!} \in \mathbb{N}\) as \(p \in \mathbb{N}\) and
\(\binom{p}{k} \in \mathbb{N}\). Thus
\(p \cdot \frac{(p-1)!}{k!(p-k)!}\) can be written as \(p \cdot x\)
where \(x \in \mathbb{N}\) and thus \(x \in K\) as well.

This means \(px = (x + x + \dots + x) = x(1+ \dots + 1) = 0\) by
definition of \(char(K) = p\).

\hypertarget{consider-the-field-mathbbz_2x-mx-where-mx-x3-x4-x3-x-1.-hence-the-residue-classes-module-mx-are}{%
\paragraph{\texorpdfstring{105) Consider the field
\(\mathbb{Z}_2[x] / m(x)\) where \(m(x) = x^3 + x^4 + x^3 + x +1\).
Hence the residue classes module \(m(x)\)
are}{105) Consider the field \textbackslash mathbb\{Z\}\_2{[}x{]} / m(x) where m(x) = x\^{}3 + x\^{}4 + x\^{}3 + x +1. Hence the residue classes module m(x) are}}\label{consider-the-field-mathbbz_2x-mx-where-mx-x3-x4-x3-x-1.-hence-the-residue-classes-module-mx-are}}

\[
\bar{b(x)} = \bar{b_7 x^7 + b_6 x^6 + \dots + b_1x + b_0} 
\] and can be identified with a byte \(b_7b_6 \dots b_1b_0\). Compute
the sum of the two bytes \(10010101\) and \(11001100\) in the field. \[
\begin{aligned}
&10010101\\
+&11001100\\
\hline
&01011001
\end{aligned}
\] no carry by addition over polynoms and xor since \(\mathbb{Z}_2\).
Thus \(x^6 + x^4 + x^3 +1\) mod \(m(x)\).

\hypertarget{show-that-the-repetition-code-of-order-r-i.e.-each-bit-of-the-original-word-is-sent-r-times-is-a-linear-code.-determine-a-generator-matrix-and-a-check-matrix-of-this-code.}{%
\paragraph{\texorpdfstring{106) Show that the repetition code of order
\(r\) (i.e.~each bit of the original word is sent \(r\) times) is a
linear code. Determine a generator matrix and a check matrix of this
code.}{106) Show that the repetition code of order r (i.e.~each bit of the original word is sent r times) is a linear code. Determine a generator matrix and a check matrix of this code.}}\label{show-that-the-repetition-code-of-order-r-i.e.-each-bit-of-the-original-word-is-sent-r-times-is-a-linear-code.-determine-a-generator-matrix-and-a-check-matrix-of-this-code.}}

We show \(C\) is a linear code by defining a generating matrix of
dimensions \(k \times r \cdot k\) for it:

\begin{bmatrix}
1^r & 0^r & 0^r & \dots \\
0^r & 1^r & 0^r & \dots \\
0^r & 0^r & 1^r & \dots \\
\vdots & \vdots & \vdots & \ddots
\end{bmatrix}

and the check matrix of dimensions \((r-1)k \times (r\cdot k)\) by:

\begin{bmatrix}
m & m_0 & m_0 & \dots \\
m_0 & m & m_0 & \dots \\
m_0 & m_0 & m & \dots \\
\vdots & \vdots & \vdots & \ddots
\end{bmatrix}

where each \(m,m_0\) represents a block of size \((r-1)\cdot r\) and
\(m_0\) is defined by a block of only zeros while \(m\) is defined by:

\begin{matrix}
1 & 1 & 1 & \dots \\
-1 & 0 & 0 & \dots \\
0 & -1 & 0 & \dots \\
0 & 0 & -1 & \dots \\
\vdots & \vdots & \vdots & \ddots
\end{matrix}

each of these blocks represents a system of equations which check all
\(r\) bits are the same, i.e.~bit \(1\) is the same as bit \(2\) etc.

\hypertarget{four-symbols-have-to-be-encoded-with-elements-of-mathbbz_25-such-that-the-code-forms-a-5k-linear-code-k-to-be-determined-with-which-1-bit-errors-can-be-detected-and-corrected.-determine-a-generating-matrix-and-a-check-matrix-of-this-code.}{%
\paragraph{\texorpdfstring{107) Four symbols have to be encoded with
elements of \(\mathbb{Z}_2^5\) such that the code forms a \((5,k)\)
linear code (\(k\) to be determined) with which \(1\)-bit errors can be
detected and corrected. Determine a generating matrix and a check matrix
of this
code.}{107) Four symbols have to be encoded with elements of \textbackslash mathbb\{Z\}\_2\^{}5 such that the code forms a (5,k) linear code (k to be determined) with which 1-bit errors can be detected and corrected. Determine a generating matrix and a check matrix of this code.}}\label{four-symbols-have-to-be-encoded-with-elements-of-mathbbz_25-such-that-the-code-forms-a-5k-linear-code-k-to-be-determined-with-which-1-bit-errors-can-be-detected-and-corrected.-determine-a-generating-matrix-and-a-check-matrix-of-this-code.}}

We want \(d=3\) in order to be able to correct \(1\) error.

\[
G = \begin{pmatrix}
0 & 1 & 1 & 0 & 1\\
1 & 0 & 0 & 1 & 1
\end{pmatrix}
\]

\[
H = \begin{pmatrix}
1 & 1 & 0 & 1 & 0\\
1 & 0 & 1 & 1 & 0\\
-1 & 0 & -1 & 0 & 1\\
-1 & -1 & 0 & 0 & 1\\
0 & 0 & 0 & -1 & -1
\end{pmatrix}
\] from the system of equations: \[
b_1 + b_2 = b_3 + b_4\\
b_1 = b_4\\
b_2 = b_3\\
b_1 + b_2 = b_5\\
b_3 + b_4 = b_5
\]

\end{document}
